\section{Application}

In this section we will elaborate on how using e-hypergraphs for lambda calculus with explicit substitution may be more beneficial than (slotted) e-graphs.


While the work of~\cite{slotted-egraphs} has made the variables and binders into first class citizens of an e-graph, substitution and $\beta$-reduction in the context of $\lambda$-calculus remain challenging.
The issue with substitution is that despite it acting on a small part of the term, it results in duplication of large portions of an e-graph which is practically undesirable.
While it is possible to define substitution as a built-in operation in a slotted e-graph, in practice, using explicit substitution is more appropriate as it allows for a finer grained control over the duplication.
Even more so, explicit substitution becomes mandatory if we want to encode lambda calculus with explicit substitution in an e-graph.
Below is the Table~\ref{tbl:slotted} that features the number of e-nodes when performing equality saturation for a $\beta$-reduction using a lambda term $plus 2 2$ using Church encoding of arithmetic.

\begin{table}
    \begin{tabular}{lcccc}
      Set of equations & built-in subst & explicit subst & explicit subst & explicit subst\\
      &&& + interchange&  + interchange\\
      &&&&  + comm\\
      \# of e-nodes & 40 & 161 & 289 & 356
    \end{tabular}
    \caption{Slotted e-graphs benchmarks}
    \label{tbl:slotted}
\end{table}

The first row shows which rewrite rules were enabled, and the second row features the number of e-nodes after the e-graph is saturated (the initial term contained 16 e-nodes).
The first column used the built-in implementation of substitution that does not use the explicit substitution.
As it could be seen from the table, saturating the e-graph with respect to explicit substitution equations introduces a huge overhead (as the number of e-nodes that actually contain the result of a substitution is 40 as per the second column).
These equations, however, are absorbed by string diagrammatic (e-hypergraph) representation. Therefore, it is theoretically possible to achieve the size efficiency of built-in substitution without sacrificing any equations, i.e. providing complete information of all the redexes contained in a lambda term with explicit substitutions.

\textcolor{red}{Put the string diagrammatic encoding of substitution here?}