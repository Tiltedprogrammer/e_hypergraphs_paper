\section{Introduction}%
\label{sec:introduction}

\subsection{Motivation and contribution}

E-graphs (short for \emph{equality graphs})~\cite{EggPaper} are an efficient data structure for non-destructive rewriting, incorporating equivalence representation, used in automated theorem proving, compiler optimisation, and program analysis.
They mitigate the \emph{phase-ordering problem} of term rewriting: the way rewrites are scheduled may enable or block further rewrites due to introduction or elimination or redexes, resulting in a suboptimal result.
This is achieved by making rewrites on e-graphs non-destructive: an application of a rewrite rule, instead of replacing the left-hand side by the right-hand side, adds the latter to the e-graph so that all pre-existing redexes are protected and new ones are introduced.
Explosive growth of the e-graph is avoided by the clever use of sharing of vertices.

E-graphs were first introduced in the automated theorem proving literature in the 1980s~\cite{nelson1980techniques}, but recent years have seen renewed interest, particularly for optimisation and synthesis using the technique of \emph{equality saturation}~\cite{10.1145/1594834.1480915, flatt2022_small, EggPaper,flatt_small_2022}, wherein rewrites to an e-graph are repeated until a fixpoint (or timeout) is reached.
The wide range of application domains led to extensions of the e-graph formalism, for instance to account for conditional rewriting~\cite{singher2023colored} or  to combine e-graphs with database queries (\emph{relational e-matching})~\cite{zhang_relational_2022}.
We are interested in using e-graphs in the context of functional programming languages, which builds upon an internal representation of the $\lambda$-calculus.
Pioneering work in this direction has been done by~\citet{koehler2022sketchguided}, where the authors outlined the two interconnected challenges of binding and capture-avoiding substitution that arise when applying equality saturation to an encoding of the $\lambda$-calculus in e-graphs.

In the textual representation of $\lambda$-terms, $\alpha$-equivalent terms are distinct and thus cannot be shared, which adds a layer of bookkeeping and complexity.
This is traditionally handled by using de Bruijn indices~\cite{de1972lambda}, for which $\alpha$-equivalent terms are syntactically identified; however, this creates issues because the same variable can be assigned different indices depending on its scope with respect to the instance of that variable, again interfering with sharing.
As a simple example, consider the term $\lambda f . f ((\lambda x . f x) 2)$ which, with de Bruijn indices,  is encoded as $\lambda . \%0 ((\lambda . \%1 \%0) 2)$.
The two different occurrences of the same bound variable $f$ are represented differently by $\%0$ and $\%1$, which cannot be shared in an e-graph.

A simple extension of $\lambda$-calculus is given by adding \emph{explicit substitution}, which models \textbf{let}-bindings in programming languages.
Here, a term equipped with explicit substitutions $t[y/v][x/u]$ (modelling $\textbf{let} \; x = u \; \textbf{in} \; \textbf{let} \; y = v \; \textbf{in} \; t$) can be identified with the corresponding term where the order of substitution is reversed $t[x/u][y/v]$, supposing the substitutions are non-interacting.
This imposes a \emph{graphical equivalence}~\cite{accattoli2014nonstandard} of terms.
Modelling this in the conventional language of e-graphs requires the introduction of additional nodes to represent the explicit substitutions, and additional rewrite rules which capture graphical equivalence.
This pollutes the state space of equality saturation for these languages: the interesting rewriting is that of the user-specified algebraic theory one wishes to model, not between semantically equivalent but syntactically distinct forms in the language.

All these issues can be resolved by the use of \emph{string diagrams} in place of textual syntax.
\citet{ghica2024equivalencehypergraphsegraphsmonoidal}~shows how e-graphs (without binding) can be given a categorical semantics using symmetric monoidal semilattice-enriched categories, which allows e-graphs to be reconstructed as string diagrams.
In this paper, we extend this technique to e-graphs with bindings, lifting to monoidal \emph{closed} categories.

\subsection{String diagrams}

String diagrams are a direct bridge from the language of category theory, which in turn can be used to give semantic models to a variety of programming languages and logics, to the concrete language of graphs, which are a fundamental data structure in algorithm design.
These categorical techniques provide insight in how to solve the problems of binding and capture-avoiding substitution in e-graphs, as we develop in this paper.

The general background literature for string diagrams and their relationship to formal languages is summarised by~\citet{piedeleu2023introductionstringdiagramscomputer}.
The particular context of string diagrams with bindings is given in~\citet{ghica2024stringdiagramslambdacalculifunctional}, while string diagrams for e-graphs without bindings are defined in~\citet{ghica2024equivalencehypergraphsegraphsmonoidal}.

String diagrams provide a topological framework for understanding monoidal categories, representing objects as \emph{strings} and morphisms as \emph{nodes}, where input strings flow in and output strings flow out.
In this article, our string diagrams are oriented left-to-right, bottom-to-top to make the pictures look more similar to those commonly drawn for e-graphs.
The structure of symmetric monoidal categories is absorbed into the graphical syntax, with the composition and tensoring represented by juxtaposition, and symmetries as weaving of strings.
This visual approach makes hides the bureacracy in complex operations and reveals insights sometimes obscured by traditional one-dimensional symbolic notation.
Originating from notations by~\citet{hotzsd} (for automata) and~\citet{penrose1984spinors} (for tensor networks), extensions of string diagrammatic calculi exist for various forms of monoidal categories and even bicategories~\cite{Selinger_2010}, and provide a mathematically rigorous reasoning toolkit~\cite{joyal_geometry_1991}.

Commonly, string diagrams are used informally to provide a two-dimensional syntax for the internal language of a category, but for our purposes we must be more formal.
Specifically, we require a concrete encoding of string diagrams, analogous to the de Bruijn index representation of $\lambda$-terms.
For this, we use \emph{hypergraphs}~\cite{bonchi_string_2022-1}, which have nice syntactic characteristics.
For instance, hypergraph isomorphism captures when string diagrams \enquote{look the same} --- topologically, that one may be deformed continuously into the other without obstruction\footnote{This is the informal notion of the \enquote{isotopy of string diagrams}.}; secondly, string-diagrammatic equational reasoning can be formalised as (categorical) graph rewriting over these hypergraphs using the technique of double-pushout rewriting (DPO)~\cite{bonchi_string_2022-1}.

Finally, for the diagrammatic representation of two concepts essential to both programming languages (variable binding) and to e-graphs (equivalence classes) we rely on the concept of \emph{functorial boxes} as introduced by Melli\`es~\cite{10.1007/11874683_1}.
The diagrammatic boxes require the introduction of further structure to hypergraphs in the form of a hierarchical layering of the graphs (\emph{hierarchical hypergraphs})~\cite{fscd}.

\begin{figure}
	\centering
	\adjustbox{scale=0.6}{
		\tikzfig{figures/de-brujin-string}
	}
	\caption{String-diagrammatic representation of $\lambda f . f ((\lambda x . f x) 2)$.}\label{fig:de-brujin-string}
\end{figure}

Revisiting the earlier example $\lambda f . f ((\lambda x . f x) 2)$, in~\autoref{fig:de-brujin-string}, observe that the two occurrences of $f$ are now shared.
$\lambda$-abstraction is represented by a rounded box with the bound variable a dangling wire exiting the frame along its vertical boundary, the application node a half-circle
\adjustbox{scale=0.15}{\begin{tikzpicture}
		\begin{pgfonlayer}{nodelayer}
			\node [style=none] (0) at (1, 6.75) {};
			\node [style=none] (1) at (1.25, 7.75) {};
			\node [style=none] (2) at (2, 8) {};
			\node [style=none] (3) at (2.75, 7.75) {};
			\node [style=none] (4) at (3, 6.75) {};
			\node [style=none] (5) at (1.5, 6.75) {};
			\node [style=none] (6) at (2.5, 6.75) {};
			\node [style=none] (7) at (2, 9) {};
			\node [style=none] (8) at (1.5, 5.75) {};
			\node [style=none] (9) at (2.5, 5.75) {};
		\end{pgfonlayer}
		\begin{pgfonlayer}{edgelayer}
			\draw [style=lambda_unit_string] (3.center)
			to [in=60, out=-30, looseness=0.75] (4.center)
			to (0.center)
			to [in=-150, out=120, looseness=0.75] (1.center)
			to [in=-180, out=45, looseness=0.75] (2.center)
			to [in=135, out=0, looseness=0.75] cycle;
			\draw (2.center) to (7.center);
			\draw (5.center) to (8.center);
			\draw (9.center) to (6.center);
		\end{pgfonlayer}
	\end{tikzpicture}
},
and sharing a \emph{copy node} \comonoid.

\subsection{E-graphs}

E-graphs can efficiently simultaneously represent several equivalent terms of an algebraic theory, generalising the \textit{directed acyclic graph} (DAG) representation of terms to include equivalence classes of subterms (or subgraphs).
The key concepts of e-graphs can be intuitively grasped from some simple examples: \autoref{fig:e-graph-example} shows how a series of rewrite rules can be non-destructively applied to an e-graph~\cite{EggPaper}.

\begin{figure*}[h!]
	\[
		\adjustbox{scale=0.6}{
			\tikzfig{figures/e-graph-example-2}
		}
	\]
	\caption{E-graph example (top) and its equivalent string diagram representation (bottom)~\cite{ghica2024equivalencehypergraphsegraphsmonoidal}}
	\label{fig:e-graph-example}
\end{figure*}

Each column represents a term (or term rewrite rule), its conventional e-graph, and the equivalent string diagram representation using the \emph{e-hypergraphs} of~\citet{ghica2024equivalencehypergraphsegraphsmonoidal}.
For the initial term, $(a*2)/2$, which is already represented efficiently in the e-graph by sharing the node $2$, the string diagram version shares this node explicitly, using a copy node.

E-graphs consist of two types of nodes: \emph{e-classes}, representing equivalence classes of terms and denoted by dashed boxes, containing \emph{e-nodes}, representing term formers.
E-nodes reference e-classes instead of other e-nodes, in contrast to a traditional term graph.
By making a sequence of choices for e-nodes in e-classes, we can extract terms from the graph; intuitively, a different sequence of choices results in an equivalent term.
A term graph is embedded as an e-graph by an e-graph where each e-class has a single e-node.
The first rewrite rule, replacing multiplication by \emph{shift-left} creates the first non-singleton e-class, which includes the multiplication ($*$) and shift-left ($<\!\!<$) nodes, now sharing the node $a$.
This second e-graph illustrates the other representational difference, besides sharing, between conventional e-graphs and e-hypergraphs.
In the former, edges connect directly to e-nodes inside the e-class, whereas in the latter edges connect to the equivalence class itself.
Explicit deletion nodes are depicted with black dots with a single incoming wire
\scalebox{0.4}{
	\begin{tikzpicture}
		\begin{pgfonlayer}{nodelayer}
			\node [style=vertex] (0) at (1.5, 3.5) {};
			\node [style=none] (1) at (1.5, 3) {};
		\end{pgfonlayer}
		\begin{pgfonlayer}{edgelayer}
			\draw (0) to (1.center);
		\end{pgfonlayer}
	\end{tikzpicture}
}
inside the dashed boxes, indicating which of the equivalence-class-level edges are connected to which nodes inside the equivalence class.

The other steps represent further elaborations of the e-graph, or e-hypergraph, via the application of more rewrite rules.

\subsection{Conventional vs.\ slotted vs.\ categorical e-graphs with binding}%
\label{sec:vs-e-graphs-with-binding}

First, let us compare how substitution works in conventional e-graphs as compared to our proposed approach.

A common way of dealing with $\beta$-reduction for the $\lambda$-calculus in e-graphs is by representing the rules of $\lambda$-calculus explicitly, using special nodes for explicit substitution and associating them with particular rewrite rules~\cite{EggPaper,koehler2022sketchguided}.
The auxiliary nature of these nodes is evident in that, after rewriting, they may be removed from the graph.
We will see how in our categorical representation these nodes are not necessary, because explicit substitution is simply modelled categorically as composition, a significant simplification.

In~\autoref{fig:e-graph-substitution} we show successive iterations of a conventional e-graph for $(\lambda x . (y + y) + x) 1$ following $\beta$-reductions.
A subterm for an explicit substitution must be introduced, which is deemed equivalent to the top-level application term (hence they are inside the same dashed box).
The substitution is then propagated through child e-classes using additional rewrite rules (depicted at the top of each column).
At the end, all explicit substitution nodes can be removed.
The same example is then given again in~\autoref{fig:e-string-substitution}, this time using string diagrams.

Using string diagrams, $\beta$-reduction is simply the re-wiring of the diagram where an application node meets an abstraction, and explicit substitution simplifies into mere composition.
The explicit occurrences of variable names in the conventional approach raise obvious questions of scope, binding and explicit substitution, whereas the use of terms inside the explicit substitution nodes also raises many technical difficulties which makes proving various mathematical properties essentially intractable.
Additionally, explicit substitutions need to be propagated recursively throughout the graph, which adds another layer of technical and conceptual overhead.
Using de Bruijn indices instead when performing $\beta$-reduction would only push the problem elsewhere, as it greatly increases the complexity of representing the reduction as a rewrite rule, and special shifting operators need to be introduced on the right-hand side of a rewrite rule, which is a meta-syntactic operation~\cite{koehler2022sketchguided}.
Recent work on \textit{slotted} e-graphs~\cite{slotted-egraphs} uses nominal techniques to address the sharing issues when representing terms that are equivalent up to renaming of variables (even free ones) by incorporating variables as first-class data of an e-graph and making binding a built-in operation.
They also solved the problem of explicit variable names, guaranteeing freshness inside binders, explicitly differentiating between bound and free variables.
An example of a slotted e-graph is given in~\autoref{fig:slotted}.

\begin{figure}[t!]
	\begin{subfigure}[t]{0.45\linewidth}
		\[
			\adjustbox{scale=0.8}{
				\tikzfig{figures/slotted-e-graph-example}
			}
		\]
		\caption{Slotted e-graph}%
		\label{fig:slotted}
	\end{subfigure}
	\hfill
	\begin{subfigure}[t]{0.45\linewidth}
		\[
			\adjustbox{scale=0.6}{
				\tikzfig{figures/slotted-string-2}
			}
		\]
		\caption{Closed e-hypergraph}%
		\label{fig:closed}
	\end{subfigure}
	\caption{Extensions of e-graphs to handle binding in $\lambda x . (a + b) + (x + c)$.}%
	\label{fig:extended-egraphs}
\end{figure}

In slotted e-graphs, every e-class is parameterised by a set of free variables of the terms it represents.
The edges then additionally convey the information about how variables are instantiated.
For example, in~\autoref{fig:slotted}, an e-class containing $+$ that is parameterised by variable names $s_2$ and $s_3$ gets instantiated once with $a,b$ (via $s_4 \mapsto a, s_2 \mapsto s_4$ and $s_5 \mapsto b, s_3 \mapsto s_5$) and once with $x,c$.
The fact that the variable $x$ is bound is evident from the corresponding e-class being parameterised by only three variables.
If we were to add another e-node such as $e + d$, there would be an edge added to an already existing e-class such that $s_1 \mapsto e$, $s_2 \mapsto d$.
We can achieve a similar degree of sharing (in the sense that we have the same number of nodes for $+$) using string diagrams by performing a closure conversion of $+$ (\autoref{fig:closed}).
Fundamentally, there is a difference in approach here: slotted e-graphs solve the problem of variable binding generically using nominal techniques, in a way that can be applied to arbitrary formal languages with binders (e.g. $\pi$-calculus), while ours utilises an embedding into the internal language of our categorical model, which resembles a kind of metatheoretic $\lambda$-calculus, for which we use the metatheoretic $\lambda$-abstraction to handle the binding of variables.
We defer a case study to~\autoref{sec:application}, where we argue that our approach has advantages if one is particularly interested in e-graphs for (variations on) $\lambda$-calculus, and, by extension, programming languages.
However, these constructions are not unrelated: our categorical model can help justify the congruences maintained by slotted e-graphs, which we further elaborate on in~\autoref{sec:categorical}.
We also note that there is also a difference in the scopes of these related works: \citet{slotted-egraphs} focus on the e-graph as a data structure, providing a full implementation of their slotted e-graphs, while we focus on the foundational categorical semantics of e-graphs with binding and their combinatorial representation.

% Do we need this section?
% \subsection{Lambda calculus with explicit substitutions and graphical equivalence of terms}

\begin{figure*}
	\begin{subfigure}{\linewidth}
		\[
			% \hspace{1.3cm}
			\adjustbox{width=\textwidth}{
				\tikzfig{figures/e-graph-substitution}
			}
		\]
		% \captionsetup{skip=0pt}
		\subcaption{E-graph explicit substitution example.}
		\label{fig:e-graph-substitution}
	\end{subfigure}
	\begin{subfigure}{\linewidth}
		\[
			% \hspace{1.3cm}
			\adjustbox{width=\textwidth}{
				\tikzfig{figures/e-string-substitution-2}
			}
		\]
		\subcaption{String diagrammatic substitution example.}\label{fig:e-string-substitution}
	\end{subfigure}
\end{figure*}
%   \begin{figure*}
%     \[
%         % \hspace{1.3cm}
%         \adjustbox{width=\textwidth}{
%         \tikzfig{figures/e-string-substitution-2}
%         }
%     \]
% 	\captionsetup{skip=0pt}
%     \caption{String diagrammatic substitution example.}
%     \label{fig:e-string-substitution}
%   \end{figure*}

\subsection{Contributions}

\citet{ghica2024equivalencehypergraphsegraphsmonoidal} shows that e-graphs over signature $\Sigma$ can be represented as morphisms of a free cartesian symmetric monoidal semilattice-enriched category over the same signature.
We argue that restricting the categorical domain to free \textbf{closed} symmetric monoidal semilattice-enriched categories gives rise to structures that naturally support binding and equivalence classes of morphisms.
That is, morphisms (string diagrams) of such category can both encode variable binding (using the closed structure) and equivalence (using the semilattice enrichment) between subterms (sub-diagrams).
We call these string diagrams, or more precisely, the combinatorial representation of them, \emph{closed e-hypergraphs}.
First, we define closed enriched $\Sigma$-terms which generate a theory with equivalences and bindings.
Then we define what e-hypergraphs are and provide the interpretation of closed enriched $\Sigma$-terms as cospans of e-hypergraphs (and by doing this we also formalise the corresponding string diagrams as concrete combinatorial objects).
Next we formulate the DPO rewriting for e-hypergraphs and show that this notion of rewriting is sound and complete with respect to rewriting of closed enriched $\Sigma$-terms modulo SMC equations.
We finish the paper with a description of what benefits the e-hypergraph representation may bring for equality saturation in the context of $\lambda$-calculus with explicit substitution compared to conventional (slotted) e-graphs.

% DRG: LIST THE MAIN DEFINITIONS AND THEOREMS AND WHAT THEY MEAN
