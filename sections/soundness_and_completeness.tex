% !TEX root = ../main.tex

\section{Soundness and Completeness}

In this section, we will build a correspondence between terms in a semilattice enriched $\textsf{PROP}^{+}(\Sigma)$ and morphisms in $\WellTypedMdaEcospans$.
% Similarly to $\textsf{PROP}^{+}(\Sigma)$ where terms are defined up to semilattice enrichment and SMC equations we will say that two morphisms $G$ and $F$ in $\WellTypedMdaEcospans$ are equal when $G \Rrightarrow{}_{\mathcal{S}} F$. 
% As SMC equations are absorbed by e-hypergraph representation we only quotient by $\mathcal{S}$. 
We will build a semilattice-enriched SMC functor 

\[
[-] : \textsf{PROP}^{+}(\Sigma) \to \sfrac{\WellTypedMdaEcospans}{\mathcal{S}}
\]
where $\sfrac{\WellTypedMdaEcospans}{\mathcal{S}}$ is $\WellTypedMdaEcospans$ quotiented by structural rewrite rules $\mathcal{S} = \{\langle [l], [r] \rangle, \langle [r],[l] \rangle\}$ for all $l = r$ in~\ref{fig:string-equations}.
By freeness, it suffices to define how the functor acts on generators.
For every $f : n \to m \in \Sigma$, $[f]$ is defined as below.

\[
\scalebox{0.75}{
\tikzfig{soundness_and_completeness/generator_translation}
}
\]


\begin{theorem}
    If $f = g$ in $PROP^{+}(\Sigma)$ then $[f] \Rrightarrow_{\mathcal{S}} [g]$ in $\WellTypedMdaEcospans$.
    \begin{proof}
        $f = g$ means that $f$ as a term can be turned into $g$ by applying SMC equations and the equations from~\ref{fig:string-equations}.
        Since $f_1 =_{\text{SMC}} f_2$ means that $[f_1] = [f_2]$ we will only consider the transformations induced by the equations from~\ref{fig:string-equations}.
        We will also note that every subterm of $f$ corresponds to a convex down-closed sub e-hypergraph of $[f]$.
        Then, proving the above statement boils down to showing that for every rewrite rule in $\mathcal{S}$ the corresponding pushout complement and pushout exist.
        Suppose, $f =_{r} g$ where $r$ is the top left equation from~\ref{fig:string-equations}.
        $[r]$ is then given below.
        \[
            \scalebox{0.45}{
                \tikzfig{combinatorial_semantics/structural_rule_pushout}
            }
        \]
        After we identify a convex down-closed image of the left-hand side of this rule within a graph $[f]$ we remove the image by keeping only the outermost nodes.
        Note that because the image of all $a$'s and $b$'s are outermost nodes, we are allowed to pick any monomorphism for our pushout complement.
        And for the same reason the pushout for the right square exists.
        Then we simply glue the right-hand side of the rule into the hole.
        Similar argument applies to other rules as well.
        The result for any number of applied equations follows by induction.
    \end{proof}
\end{theorem}

\begin{corollary}
    $\sfrac{\WellTypedMdaEcospans}{\mathcal{S}}$ is a semilattice-enriched SMC category and $[-]$ is a semilattice-enriched SMC functor.
\end{corollary}

\begin{proposition}
    For each cospan $A$ in $\sfrac{\WellTypedMdaEcospans}{\mathcal{S}}$ there is a week normal form such that $A =_{\mathcal{S}} A_1 + \ldots A_m$ where each of $A_i$ does not contain any hierarchical edges.
    \begin{proof}
    By applying the rules 1-6 in $\mathcal{S}$ (the ones from Figure~\ref{fig:string-equations}) we either remove 1 hierarchical edge (rules 5-6) or do not change the number of them, but expand the edge onto other edges.
    We will have a redex for such a rule until there is only one hierarchical edge at the outermost level (and only one outermost edge) with no other hierarchical edges within it.
    \end{proof}
\end{proposition}

\begin{proposition}
    Functor $\llbracket \cdot \rrbracket; \mathcal{I}$ is faithful.
    \begin{proof}
        Consider the commutative diagram below.

        % https://q.uiver.app/#q=WzAsNCxbMCwwLCJcXHRleHR7UFJPUH0oXFxTaWdtYSkiXSxbMiwwLCJcXHRleHR7UFJPUH1eeyt9KFxcU2lnbWEpIl0sWzAsMiwiXFxNZGFDb3NwYW5zIl0sWzIsMiwiXFxzZnJhY3tcXFdlbGxUeXBlZE1kYUVjb3NwYW5zfXtcXG1hdGhjYWx7U319Il0sWzAsMSwiXFxtYXRoY2Fse0p9Il0sWzAsMiwiXFxsbGJyYWNrZXQgXFxjZG90IFxccnJicmFja2V0IiwyXSxbMSwzLCJbXFxjZG90XSJdLFsyLDMsIlxcbWF0aGNhbHtJfSIsMl1d
        \[
        \begin{tikzcd}
            {\text{PROP}(\Sigma)} && {\text{PROP}^{+}(\Sigma)} \\
            \\
            \MdaCospans && {\sfrac{\WellTypedMdaEcospans}{\mathcal{S}}}
            \arrow["{\mathcal{J}}", from=1-1, to=1-3]
            \arrow["{\llbracket \cdot \rrbracket}"', from=1-1, to=3-1]
            \arrow["{[-]}", from=1-3, to=3-3]
            \arrow["{\mathcal{I}}"', from=3-1, to=3-3]
        \end{tikzcd}
        \]
    \end{proof}

    $\llbracket \cdot \rrbracket$ is faithfull by Corollary 4.3 in~\cite{Frobenius}.
    $\mathcal{J}$ is an inclusion that maps terms in $\textsf{PROP}(\Sigma)$ to SMC fragment of $\textsf{PROP}^{+}(\Sigma)$ and is automatically faithful. 
    Similarly, $\mathcal{I}$ is faithful because it is inclusion.
    Then, by composition, $\llbracket \cdot \rrbracket; \mathcal{I}$ is faithful. 
\end{proposition}

\begin{theorem}
$[-]$ is faithful.
\begin{proof}
Essentially we need to show that if $[f] = [g]$ then $f = g$ which is equivalent to if $f \not = g$ then $[f] \not = [g]$.
First, we note that we have a \question{weak (modulo commutativity)} normal form for any term \update{string diagram?} in $PROP^{+}(\Sigma)$ which is achieved by applying distributivity for tensor and composition as well as unpacking equations, that is, every term is representable as in Figure~\ref{fig:normal_form_string} where $\phi_1,\;\ldots\;,\phi_n$ are pure SMC terms.

\begin{figure}
    \[
    \scalebox{0.75}{
        \tikzfig{soundness_and_completeness/normal_form_string}
    }    
    \]
    \caption{Weak normal form of a term in $PROP^{+}(\Sigma)$}
    \label{fig:normal_form_string}
\end{figure}

Then we get a corresponding cospan for such weak normal form in $\sfrac{\WellTypedMdaEcospans}{\mathcal{S}}$ as in Figure~\ref{fig:normal_form_ehyp}

\begin{figure}
    
\[
\scalebox{0.75}{
\tikzfig{soundness_and_completeness/normal_form}
}
\]
\caption{Interpretation of the weak normal form in $\sfrac{\WellTypedMdaEcospans}{\mathcal{S}}$}
\label{fig:normal_form_ehyp}
\end{figure}

where all $A_1\; \ldots \;A_n$ do not contain any hierarchical edges.
We will write such forms as $\phi_1 + \ldots \phi_n$ and $A_1 + \ldots A_n$ respectively.
Now, assume that $f = \phi_1 + \ldots + \phi_n$ and $g = \psi_1 + \ldots + \psi_m$.
Then, by functoriality, $[f] = [\phi_1] + \ldots + [\phi_n] = A_1 + \ldots + A_n$ and $g = [\psi_1] + \ldots + [\psi_m] = B_{1} + \ldots + B_{m}$. 
$f \not = g$ gives us two options.

\begin{itemize}
    \item $n \not = m$. Then $[f] \not = [g]$ as they are simply not isomorphic cospans of e-hypegraphs as they have different number of summands.
    \item $n = m$. Then there exists a permutation $\sigma$ and index $i$ such that $\phi_i \not = \psi_{\sigma(i)}$. 
    Remember that these are just SMC terms and by faithfulness of $\llbracket \cdot \rrbracket;\mathcal{I}$ we get $A_i \not = B_{\sigma(i)}$ and $[f] \not = [g]$.
\end{itemize}

\end{proof}
\end{theorem}

\begin{theorem}
$[-]$ is full.
\begin{proof}
As we quotient by $\mathcal{S}$ instead of arbitrary cospans we can only consider their normal forms $A_1 + \ldots A_n$ where all of $A_i$ do not contain any hierarchical edges as per the proposition above.
Because each of $A_i$ is a plain hypergraph, they have a pre-image in $\textsf{PROP}(\Sigma)$, i.e. for all $i$ there exists $a_i$ in $\textsf{PROP}(\Sigma)$ such that $A_i = \llbracket a_i \rrbracket;\mathcal{I}$.
And then $[\mathcal{J}(a_1) + \ldots + \mathcal{J}(a_n)] = [\mathcal{J}(a_1)] + \ldots + [\mathcal{J}(a_n)] = A_1 + \ldots + A_n$ and hence $\mathcal{J}(a_1) + \ldots + \mathcal{J}(a_n)$ is the pre-image of $A_1 + \ldots + A_n$.
\end{proof}
\end{theorem}

% \begin{theorem}

%     Let $\mathcal{R}$ be a rewriting system on $\textsf{PROP}(\Sigma)$ and $\langle l, r \rangle \in \mathcal{R}$.
%     $a \Rightarrow{}^{+}_{\langle l, r \rangle} b$ iff $[a] \Rrightarrow{}^{+}_{\langle [l], [r] \rangle} [b]$.
%     \begin{proof}
%         $a \Rrightarrow^{+}_{\langle l, r \rangle} b$ means that either
%         \begin{itemize}
%             \item $a$ and $b$ are decomposable as
%                   \[
%                     \scalebox{0.75}{
%                         \tikzfig{combinatorial_semantics/prop_rewriting}
%                     }
%                   \]
%                   \question{Here we can probably just reuse the proof going to $\ulcorner a \urcorner$ and then the result follows by functoriality of $[\cdot]$ (or $\llangle \cdot \rrangle$)}
%             \item or
%                  \[
%                     \scalebox{0.75}{
%                         \tikzfig{combinatorial_semantics/prop_plus_rewriting}
%                     }   
%                  \]
%                  and $a' \Rightarrow_{\langle l,r \rangle} b'$. This gives us the following decomposition in cospans
%                  % https://q.uiver.app/#q=WzAsMjcsWzAsNCwibSJdLFsxLDQsIkFfMSJdLFsyLDQsImkgKyBrIl0sWzMsNCwiOyJdLFs0LDAsImkiXSxbNSwwLCJYXzEiXSxbNiwwLCJqIl0sWzUsMiwiXFx2ZG90cyJdLFs1LDQsIkEnIl0sWzQsNCwiaSJdLFs2LDQsImoiXSxbNSwxLCIrIl0sWzUsMywiKyJdLFs1LDUsIisiXSxbNSw2LCJcXHZkb3RzIl0sWzUsNywiKyJdLFs0LDgsImkiXSxbNSw4LCJYX3ciXSxbNiw4LCJqIl0sWzcsNCwiXFxvdGltZXMiXSxbOCw0LCJrIl0sWzksNCwiayJdLFsxMCw0LCJrIl0sWzExLDQsIjsiXSxbMTIsNCwiaiArIGsiXSxbMTMsNCwiQV8yIl0sWzE0LDQsIm4iXSxbMCwxXSxbMiwxXSxbNCw1XSxbNiw1XSxbOSw4XSxbMTAsOF0sWzE4LDE3XSxbMTYsMTddLFsyMCwyMV0sWzIyLDIxXSxbMjYsMjVdLFsyNCwyNV1d
%                 \[
%                     \adjustbox{scale=0.5, center}{
%                 % \scalebox{0.75}{
%                     \begin{tikzcd}
%                     &&&& i & {X_1} & j \\
%                     &&&&& {+} \\
%                     &&&&& \vdots \\
%                     &&&&& {+} \\
%                     m & {A_1} & {i + k} & {;} & i & {A'} & j & \otimes & k & k & k & {;} & {j + k} & {A_2} & n \\
%                     &&&&& {+} \\
%                     &&&&& \vdots \\
%                     &&&&& {+} \\
%                     &&&& i & {X_w} & j
%                     \arrow[from=5-1, to=5-2]
%                     \arrow[from=5-3, to=5-2]
%                     \arrow[from=1-5, to=1-6]
%                     \arrow[from=1-7, to=1-6]
%                     \arrow[from=5-5, to=5-6]
%                     \arrow[from=5-7, to=5-6]
%                     \arrow[from=9-7, to=9-6]
%                     \arrow[from=9-5, to=9-6]
%                     \arrow[from=5-9, to=5-10]
%                     \arrow[from=5-11, to=5-10]
%                     \arrow[from=5-15, to=5-14]
%                     \arrow[from=5-13, to=5-14]
%                 \end{tikzcd}}
%                 \]
%                 and there is a convex match of $L$ in $A'$.
%                 It is then also down-closed and conflict reflecting as $A'$ is not hierarchical.
%                 Need to show that arising pushout complement is a boundary complement.
%                 \question{A problem is that the above terms does not decompose into a form of $A';C$ where $C$ is a pushout complement.
%                 This happens because we take composition along the external interfaces and this $A'$ is within a box, so it is not a part of an external interface.
%                 }
%                 \Aleksei{discuss this. We can simplify PROP+ rewriting to just a box, but still have a problem with context C}
%         \end{itemize}
%         \update{It is tempting to reuse a similar result for $\llbracket \cdot \rrbracket$ because according to our definition $f \Rrightarrow_{\mathcal{R}^{+}}$ means that there is an SMC piece $f'$ inside some box that rewrites to $g'$.
%             But we only have this result for $\llangle \cdot \rrangle$ for terms with bent wires which is a functor to $PROP + Frob$.
%         }
%     \end{proof}
% \end{theorem}

% We will next consider functors
% \[
%     \textsf{PROP}(\Sigma, \mathcal{E}) \xrightarrow{\mathcal{J}} \textsf{PROP}^{+}(\Sigma, \mathcal{E}) \xrightarrow{\llangle - \rrangle} \sfrac{\WellTypedMdaEcospans}{\mathcal{S},\mathcal{E}}
% \]
% where $\mathcal{E}$ is a set of equations of the corresponding SMT.
% \begin{theorem}[$\llangle - \rrangle$ is a functor]
% \end{theorem}

% \begin{theorem}[$\llangle - \rrangle$ is faithful]
% \end{theorem}

% \update{Or we can define rewriting for string diagrams in $\textsf{PROP}^{+}(\Sigma)$ based on rewriting in $\textsf{PROP}(\Sigma)$ and then say that $f \xrightarrow{}_{\mathcal{R}} g$ iff $[f] [g]$}

\begin{definition}[Rewriting in $\textsf{PROP}^{+}(\Sigma)$]

Recall the definition of rewriting in some $\textsf{PROP}(\Sigma) = \mathbb{A}$.
We say that $a \in \mathbb{A}$ rewrites into $b \in \mathbb{A}$ under a rewrite rule $\langle l, r \rangle$ where $l,r \in \mathbb{A}$ if $a$ and $b$ are decomposable as below.
\[
\scalebox{0.75}{
\tikzfig{combinatorial_semantics/prop_rewriting}
}
\]
and we will use notation $a \Rightarrow_{\langle l, r \rangle} b$.

Now consider an enriched prop $\mathbb{A}^{+}$ over the signature $\Sigma$ such that there is a functor $\mathcal{J} : \mathbb{A} \to \mathbb{A}^{+}$.
We than say that $a \in \mathbb{A}^{+}$ rewrites into $b \in \mathbb{A}^{+}$ under a rewrite rule $\langle l, r \rangle$ where $l,r \in \mathcal{J}(\mathbb{A})$ if $a$ and $b$ are decomposable as below.

\[
\scalebox{0.75}{
\tikzfig{combinatorial_semantics/prop_plus_rewriting}
}
\]

and $a' \Rightarrow_{\langle l, r \rangle} b'$.
We will use notation $a \Rightarrow_{\langle l, r \rangle}^{+} b$.
\end{definition}

\begin{theorem}[Theorem 35~\cite{Frobenius2}]

    $a \Rightarrow_{R} b$ iff $\llangle {}^{\ulcorner} a {}^{\urcorner} \rrangle \Rrightarrow_{\llangle {}^\ulcorner R {}^\urcorner \rrangle} \llangle {}^{\ulcorner}b {}^{\urcorner} \rrangle$.
    Where $\llangle \cdot \rrangle : \textsf{PROP}(\Sigma) + \textsf{Frob} \to \catname{Csp_{D}(Hyp_{\Sigma})}$ and $\Rrightarrow$ is a convex rewriting relation for morphisms in $\catname{Hyp_{\Sigma}}$.
\end{theorem}

\begin{remark}
    If $\llbracket a \rrbracket = n \xrightarrow{} A \xleftarrow{} m$ then $\llangle ^{\ulcorner} A ^{\urcorner} \rrangle$ is isomorphic to $0 \xrightarrow{} A \xleftarrow{} n + m$.
\end{remark}

\begin{proposition}
    $\llangle {}^{\ulcorner} a {}^{\urcorner} \rrangle \Rrightarrow_{\llangle {}^\ulcorner R {}^\urcorner \rrangle} \llangle {}^{\ulcorner}b {}^{\urcorner} \rrangle$ iff $[\mathcal{J}(a)] \Rrightarrow_{[R]}^{+} [\mathcal{J}(b)]$.
     Note that $a$ and $b$ are from $\textsf{PROP}(\Sigma)$.
    \begin{proof}
        The forward direction follows by the remark above and by recalling that $\Rrightarrow$ implies there exists a pushout square below
        \[
        \scalebox{1}{
            \tikzfig{soundness_and_completeness/pushout_in_Hyp}
        }    
        \]
        where the matching $m$ is convex, $i+j \to C \to G$ is a boundary complement~\cite{Frobenius2} and marked squares are pushouts
        and by using the injection $\mathcal{K} : \catname{Hyp_{\Sigma}} \to \catname{EHyp_{\Sigma}}$ there exists the following pushout square
        \[
          \scalebox{0.75}{
            \tikzfig{soundness_and_completeness/pushout_in_EHyp}
          }  
        \]
        where $m$ is convex and down-closed (as there are no hierarchical edges) and $\mathcal{K}(i+j) \to C \to G$ is a boundary complement.
        This by definition means that $\mathcal{K}(n) \xrightarrow{} \mathcal{K}(n) \xrightarrow{} G \xleftarrow{} \mathcal{K}(m) \xleftarrow{} \mathcal{K}(m) = [\mathcal{J}(a)] \Rrightarrow_{[R]}^{+} = [\mathcal{J}(b)] = \mathcal{K}(n) \xrightarrow{} \mathcal{K}(n) \xrightarrow{} G \xleftarrow{} \mathcal{K}(m) \xleftarrow{} \mathcal{K}(m)$.
        The other direction follows by applying the same argument backwards.
    \end{proof}
\end{proposition}


\begin{theorem}
    $a \Rightarrow_{\langle l, r \rangle}^{+} b$ in $\textsf{PROP}^{+}$ iff $[a] \Rrightarrow_{\langle [l],[r] \rangle} [b]$ in $\sfrac{\WellTypedMdaEcospans}{\mathcal{S}}$.
    \begin{proof}
        $a \Rightarrow_{\langle l, r \rangle}^{+} b$ implies that $a$ is decomposable as $a_1 + \ldots + a' + \ldots + a_k$ and $b$ is decomposable as $a_1 + \ldots + b' + \ldots + a_k$ and $a' \Rightarrow_{\langle l, r \rangle} b'$.
        The latter implies by the Theorem 35~\cite{Frobenius2} that $\llangle \ulcorner a' \urcorner \rrangle \Rrightarrow_{\langle \llangle ^\ulcorner l ^\urcorner \rrangle, \llangle ^\ulcorner r ^\urcorner \rrangle \rangle} \llangle \ulcorner b' \urcorner \rrangle$ which means that $[a'] \Rrightarrow_{\langle [l], [r] \rangle}^{+} [b']$.
        This means that there exists a convex matching of $[l]$ within $[a']$ and hence there is a convex match of $[l]$ in $[a_1] + \ldots + [a'] + \ldots + [a_k]$ and that the pushout complement is $[a_1] + \ldots + C + \ldots + [a_k]$ where $C$ is the pushout complement for $[a'] \Rrightarrow_{\langle [l], [r] \rangle} [b']$.
        Similarly, the pushout is $[a_1] + \ldots [b'] + \ldots + [a_k] = [b]$.

        For the converse direction, $[a] \Rrightarrow_{\langle [l],[r] \rangle}^{+} [b]$ implies that $[a] = A_{1} + \ldots A' + \ldots + A_{n}$ and $[b] = A_{1} + \ldots B' + \ldots + A_{n}$ and $A' \Rrightarrow_{\langle [l],[r] \rangle}^{+} B'$ because $[l]$ and $[r]$ have a pre-image in $\textsf{PROP}(\Sigma)$ and because we quotient by $\mathcal{S}$.
        The latter means by the proposition above that $a' \Rightarrow_{\langle l, r \rangle} b'$ such that $A' = \llbracket a' \rrbracket; \mathcal{I}$ and $B' = \llbracket b' \rrbracket; \mathcal{I}$ and hence $a = a_1 + \ldots + \mathcal{I}(a') + \ldots + a_n \Rightarrow_{\langle l, r \rangle}^{+} a_1 + \ldots \mathcal{I}(b') + \ldots a_n = b$
    \end{proof}
    \update{It's again somewhat handwaving, do we need any supporing proofs for the above?}
\end{theorem}