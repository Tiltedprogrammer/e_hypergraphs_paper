% !TEX root = ../main.tex

\section{E-Rewriting for E-Hypergraphs}

The (convex) \textit{matching} of one e-hypergraph $L$ (\textit{e.g.}, the left-hand side of a rewrite rule) into another e-hypergraph $G$ was defined in the previous section as a monic morphism $L \to G$ (whose image is additionally convex, \update{consistent}). In other words, it corresponds to finding a subgraph isomorphic to $L$ within $G$ (subject to conditions respecting the hierarchical and conflict structures). 


However, in general we will have to explode the e-hypergraph, by applying the structural rules for (horizontal, vertical) congruence first, in order to find such a match. In this section, we define the  e-matching (and subsequent e-rewriting) as the problem of finding matches (and applying rewrites) within an e-hypergraph \textit{without} first inducing duplication of subgraphs. We begin first with definitions, before proceeding to prove them correct with respect to the plain notion of matching (rewriting). 

\begin{definition}[Projection]
The set of projections of an e-hypergraph $G$ are \update{project each e-box}.    
\end{definition}

\begin{definition}[E-matching]
We define the \textit{e-matching problem} as follows: given two e-hypergraphs \update{(with interfaces??)} $L$ and $G$, we say that a convex subgraph $G' \subseteq G$ is a \textit{e-match} for $L$ if there exists a projection of $G'$ which isomorphic to $L$.
\end{definition}

\update{What about refolding graph?}
\begin{lemma}[Correctness of e-matching]
$G' \subseteq G$ is an e-match for $L$ if and only if there is a sequence of structural rewrites $G \to^* H$ such that $L$ is a match in $H$.
\end{lemma}
\begin{proof}
The implication

The converse direction is trivial. 
\end{proof}

\begin{definition}[E-matching]
Given an e-match $G' \subset G$ of $L$, 
define its e-rewriting as the application of the DPO rewriting rule $G' \to G' + L$  to the match $L \to G$ induced by the ematching. \update{diagram? interfaces?}
\end{definition}

\begin{definition}[E-rewriting]

Given a rewrite rule $L \to R$ and an e-match $G' \subset G$ of $L$ in $G$, define the \textit{e-rewriting} of $L \to R$ in $G$ to $H$, written $G \Rightarrow_{L \to R}^e H$, where $H$ is defined as the result of applying DPO-rewriting of the rewrite $G \to G+R$ to the match $L$ in $G$. \update{interfaces.}
\end{definition}

In order to state the correctness of e-rewriting, we first require the notion of \textit{maximal unfolding} of an e-hypergraph to a normal form. 
\begin{definition}
Given an e-hypergraph $G$, there exists a sequence of structural rewrites transforming it to the form $G_1 + \ldots + G_n$, where each $G_i$ is a plain hypergraph. This is called its \textit{sharing normal form}.
\end{definition}
\update{Above should be lemma then defn.}


\begin{lemma}[Correctness of e-rewriting]
Statement?
\end{lemma}

In the case of plain string diagrams represented as (mda-)hypergraphs, the problem of finding a matching of the left-hand side of a rewrite rule % is essentially the (induced??) sub-hypergraph isomorphism problem [up to interfaces, and convexity?]


Informally, for every e-box $e = \phi_1 + \ldots + \phi_n$ appearing in a string diagram $\psi$ as $\psi\{\phi_1 + \ldots + \phi_n\}$ we can define the $i$th projection of $e$ in $\psi$ as $\pi^e_i = \phi\{\psi_i\}$. 