\section{Soundness and Completeness}\label{sec:soundness-and-completeness}

This section contains the main technical results of the paper,  building on those of \cite{bonchi_string_2022-2}.
We give the interpretation of $\catname{S}(\Sigma)^{+}$ terms in $\WellTypedMdaEcospans$ by extending the interpretation of $\catname{S}(\Sigma)$ in $\MdaCospans$ by first endowing each hom-set of cospans in $\WellTypedMdaEcospans$ with a semilattice structure turning it into a semilattice-enriched category.
To achieve the latter we quotient each hom-set by EDPOI structural rewrite rules that correspond to the axioms of a semilattice-enriched SMC:
as expected,  the SMC equations are factored out in our representation of terms in $\WellTypedMdaEcospans$,  but the semilattice equations are not.  

% We first construct an interpretation of terms generating $\catname{S}(\Sigma)^{+}$ in $\WellTypedMdaEcospans$.  
% As expected,  the SMC equations are factored out in our representation of terms in $\WellTypedMdaEcospans$,  but the semilattice equations are not.  
% Instead,  we implement the semilattice equations via EDPOI rewrites,  leading to the following soundness and completeness result: for any morphisms $f$ and $g$ in $\catname{S}(\Sigma, \mathcal E)$,  $f = g$  if and only if there exists a sequence of EDPOI-rewrites --- each induced by either a structural semilattice equation or $\mathcal E$ --- between their combinatorial representations. 

\begin{definition}
We define $+$ of two extended cospans in $\WellTypedMdaEcospans(n,m)$ as in Figure~\ref{fig:A+B}
by introducing a hierarchical edge with $n$ inputs and $k$ outputs that has both carriers of the two cospans as its successors.
The two carriers become two different consistent components of a newly introduced hierarchical edge such that the interface morphisms are preserved.
In particular, $f_{1,\text{int}}(b_i) = f_{\text{int}}(b_i)$ and so on.
The only extended cospan of type $0 \to 0$ has the empty e-hypergraph as its carrier and we will overload $+$ for such cospans to have the empty carrier as well.
That is, we will let 
\begin{align*}
	0 \to 0 \to \;&\varnothing \xleftarrow{} 0 \xleftarrow{} 0\\
	&\;+ \hspace{6em} \Coloneqq \hspace{2em} 0 \to 0 \to \varnothing \xleftarrow{} 0 \xleftarrow{} 0\\
	0 \to 0 \to \;&\varnothing \xleftarrow{} 0 \xleftarrow{} 0
\end{align*}
\end{definition}

To make this newly defined $+$ obey the laws of a semilattice-enriched SMC we will quotient $\WellTypedMdaEcospans$ by certain EDPOI structural rewrite rules.

We define the appropriate notion of quotient by EDPOI rewrites below.  
% Let the function $\llbracket - \rrbracket$ extend in the obvious way to apply to equations (\textit{i.e.}, pairs of $\Sigma^+$-terms).
% Given a set of equations $\mathcal{E}$ of $\Sigma^{+}$-terms, $\llbracket \mathcal{E} \rrbracket$ is defined as $\{\langle \llbracket l \rrbracket, \llbracket r \rrbracket \rangle \text{ for each } l = r \in \mathcal{E} \} \cup \{\langle \llbracket r \rrbracket, \llbracket l \rrbracket \rangle \text{ for each } l = r \in \mathcal{E}\}$.
% \update{below definition depends on whether we require the homomorphism to reflect the conflicts}
\begin{definition}[Quotient by rewrites]  
Given a set of EDPOI rewrite rules $\mathcal{E}$,  we denote by $\WellTypedMdaEcospans/\mathcal{E}$ the category $\WellTypedMdaEcospans$ quotiented by the following relation.
\[
	f \sim g \quad \text{if} \quad f \Rrightarrow^{*}_{\mathcal{E}} g ~ . 
\]
\end{definition}


\begin{figure}
    \[
    \scalebox{0.55}{
    \tikzfig{../figures/combinatorial_semantics/f_plus_g_new}
    }
    \]
    \captionsetup{belowskip=-3.5ex}
    \caption{$+$ of two morphisms in $\Ecospans$}
    \label{fig:A+B}
\end{figure}
% Recall from Section \ref{sec:e-hypergraphs} that $ \WellTypedMdaEcospans$ has a symmetric monoidal structure inherited from the coproduct structure of $\textbf{EHypI}(\Sigma)$.  
% Recall further that typed $\Sigma^+$-terms are those categorical combinator terms freely constructed from generators $c \in \Sigma$, $\textsf{id}_I$, $\textsf{id}_1$, $\sym_{1,1}$, $(;\!)$ and $\otimes$,  and $f+g$.  
% We will write $\Sigma^+(n,m)$ for the set of $\Sigma^+$-terms of type $n \to m$.  

\begin{definition}
	Let $\mathcal{S}$ be the set of EDPOI rewrite rules (or rather schema rules) that correspond to string diagrammatic equations in Figure~\ref{fig:string-equations}.
\end{definition}

For example, the first equations would yield two EDPOI schema rules (one for each reading the span from left to right and from right to left):

\[
\adjustbox{scale=0.4}{
	\tikzfig{../figures/soundness_and_completeness/semilattice_rule_1}~.
}
\]
Also, we do not need to introduce rewrite rules for commutativity equations as these are absorbed by our notion of isomorphism~\ref{def:iso}.
Note that not all pushouts (or pushout complements) in $\textbf{MEHypI}(\Sigma)$ are guaranteed to exist: however,  pushouts (and pushout complements) for rewrites for the rules we are interested in do exist~\ref{proof:appendix:soundness} and thus the quotienting is well-defined.

% We wish to specify a set of EDPOI rewrite rules corresponding to those in Figure \ref{fig:string-equations}.
% In order to do this,  we specify a function which maps  uninterpreted $\Sigma^+$-terms into $\WellTypedMdaEcospans$ morphisms,  and which also extends the expected interpretation of $\Sigma$-terms in $\MdaCospans$ (as given by the free functor from $\catname{S}(\Sigma)$ induced by the interpretation of $\Sigma$ as itself,  detailed in \cite{bonchi_string_2022-2}). 
% \begin{definition}[Interpretation function] 
% %Let $\llbracket - \rrbracket: \textbf{PROP}(\Sigma) \to \WellTypedMdaEcospans$ be the unique SMC-functor induced by the interpretation sending $\Sigma$ to itself.  
% Let $\llbracket - \rrbracket_{n,m}: \Sigma^+(n,m) \to \WellTypedMdaEcospans(n,m)$ be the function defined by induction on the type derivation of $\Sigma$-terms by 
% \begin{align*}
% 	\llbracket \id \rrbracket_{n,m} &= \id  &	\llbracket f ; g \rrbracket_{n,m} &= \llbracket f \rrbracket_{n,m} ; \llbracket g \rrbracket_{n,m} \\
% 	\llbracket I \rrbracket_{n,m} &= I & \llbracket f \otimes g \rrbracket_{n,m} &= \llbracket f \rrbracket_{n,m} \otimes \llbracket g \rrbracket_{n,m} \\
% 	\llbracket \sym \rrbracket_{n,m} &= \sym  & \llbracket c \rrbracket_{n,m} &= c   \\ 
% 	\llbracket f+g \rrbracket_{n,m} &= \llbracket f \rrbracket_{n,m} + \llbracket g \rrbracket_{n,m}
% \end{align*}
% where $c \in \Sigma$ and $+$ in $\WellTypedMdaEcospans$ is defined by Figure \ref{fig:A+B}.  It uses the notation of dashed boxes for hierarchical edges, with its children drawn inside,  and maximally consistent components separated by another dashed line.  We will omit the subscripts in $\llbracket - \rrbracket_{n,m}$ where it is clear from context.
% The definitions of the right-hand sides of the base cases above can be found in~\cite{bonchi_string_2022-1} that defines the corresponding cospans of hypergraphs. One can then get extended cospans by following~\ref{remark:embedding_functor}.
% \end{definition}

% \begin{remark}
% \label{remark:f+g}
% The action of the operation $+$ defined in Figure~\ref{fig:A+B} on two extended cospan is given by introducing a hierarchical edge with $n$ inputs and $k$ outputs that has both carriers of the two cospans as its successors and where the carriers are different equivalence classes under $\consistency$ as implied by the dashed delimiter.
% The only extended cospan of type $0 \to 0$ has the empty e-hypergraph as its carrier and we will overload $+$ for such cospans to have the empty carrier as well.
% That is, we will let 
% \begin{align*}
% 	0 \to 0 \to \;&\varnothing \xleftarrow{} 0 \xleftarrow{} 0\\
% 	&\;+ \hspace{6em} \Coloneqq \hspace{2em} 0 \to 0 \to \varnothing \xleftarrow{} 0 \xleftarrow{} 0\\
% 	0 \to 0 \to \;&\varnothing \xleftarrow{} 0 \xleftarrow{} 0
% \end{align*}
% \end{remark}

% In order to phrase our soundness and completeness results categorically,  we define the following quotient of $\WellTypedMdaEcospans$ by EDPOI rewrites.  
% This means that soundness is expressed as the extension of $\llbracket - \rrbracket$ to a semilattice-enriched SMC functor from $\textbf{S}(\Sigma, \mathcal E)^{+}$,  and completeness is expressed as the faithfulness of this functor.  
% In fact,  we prove that the functor is also full,  giving an equivalence of categories. 

% We define the appropriate notion of quotient by EDPOI rewrites below.  
% Let the function $\llbracket - \rrbracket$ extend in the obvious way to apply to equations (\textit{i.e.}, pairs of $\Sigma^+$-terms).
% Given a set of equations $\mathcal{E}$ of $\Sigma^{+}$-terms, $\llbracket \mathcal{E} \rrbracket$ is defined as $\{\langle \llbracket l \rrbracket, \llbracket r \rrbracket \rangle \text{ for each } l = r \in \mathcal{E} \} \cup \{\langle \llbracket r \rrbracket, \llbracket l \rrbracket \rangle \text{ for each } l = r \in \mathcal{E}\}$.
% \update{below definition depends on whether we require the homomorphism to reflect the conflicts}
% \begin{definition}[Quotient by rewrites]  
% Given a set of equations $\mathcal{E}$,  we denote by $\WellTypedMdaEcospans/\mathcal{E}$ the category $\WellTypedMdaEcospans$ quotiented by the following relation. 
% \[
% 	f \sim g \quad \text{if} \quad f \Rrightarrow^{*}_{\llbracket \mathcal E \rrbracket} g ~ . 
% \]
% \end{definition}

% We will then omit double square brackets around the set of equations and just write $\Rrightarrow_{\mathcal{E}}$ to mean the above.
% Note that not all pushouts (or pushout complements) in $\textbf{MEHypI}(\Sigma)$ are guaranteed to exist: however,  pushouts (and pushout complements) for rewrites generated from $\Sigma$-equations and those generated from the semilattice equations on $\Sigma^+$-terms do exist~\ref{proof:appendix:soundness}.
% Let $\mathcal{S}$ be the set equations of Figure \ref{fig:string-equations},  \textit{i.e.},  the semilattice equations over $\Sigma^+$-terms. 
%Let $\mathcal E$ be an arbitrary set of $\Sigma$-equations. 
Then we have the following result. 
\begin{proposition}[Soundness]
\label{prop:soundness}
The category $\WellTypedMdaEcospans/{\mathcal{S}}$ is a semilattice-enriched PROP. 
\end{proposition}
Note that the symmetric monoidal equations are absorbed in the hypergraph representation,  but the semilattice equations are covered by the fact we have quotiented our representation by certain rewrites. 
Ultimately we want to give interpretation of morphisms in $\catname{S}(\Sigma)$ in $\WellTypedMdaEcospans / \mathcal{S}$ for which we will need a few lemmas.

First notice, that this quotienting gives us certain normal forms.

\begin{lemma}
\label{lemma:normal_form}
For each cospan $f$ in ${\WellTypedMdaEcospans}/{\mathcal{S}}$ there is a \textit{normal form} such that 
\[
	f = f_1 + \ldots + f_n
\]
	such that each $f_i$ contains no hierarchical edges,  and for all $i \neq j$ we have $f_i \neq f_j$.
\end{lemma}

% The reason for calling the previous result \textit{soundness} is that the interpretation function on $\Sigma^+$-terms induces a semilattice-enriched SMC functor $\llbracket - \rrbracket: \catname{S}(\Sigma)^{+} \to \WellTypedMdaEcospans/{\mathcal{S}}$,  which further induces the following semilattice-enriched SMC functor,  for a set of equations $\mathcal E$:
% \[
% 	\llbracket - \rrbracket_{\mathcal E}: \catname{S}(\Sigma, \mathcal E)^{+} \to \WellTypedMdaEcospans/{\mathcal{S,E}}
% \]
% which by uniqueness must be the free semilattice-enriched functor from $\catname{S}(\Sigma, \mathcal E)^{+}$ induced by the interpretation of $\Sigma$ as itself.  
% Unfolding definitions,  it follows that
% \[
% 	f = g \quad \Rightarrow \quad \llbracket f \rrbracket \Rrightarrow^{*}_{\mathcal{S}, \mathcal{E}} \llbracket g \rrbracket~ . 
% \]
% Note that it is immediate from the previous result that $\WellTypedMdaEcospans/{\mathcal{S,E}}$ is indeed a semilattice-enriched PROP.  

\begin{lemma}
\label{lemma:2functor_equiv}
We have the following equivalence obtained by applying the free enrichment 2-functor $\mathcal{F}$ to $\catname{S}(\Sigma) \cong \MdaCospans$
\[
\catname{S}(\Sigma)^{+} \cong \MdaCospans^{+}~.
\]
\end{lemma}

% \begin{lemma}

% There is an inclusion
% \[
% \MdaCospans \xhookrightarrow{In} \MdaEcospans
% \]

% That is identity on objects and which maps a cospan $n \xrightarrow{} F \xleftarrow{} m$ to $n \xrightarrow{} n \xrightarrow{} F \xleftarrow{} m \xleftarrow{} m$ by recalling that every hypergraph is also an e-hypergraph.
% \end{lemma}

\begin{lemma}
\label{lemma:cospans_plus_equiv}
We have the following equivalence of semilattice-enriched categories
\[
\MdaCospans^{+} \cong \MdaEcospans/\mathcal{S}~.
\]
\end{lemma}
\begin{proof}
We will denote the semilattice-enriched functor giving this equivalence as $M$.
Observe that in $\MdaCospans^{+}$ we can consider only canonical morphisms of the form $f_{1} + \ldots + f_{n}$ where each of $f_{i}$ is an mda-cospan and for $i \not = j$, $f_{i} \not = f_{j}$.
Then, $M$ is defined as follows.
\begin{itemize}
\item It maps each discrete hypergraph in $\MdaCospans^{+}$ to a corresponding discrete e-hypergraph in $\MdaEcospans/\mathcal{S}$.
\item Every morphism $f_{1} + \ldots + f_{n}$ is mapped to $\text{In}(f_{1}) + \ldots + \text{In}(f_{n})$ where $\text{In}$ is the faithful functor induced by Remark~\ref{remark:embedding_functor}.
\end{itemize}
To show the functor $M$ is faithful, suppose $f = f_{1} + \ldots + f_{n} \not = g_{1} + \ldots + g_{m} = g$.
Then it must be that $Mf \not = Mg$.
If $n \not = m$ then it is immediately $Mf \not = Mg$ as $Mf$ and $Mg$ contain different number of non-isomorphic components.
If $n = m$ then assume for a contradiction that $Mf = Mg$. 
We have that there exists a permutation $\sigma$ such that for every index $i$ we have $Mf_{i} = Mg_{\sigma(i)}$ and thus by $\text{In}$ that $f_{i} = g_{\sigma(i)}$, implying $f = g$ which is a contradiction.
Therefore, we have $Mf \not = Mg$ as required.

To show that the functor is full consider a morphism $f = f_{1} + \ldots + f_{n}$ in $\WellTypedMdaEcospans / \mathcal{S}$ (by Lemma~\ref{lemma:normal_form}), as each of $f_{i}$ does not contain any hierarchical edges they are in the image of $\text{In}$ which for each $f_{i}$ gives us $g_{i}$ such that $f_{i} = Mg_{i}$.
This gives us a morphism $g = g_{1} + \ldots + {g_{n}}$ in $\MdaCospans^{+}$ such that $Mg = f$.
This concludes that the functor is full.
As it is also surjective on objects, it makes it into an equivalence.
\end{proof}


\begin{theorem}
	\label{thm:completeness_simple}
	We have the following equivalence of categories
	\[
		\catname{S}(\Sigma)^{+} \cong \WellTypedMdaEcospans / \mathcal{S}~.
	\]	
	\end{theorem}
\begin{proof}
	The equivalence is given by a semilattice-enriched functor $\llbracket \rrbracket$ that is obtained by composing the functors from Lemma~\ref{lemma:2functor_equiv} and Lemma~\ref{lemma:cospans_plus_equiv}.
\end{proof}
The benefit of working with the equivalence defined above as compared to the one in Lemma~\ref{lemma:2functor_equiv} is that the former allows us to define EDPOI rewriting more naturally as every morphism is an (extended) cospan.
Finally, we can state the following result.

% The equations of Figure~\ref{fig:string-equations} -- excepting the commutativity equation -- can all be directed from left to right way to form a terminating rewrite system,  which results in a cospan in a \textit{weak normal form}. 
% \begin{proposition}[Weak normal form]
% \label{prop:wnormal_form}
% For each cospan $f$ in ${\WellTypedMdaEcospans}/{\mathcal{S}}$ there is a \textit{weak normal form} such that 
% \[
% 	f = f_1 + \ldots + f_n
% \]
%  such that each $f_i$ contains no hierarchical edges,  and for all $i \neq j$ we have $f_i \neq f_j$.
% \end{proposition}

% The proof follows the familiar argument (\textit{i.e.},  termination of distributivity rewrites) that $\Sigma^+$-terms can be rewritten via the semilattice-enrichment equations into a similar normal form.  

% This normal form can be used to prove the main theorems of the paper: the (full) completeness of our combinatorial representation of $\textbf{SMT}^{+}(\Sigma)$ and $\textbf{SMT}^{+}(\Sigma, \mathcal {E} )$. 
% This normal form will come in handy when proving the following theorem.

% \begin{theorem}
% \label{thm:completeness_simple}
% We have the following equivalence of categories
% \[
% 	\catname{S}(\Sigma)^{+} \cong \WellTypedMdaEcospans / \mathcal{S}~.
% \]	
% \end{theorem}
% \begin{proof}
% The equivalence is given by the semilattice-enriched SMC functor $\llbracket - \rrbracket$.
% First, note that the functor is surjective on objects,  so it suffices to prove the functor is both full and faithful.  
% In each case,  the result follows by a normal form argument in combination with relying on the equivalence $(*)$ $\catname{S}(\Sigma) \cong \textbf{MHypI}(\Sigma)$ 
% % of symmetric monoidal theories and plain MDA cospans which is given by the obvious restriction of $\llbracket - \rrbracket$ to $\textbf{SMT}(\Sigma)$
% ; see Theorem 35 of ~\cite{bonchi_string_2022-2}.

% To see that the functor is faithful,  we prove that $f \neq g$ in $\textbf{SMT}^+(\Sigma)$ implies $\llbracket f \rrbracket \neq \llbracket g \rrbracket$.  
% First,  calculate weak normal form $\Sigma^+$-terms of $f$ and $g$ as $f_1 + \ldots + f_n$ and $g_1 + \ldots + g_m$,  respectively.  Applying the functor $\llbracket - \rrbracket$,  we thus have that
% \[
% 	\llbracket f \rrbracket = \llbracket f_1 \rrbracket + \ldots + \llbracket f_n \rrbracket \qquad
% 	\llbracket g \rrbracket = \llbracket g_1 \rrbracket + \ldots + \llbracket g_m \rrbracket ~ . 
% \]
% Observe that $\llbracket f_1 \rrbracket + \ldots + \llbracket f_n \rrbracket$ and $\llbracket g_1 \rrbracket + \ldots + \llbracket g_m \rrbracket$ are weak normal forms.  
% We consider two cases: where $n \neq m$ and where $n = m$.  
% In the first case,  it is immediate that $\llbracket f \rrbracket \neq \llbracket g \rrbracket $,  since equal terms have weak normal forms with the same number of components.  
% Otherwise,  if $n = m$,  assume for contradiction that $\llbracket f \rrbracket = \llbracket g \rrbracket$.  
% We have that there exists a permutation $\sigma$ such that for every index $i$ we have $\llbracket f_i \rrbracket = \llbracket g_{\sigma(i)} \rrbracket $,  and thus by  $(*)$ that $f_i = g_{\sigma(i)}$,  implying that $f = g$ which is a contradiction.  
% Therefore $\llbracket f \rrbracket \neq \llbracket g \rrbracket$ as required. 

% To see that the functor is full,  consider a morphism $f$ in $\WellTypedMdaEcospans / \mathcal{S}$.  
% We can find an equivalent weak normal form $f_1 + \ldots + f_n$,  where each $f_i$ contains no hierarchical edges.  
% By $(*)$ we can find morphisms $g_i$ in $\textbf{SMT}(\Sigma)$ such that $\llbracket g_i \rrbracket = f_i$.  
% Thus,  we can construct a morphism $g_1 + \ldots + g_n$ such that $\llbracket g_1 + \ldots + g_n \rrbracket = \llbracket g_1 \rrbracket + \ldots \llbracket g_n \rrbracket = f_1 + \ldots + f_n$,  as required. 
% \end{proof}

\begin{theorem}[Full completeness]
\label{thm:full-completeness}
Let $\mathcal{E}$ be the set of equations $l = r$ of $\Sigma$-terms, then we have the following equivalence of categories
\[
	\catname{S}^{+}(\Sigma, \mathcal {E} ) \cong \WellTypedMdaEcospans / \mathcal{S,E}~.
\]
where $\mathcal{E}$ on the right is overloaded as a set of EDPOI rewrite rules containing $\langle \llbracket l \rrbracket, \llbracket r \rrbracket  \rangle$ and $\langle \llbracket r \rrbracket, \llbracket l \rrbracket  \rangle$ for every equation $l = r$.
\end{theorem}

The literal meaning of the above statement is that $f = g$ in $\catname{S}^{+}(\Sigma, \mathcal{E})$ if and only if $\llbracket f \rrbracket \Rrightarrow{}^{*}_{\mathcal{E}} \llbracket g \rrbracket$.

Now we can define DPO (EDPOI) rewriting for e-graphs by following the informal translation in Section~\ref{sec:introduction} that renders a given \textit{canonical} \textit{acyclic} \textit{connected} e-graph as a morphism of $\catname{S}(\Sigma \cup \Sigma_{C}, \mathcal{E} \cup \mathcal{E}_{C})$ and applying the functor of Theorem~\ref{thm:completeness_simple}.
\begin{proposition}
The data above define a function $\llbracket - \rrbracket : \catname{E}\text{-}\catname{graph} \to \WellTypedMdaEcospans / \mathcal{S}$ where the latter has a naturally defined EDPOI rewriting.
Moreover, we have that the following diagram commutes
\[
\begin{tikzcd}
		e_1 \arrow[r, "\leadsto"] \arrow[d, shift right=2]                & e_2 \arrow[d]                                        \\
		\llbracket e_1 \rrbracket \arrow[r, "\Rrightarrow^{*}"] \arrow[u] & \llbracket e_{2} \rrbracket \arrow[u, shift right=2]
\end{tikzcd}
\]
\end{proposition}

Note that the domain of this function is just the set of e-graphs over the same signature and equations which is not a category.
The arrow $\leadsto$ expresses the fact that e-graph $e_2$ is obtainable from e-graph $e_1$ via the application of equations from $\mathcal{E}$: it includes adding nodes and merging e-classes.
% \begin{proof}
% The equivalence is given by the functor $\llbracket - \rrbracket_{\mathcal{E}}$.
% The functor is surjective on objects,  so it suffices to prove the functor is both full and faithful.
% The fullness follows by the same argument as above, so we will proceed to show faithfulness.
% The latter amounts to showing that if $\llbracket f \rrbracket \Rrightarrow_{\llbracket \mathcal{E} \rrbracket}^{*} \llbracket g \rrbracket$ in $\WellTypedMdaEcospans / \mathcal{S}$ then $f = g$ in $\catname{SMT}^{+}(\Sigma,\mathcal{E})$.
% % By the theorem~\ref{thm:completeness_simple} if $\llbracket f \rrbracket = \llbracket g \rrbracket$ via rules from $\mathcal{S}$ then $f = g$ so we will show that the same holds when $\llbracket f \rrbracket = \llbracket g \rrbracket$ via rules from $\mathcal{E}$.
% Consider a rewrite rule $\langle \llbracket l \rrbracket, \llbracket r \rrbracket \rangle = \langle \mathcal{L}, \mathcal{R} \rangle$ such that $\llbracket f \rrbracket \Rrightarrow_{\langle \mathcal{L}, \mathcal{R} \rangle} \llbracket g \rrbracket$ and suppose
% \[
% 	\llbracket f \rrbracket = n \xrightarrow{} n' \xrightarrow{} \mathcal{F} \xleftarrow{} m' \xleftarrow{} m ~.
% \]
% This implies the existence of a diagram as in~\ref{def:dpoi-e}.
% We will proceed by induction on the level of the occurrence of $\mathcal{L}$ within $\mathcal{F}$.
% \begin{itemize}
% 	\item Suppose the image of $\mathcal{L}$ in $\mathcal{F}$ is top-level.
% 		  Then, by lemma~\ref{lemma:decomposition}, there exists a decomposition of $\llbracket f \rrbracket$ in terms of MDA cospans
% 		\begin{align*}
% 			(k \xrightarrow{id_{k}} k \xrightarrow{id_{k}} &k \xleftarrow{id_{k}} k \xleftarrow{id_{k}} k)\\
% 			n \xrightarrow{} n_{1}' \xrightarrow{} \mathcal{C}_{1} \xleftarrow{} m_{1}' + i + k \xleftarrow{} i + k; \qquad\qquad\qquad &\otimes \qquad\qquad\qquad ; j + k \xrightarrow{} j + k + n'_{2} \xrightarrow{} \mathcal{C}_{2} \xleftarrow{} m'_{2} \xleftarrow{} m\\
% 			(i \xrightarrow{} i' \xrightarrow{} &\mathcal{L} \xleftarrow{} j' \xleftarrow{} j)
% 		\end{align*}
% 		By fullness of $\llbracket - \rrbracket$ there exist $c_{1}$ and $c_{2}$ such that 
% 		\[
% 		\llbracket c_{1} \rrbracket = n \xrightarrow{} n_{1}' \xrightarrow{} \mathcal{C}_{1} \xleftarrow{} m_{1}' + i + k \xleftarrow{} i + k \qquad \llbracket c_{2} \rrbracket = j + k \xrightarrow{} j + k + n'_{2} \xrightarrow{} \mathcal{C}_{2} \xleftarrow{} m'_{2} \xleftarrow{} m
% 		\]
% 		By functoriality, $\llbracket f \rrbracket = \llbracket c_{1};(id \otimes l);c_{2} \rrbracket$ and by faithfulness we have that $f = c_{1};(id \otimes l);c_{2}$ to which we can apply $\langle l, r \rangle$ and get $g' = c_{1};(id \otimes r);c_{2}$.
% 		We then have $\llbracket g' \rrbracket = \llbracket c_{1};(id \otimes r);c_{2} \rrbracket =  \llbracket g \rrbracket$ since the boundary complement is unique (Proposition ~\ref{prop:boundary_unique}) and $g' = g$ by faithfulness.

% 		\item If the image of $\mathcal{L}$ is not top-level, then either $\llbracket f \rrbracket$ is decomposable as
% 		\begin{align*}
% 			(k \xrightarrow{id_{k}} k \xrightarrow{id_{k}} &k \xleftarrow{id_{k}} k \xleftarrow{id_{k}} k)\\
% 			n \xrightarrow{} n_{1}' \xrightarrow{} \mathcal{C}_{1} \xleftarrow{} m_{1}' + i + k \xleftarrow{} i + k; \qquad\qquad\qquad &\otimes \qquad\qquad\qquad ; j + k \xrightarrow{} j + k + n'_{2} \xrightarrow{} \mathcal{C}_{2} \xleftarrow{} m'_{2} \xleftarrow{} m\\
% 			(i \xrightarrow{} i' \xrightarrow{} &\mathcal{F'} \xleftarrow{} j' \xleftarrow{} j)
% 		\end{align*}
% 		such that $\mathcal{L}$ is a convex down-closed sub-e-hypergraph of $F'$.
% 		Since the image of $\mathcal{L}$ lies within $\mathcal{F}'$, there is an EDPO rewrite step that turns
% 		\[
% 			(i \xrightarrow{} i' \xrightarrow{} \mathcal{F'} \xleftarrow{} j' \xleftarrow{} j)	
% 		\]
% 		into
% 		\[
% 			(i \xrightarrow{} i'' \xrightarrow{} \mathcal{G'} \xleftarrow{} j'' \xleftarrow{} j)	
% 		\]
% 		via $\langle \mathcal{L}, \mathcal{R} \rangle$.
% 		By fullness and inductive hypothesis, there exist $f'$ and $g'$ such that 
% 		\[
% 			\llbracket f' \rrbracket = i \xrightarrow{} i' \xrightarrow{} \mathcal{F'} \xleftarrow{} j' \xleftarrow{} j \qquad \llbracket g' \rrbracket = (i \xrightarrow{} i'' \xrightarrow{} \mathcal{G'} \xleftarrow{} j'' \xleftarrow{} j)
% 		\]
% 		and that $f' = g'$.
% 		By decomposition and fullness we have $\llbracket f \rrbracket = \llbracket c_{1};(id \otimes f');c_{2} \rrbracket$ and $f = c_{1};(id \otimes f');c_{2}$ by faithfulness.
% 		By applying $f' = g'$ we get $f = g'' = c_{1};(id \otimes g');c_{2}$.
% 		Finally, $\llbracket g'' \rrbracket = \llbracket c_{1};(id \otimes g');c_{2} \rrbracket = \llbracket g \rrbracket$ (by the uniqueness of the boundary complement) and $g'' = g = f$ by faithfulness.

% 		\item or, $\llbracket f \rrbracket$ is decomposable as
% 		\begin{align*}
% 			n_1 \xrightarrow{} n_1' \xrightarrow{} &\mathcal{F}_1 \xleftarrow{} m_1' \xleftarrow{} m_1\\
% 			&\;+\\
% 			&\vdotswithin{+}\\
% 			&\;+\\
% 			n_p \xrightarrow{} n_p' \xrightarrow{} &\mathcal{F}_p \xleftarrow{} m_p' \xleftarrow{} m_p\\
% 			&\;+\\
% 			&\vdotswithin{+}\\
% 			&\;+\\
% 			n_k \xrightarrow{} n_k' \xrightarrow{} &\mathcal{F}_k \xleftarrow{} m_k' \xleftarrow{} m_k
% 		\end{align*}
% 		such that the image of $\mathcal{L}$ lies within $\mathcal{F}_{p}$.
% 		By inductive hypothesis we have that $\llbracket f_{p} \rrbracket = n_p \xrightarrow{} n_{p}' \xrightarrow{} \mathcal{F}_{p} \xleftarrow{} m_{p}' \xleftarrow{} m_{p}$ rewrites into $\llbracket g_{p} \rrbracket = n_p \xrightarrow{} n_{p}'' \xrightarrow{} \mathcal{G}_{p} \xleftarrow{} m_{p}'' \xleftarrow{} m_{p}$ such that $f_{p} = g_{p}$.
% 		By fullness and the fact that $\llbracket - \rrbracket$ is enriched, there exist $\llbracket f_1 + \ldots + f_{p} + \ldots + f_{k} \rrbracket = \llbracket f_{1} \rrbracket + \ldots + \llbracket f_{p} \rrbracket + \ldots + \llbracket f_{k} \rrbracket = \llbracket f \rrbracket$.
% 		By applying $f_{p} = g_{p}$ we get $f = g''$ and $\llbracket g'' \rrbracket = \llbracket f_{1} + \ldots + g_{p} + \ldots f_{k} \rrbracket = \llbracket g \rrbracket$ and $f = g'' = g$.
% \end{itemize}
% \end{proof}

% Unfolding the definitions,  we can spell out the previous theorem in terms of EDPOI-rewriting. 
% \begin{corollary}[Completeness of rewriting]
% For any $f, g \in \textbf{SMT}^{+}(\Sigma, \mathcal{E})$,  we have
% \[
% 	f = g \quad \iff \quad \llbracket f \rrbracket \Rrightarrow^{*}_{\mathcal{S}, \mathcal{E}} \llbracket g \rrbracket~ . 
% \]
%  \end{corollary}
%
%\begin{theorem}[Theorem 35~\cite{bonchi_string_2022-2}]
%
%    $a \Rightarrow_{R} b$ iff $\llangle {}^{\ulcorner} a {}^{\urcorner} \rrangle \Rrightarrow_{\llangle {}^\ulcorner R {}^\urcorner \rrangle} \llangle {}^{\ulcorner}b {}^{\urcorner} \rrangle$.
%    Where $\llangle \cdot \rrangle : \textsf{PROP}(\Sigma) + \textsf{Frob} \to \catname{Csp_{D}(Hyp_{\Sigma})}$ and $\Rrightarrow$ is a convex rewriting relation for morphisms in $\catname{Hyp_{\Sigma}}$.
%\end{theorem}
%
%\begin{proposition}
%    $\llangle {}^{\ulcorner} a {}^{\urcorner} \rrangle \Rrightarrow_{\llangle {}^\ulcorner R {}^\urcorner \rrangle} \llangle {}^{\ulcorner}b {}^{\urcorner} \rrangle$ iff $[\mathcal{J}(a)] \Rrightarrow_{[R]}^{+} [\mathcal{J}(b)]$.
%     Note that $a$ and $b$ are from $\textsf{PROP}(\Sigma)$.
%\end{proposition}
%
%\begin{theorem}
%    $a \Rightarrow_{\langle l, r \rangle}^{+} b$ in $\textsf{PROP}^{+}$ iff $[a] \Rrightarrow_{\langle [l],[r] \rangle} [b]$ in $\sfrac{\WellTypedMdaEcospans}{\mathcal{S}}$.
%\end{theorem}
% Note that in string diagram rewrite theory as developed in \cite{bonchi_string_2022-2},  some extra steps are taken to prove the claim that $f \rightsquigarrow g$ if and only if $\llbracket f \rrbracket \Rrightarrow^{*}_{\mathcal{S}, \mathcal{E}} \llbracket g \rrbracket$,  for an appropriate definition of $\Sigma$-term rewriting $\rightsquigarrow$.  We omit this extra detail for brevity and clarity,  dealing with equational theories rather than rewrite theories,  but it is straightforward to extend our results to prove an analogous claim.  
