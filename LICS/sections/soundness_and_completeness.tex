\section{Soundness and Completeness}\label{sec:soundness-and-completeness}

This section contains the main technical results of the paper,  building on those of \cite{bonchi_string_2022-2}. 
We first construct an interpretation of terms generating $\textbf{PROP}^+(\Sigma)$ in $\WellTypedMdaEcospans$.  
As expected,  the SMC equations are factored out in our representation of terms in $\WellTypedMdaEcospans$,  but the semilattice equations are not.  
Instead,  we implement the semilattice equations via EDPOI rewrites,  leading to the following soundness and completeness result: for any morphisms $f$ and $g$ in $\textbf{SMT}(\Sigma, \mathcal E)$,  $f = g$  if and only if there exists a sequence of EDPOI-rewrites --- each induced by either a structural semilattice equation or $\mathcal E$ --- between their combinatorial representations. 

\begin{figure}
    \[
    \scalebox{0.55}{
    \tikzfig{../figures/combinatorial_semantics/f_plus_g_new}
    }
    \]
    \captionsetup{belowskip=-4ex}
    \caption{$+$ of two morphisms in $\Ecospans$}
    \label{fig:A+B}
\end{figure}
Recall from Section \ref{sec:e-hypergraphs} that $ \WellTypedMdaEcospans$ has a symmetric monoidal structure inherited from the coproduct structure of $\textbf{EHypI}(\Sigma)$.  
Recall further that typed $\Sigma^+$-terms are those categorical combinator terms freely constructed from generators $c \in \Sigma$, $\textsf{id}_1$, $\sigma_{1,1}$, $(;\!)$ and $\otimes$,  and $f+g$.  We will write $\Sigma^+(n,m)$ for the set of $\Sigma^+$-terms of type $n \to m$.  

We wish to specify a set of EDPOI rewrite rules corresponding to those in Figure \ref{fig:string-equations}.  In order to do this,  we specify a function which maps  uninterpreted $\Sigma^+$-terms into $\WellTypedMdaEcospans$ morphisms,  and which also extends the expected interpretation of $\Sigma$-terms in $\WellTypedMdaEcospans$ (as given by the free functor from $\textbf{PROP}(\Sigma)$ induced by the interpretation of $\Sigma$ as itself,  detailed in \cite{bonchi_string_2022-2}). 
\begin{definition}[Interpretation function] 
%Let $\llbracket - \rrbracket: \textbf{PROP}(\Sigma) \to \WellTypedMdaEcospans$ be the unique SMC-functor induced by the interpretation sending $\Sigma$ to itself.  
Let $\llbracket - \rrbracket_{n,m}: \Sigma^+(n,m) \to \WellTypedMdaEcospans(n,m)$ be the function defined by induction on the type derivation of $\Sigma$-terms by 
\begin{align*}
	\llbracket \id \rrbracket_{n,m} &= \id  &	\llbracket f ; g \rrbracket_{n,m} &= \llbracket f \rrbracket_{n,m} ; \llbracket g \rrbracket_{n,m} \\
	\llbracket I \rrbracket_{n,m} &= I & \llbracket f \otimes g \rrbracket_{n,m} &= \llbracket f \rrbracket^+_{n,m} \otimes \llbracket g \rrbracket_{n,m} \\
	\llbracket \sym \rrbracket_{n,m} &= \sym  & \llbracket c \rrbracket_{n,m} &= c   \\ 
	\llbracket f+g \rrbracket_{n,m} &= \llbracket f \rrbracket_{n,m} + \llbracket g \rrbracket_{n,m}
\end{align*}
where $c \in \Sigma$ and $+$ in $\WellTypedMdaEcospans$ is defined by Figure \ref{fig:A+B}.  It uses the notation of dashed boxes for hierarchical edges, with its children drawn inside,  and maximally consistent components separated by another dashed line.  We will omit the subscripts in $\llbracket - \rrbracket_{n,m}$ where it is clear from context.
The definitions of the right-hand sides of the base cases above can be found in~\cite{bonchi_string_2022-1} that defines the corresponding cospans of hypergraphs. One can then get extended cospans by following~\ref{remark:embedding_functor}.
\end{definition}

In order to phrase our soundness and completeness results categorically,  we define the following quotient of $\WellTypedMdaEcospans$ by EDPOI rewrites.  This means that soundness is expressed as the extension of $\llbracket - \rrbracket$ to a semilattice-enriched SMC functor from $\textbf{SMT}^+(\Sigma, \mathcal E)$,  and completeness is expressed as the faithfulness of this functor.  In fact,  we prove that the functor is also full,  giving an equivalence of categories. 

We define the appropriate notion of quotient by EDPOI rewrites below.  
% Let the function $\llbracket - \rrbracket$ extend in the obvious way to apply to equations (\textit{i.e.}, pairs of $\Sigma^+$-terms).
Given a set of equations $\mathcal{E}$ of $\Sigma^{+}$-terms, $\llbracket \mathcal{E} \rrbracket$ is defined as $\{\langle \llbracket l \rrbracket, \llbracket r \rrbracket \rangle \text{ for each } l = r \in \mathcal{E} \} \cup \{\langle \llbracket r \rrbracket, \llbracket l \rrbracket \rangle \text{ for each } l = r \in \mathcal{E}\}$.
% \update{below definition depends on whether we require the homomorphism to reflect the conflicts}
\begin{definition}[Quotient by rewrites]  
Given a set of equations $\mathcal{E}$,  we denote by $\WellTypedMdaEcospans/\mathcal{E}$ the category $\WellTypedMdaEcospans$ quotiented by the following relation. 
\[
	f \sim g \quad \text{if} \quad f \Rrightarrow^{*}_{\llbracket \mathcal E \rrbracket} g ~ . 
\]
\end{definition}

Note that not all pushouts (or pushout complements) in $\textbf{MEHypI}(\Sigma)$ are guaranteed to exist: however,  pushouts (and pushout complements) for rewrites generated from $\Sigma$-equations and those generated from the semilattice equations on $\Sigma^+$-terms do exist~\ref{proof:appendix:soundness}.
Let $\mathcal{S}$ be the set equations of Figure \ref{fig:string-equations},  \textit{i.e.},  the semilattice equations over $\Sigma^+$-terms. 
%Let $\mathcal E$ be an arbitrary set of $\Sigma$-equations. 
Then we have the following result. 
\begin{proposition}[Soundness]
\label{prop:soundness}
The category $\WellTypedMdaEcospans/{\mathcal{S}}$ is a semilattice-enriched PROP. 
\end{proposition}
Note that the symmetric monoidal equations are absorbed in the hypergraph representation,  but the semilattice equations are covered by the fact we have quotiented our representation by certain rewrites. 

The reason for calling the previous result \textit{soundness} is that the interpretation function on $\Sigma^+$-terms induces a semilattice-enriched SMC functor $\llbracket - \rrbracket: \textbf{PROP}^+(\Sigma) \to \WellTypedMdaEcospans/{\mathcal{S}}$,  which further induces the following semilattice-enriched SMC functor,  for a set of equations $\mathcal E$:
\[
	\llbracket - \rrbracket_{\mathcal E}: \textbf{SMT}^+(\Sigma, \mathcal E) \to \WellTypedMdaEcospans/{\mathcal{S,E}}
\]
which by uniqueness must be the free semilattice-enriched functor from $\textbf{SMT}^+(\Sigma, \mathcal E)$ induced by the interpretation of $\Sigma$ as itself.  Unfolding definitions,  it follows that
\[
	f = g \quad \Rightarrow \quad \llbracket f \rrbracket \Rrightarrow^{*}_{\mathcal{S}, \mathcal{E}} \llbracket g \rrbracket~ . 
\]
Note that it is immediate from the previous result that $\WellTypedMdaEcospans/{\mathcal{S,E}}$ is indeed a semilattice-enriched PROP.  

The equations of Figure~\ref{fig:string-equations} -- excepting the commutativity equation -- can all be directed from left to right way to form a terminating rewrite system,  which results in a cospan in \textit{weak normal form}. 
\begin{proposition}[Weak normal form]
\label{prop:wnormal_form}
For each cospan $f$ in ${\WellTypedMdaEcospans}/{\mathcal{S}}$ there is a \textit{weak normal form} such that 
\[
	f = f_1 + \ldots + f_n
\]
 such that each $f_i$ contains no hierarchical edges,  and for all $i \neq j$ we have $f_i \neq f_j$.
\end{proposition}    
The proof follows the familiar argument (\textit{i.e.},  termination of distributivity rewrites) that $\Sigma^+$-terms can be rewritten via the semilattice-enrichment equations into a similar normal form.  

This normal form can be used to prove the main theorem of the paper: the (full) completeness of our combinatorial representation of $\textbf{SMT}^{+}(\Sigma, \mathcal {E} )$. 
\begin{theorem}[Full completeness]
We have the following equivalence of categories
\[
	\textbf{SMT}^{+}(\Sigma, \mathcal {E} ) \cong \WellTypedMdaEcospans / \mathcal{S,E}~.
\]
\end{theorem}
\begin{proof}
The equivalence is given by the functor $\llbracket - \rrbracket$.  First, note that the functor is surjective on objects,  so it suffices to prove the functor is both full and faithful.  In each case,  the result follows by a normal form argument in combination with relying on the equivalence $(*)$ $\textbf{SMT}(\Sigma, \mathcal E) \cong \textbf{MHypI}(\Sigma)/\mathcal{E}$ of symmetric monoidal theories and (appropriately quotiented) plain mda-cospans which is given by the obvious restriction of $\llbracket - \rrbracket$ to $\textbf{SMT}(\Sigma, \mathcal E)$; see Theorem 35 of ~\cite{bonchi_string_2022-2}.  

To see that the functor is faithful,  we prove that $f \neq g$ in $\textbf{SMT}^+(\Sigma, \mathcal E)$ implies $\llbracket f \rrbracket \neq \llbracket g \rrbracket$.  First,  calculate weak normal form $\Sigma^+$-terms of $f$ and $g$ as $f_1 + \ldots + f_n$ and $g_1 + \ldots + g_m$,  respectively.  Applying the functor $\llbracket - \rrbracket$,  we thus have that
\[
	\llbracket f \rrbracket = \llbracket f_1 \rrbracket + \ldots + \llbracket f_n \rrbracket \qquad
	\llbracket g \rrbracket = \llbracket g_1 \rrbracket + \ldots + \llbracket g_m \rrbracket ~ . 
\]
Observe that $\llbracket f_1 \rrbracket + \ldots + \llbracket f_n \rrbracket$ and $\llbracket g_1 \rrbracket + \ldots + \llbracket g_m \rrbracket$ are weak normal forms.  
% We consider two cases: where $n \neq m$ and where $n = m$.  
% In the first case,  it is immediate that $\llbracket f \rrbracket \neq \llbracket g \rrbracket $,  since equal terms have weak normal forms with the same number of components.  
% Otherwise,  if $n = m$,  assume for contradiction that $\llbracket f \rrbracket = \llbracket g \rrbracket$.  
% We have that there exists a permutation $\sigma$ such that for every index $i$ we have $\llbracket f_i \rrbracket = \llbracket g_{\sigma(i)} \rrbracket $,  and thus by  $(*)$ that $f_i = g_{\sigma(i)}$,  implying that $f = g$ which is a contradiction.  Therefore $\llbracket f \rrbracket \neq \llbracket g \rrbracket$ as required. 
Suppose $n \leq m$ and assume for contradiction that $\llbracket f \rrbracket = \llbracket g \rrbracket$. 
This implies that for each $1 \leq i \leq n$ there exists a family of $\{j_k\}_{i}$ such that $\llbracket f_{i} \rrbracket = \sum\limits_{\{j_{k}\}_{i}}\llbracket g_{j_{k}} \rrbracket$ and such that $\sum\limits_{i=1}^{n}(\sum\limits_{j_{k} \in \{j_{k}\}_{i}} \llbracket g_{j_{k}} \rrbracket) = \llbracket g \rrbracket$ and by (*) that $f_{i} = \sum\limits_{\{j_{k}\}_{i}}g_{j_{k}}$ and $\sum\limits_{i=1}^{n}f_{i} = g$.
Therefore, $f \neq g$ implies $\llbracket f \rrbracket \neq \llbracket g \rrbracket$.

To see that the functor is full,  consider a morphism $f$ in $\WellTypedMdaEcospans$.  We can find an equivalent weak normal form $f_1 + \ldots + f_n$,  where each $f_i$ contains no hierarchical edges.  By $(*)$ we can find morphisms $g_i$ in $\textbf{SMT}(\Sigma, \mathcal E)$ such that $\llbracket g_i \rrbracket = f_i$.  Thus,  we can construct a morphism $g_1 + \ldots + g_n$ such that $\llbracket g_1 + \ldots + g_n \rrbracket = \llbracket g_1 \rrbracket + \ldots \llbracket g_n \rrbracket = f_1 + \ldots + f_n$,  as required. 
\end{proof}

Unfolding the definitions,  we can spell out the previous theorem in terms of EDPOI-rewriting. 
\begin{corollary}[Completeness of rewriting]
For any $f, g \in \textbf{SMT}^{+}(\Sigma, \mathcal{E})$,  we have
\[
	f = g \quad \iff \quad \llbracket f \rrbracket \Rrightarrow^{*}_{\mathcal{S}, \mathcal{E}} \llbracket g \rrbracket~ . 
\]
 \end{corollary}
%
%\begin{theorem}[Theorem 35~\cite{bonchi_string_2022-2}]
%
%    $a \Rightarrow_{R} b$ iff $\llangle {}^{\ulcorner} a {}^{\urcorner} \rrangle \Rrightarrow_{\llangle {}^\ulcorner R {}^\urcorner \rrangle} \llangle {}^{\ulcorner}b {}^{\urcorner} \rrangle$.
%    Where $\llangle \cdot \rrangle : \textsf{PROP}(\Sigma) + \textsf{Frob} \to \catname{Csp_{D}(Hyp_{\Sigma})}$ and $\Rrightarrow$ is a convex rewriting relation for morphisms in $\catname{Hyp_{\Sigma}}$.
%\end{theorem}
%
%\begin{proposition}
%    $\llangle {}^{\ulcorner} a {}^{\urcorner} \rrangle \Rrightarrow_{\llangle {}^\ulcorner R {}^\urcorner \rrangle} \llangle {}^{\ulcorner}b {}^{\urcorner} \rrangle$ iff $[\mathcal{J}(a)] \Rrightarrow_{[R]}^{+} [\mathcal{J}(b)]$.
%     Note that $a$ and $b$ are from $\textsf{PROP}(\Sigma)$.
%\end{proposition}
%
%\begin{theorem}
%    $a \Rightarrow_{\langle l, r \rangle}^{+} b$ in $\textsf{PROP}^{+}$ iff $[a] \Rrightarrow_{\langle [l],[r] \rangle} [b]$ in $\sfrac{\WellTypedMdaEcospans}{\mathcal{S}}$.
%\end{theorem}
Note that in string diagram rewrite theory as developed in \cite{bonchi_string_2022-2},  some extra steps are taken to prove the claim that $f \rightsquigarrow g$ if and only if $\llbracket f \rrbracket \Rrightarrow^{*}_{\mathcal{S}, \mathcal{E}} \llbracket g \rrbracket$,  for an appropriate definition of $\Sigma$-term rewriting $\rightsquigarrow$.  We omit this extra detail for brevity and clarity,  dealing with equational theories rather than rewrite theories,  but it is straightforward to extend our results to prove an analogous claim.  
