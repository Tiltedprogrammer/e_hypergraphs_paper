The technique of equipping graphs with an equivalence relation, called equality saturation, has recently proved both powerful and practical in program optimisation, particularly for satisfiability modulo theory solvers. 
We give a categorical semantics to these structures, called e-graphs, in terms of Cartesian categories enriched over a category of semilattices.
We show how this semantics can be generalised to monoidal categories, which opens the door to new applications of e-graph techniques, from algebraic to monoidal theories.
Finally, we present a sound and complete combinatorial representation of morphisms in such a category,  based on a generalisation of hypergraphs which we call e-hypergraphs.
They have the usual advantage that many of their structural equations are absorbed into a general notion of isomorphism. 
The advantage of this new principled approach to e-graphs is that for the first time it is possible to support double-pushout (DPO) rewriting for these structures which constitutes the main contribution of this paper.