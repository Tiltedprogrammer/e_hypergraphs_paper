%!LW recipe=xelatex
\documentclass[aspectratio=169]{beamer}

\usepackage{tikz}

\usepackage{amsmath}
\usepackage{tikz-network}
\usepackage{tikz-cd}
\usepackage{tikzit}
\usepackage{subcaption}
\usepackage{minted}
\usepackage{mathtools}
\usepackage{stmaryrd}
\newcommand{\consistency}{{\smile}}
\usepackage{adjustbox}
\usepackage{appendixnumberbeamer}
\usepackage{csquotes}
\usepackage{mathpartir}

\usepackage[style=authoryear,backend=bibtex]{biblatex} % or backend=biber
\addbibresource{./bibliography.bib}

% \hypersetup{
%     colorlinks=true,
%     linkcolor=gray,
%     filecolor=magenta,      
%     urlcolor=cyan,
%     pdftitle={Overleaf Example},
%     pdfpagemode=FullScreen,
%     }

\input{../../sample.tikzstyles}
\input{../../hypergraph.tikzstyles}
\input{../../hypergraph.tikzdefs}

\usetheme[subsectionpage=simple]{metropolis}
% \usetheme{awesome}

% \setsansfont{Ubuntu}
% \setmonofont{Ubuntu Mono}


\setbeamertemplate{footline}
{
  \leavevmode
  \hbox{
  \begin{beamercolorbox}[wd=.15\paperwidth,ht=2.25ex,dp=1ex,center]{title in head/foot}
    \usebeamerfont{author in head/foot}\insertshortauthor
  \end{beamercolorbox}

  \begin{beamercolorbox}[wd=.7\paperwidth,ht=2.25ex,dp=1ex,center]{author in head/foot}
    \usebeamerfont{author in head/foot}\insertshorttitle
  \end{beamercolorbox}

  \begin{beamercolorbox}[wd=.15\paperwidth,ht=2.25ex,dp=1ex,center]{title in head/foot}
    \insertframenumber{} / \inserttotalframenumber
  \end{beamercolorbox}
  }
}

\makeatletter
\metropolis@disablesubsectionpage
\makeatother

\newcommand{\bsubsection}[1]{\subsection{$\bullet$ #1}}

\title{Categorical semantics of E-Graphs} 
\date{October, 2025} % Use metropolis theme 
\author[Aleksei Tiurin]{Aleksei Tiurin}
\begin{document} 


\renewcommand{\maketitle}{
  \begin{frame}[plain]
    \usebeamerfont{title}\usebeamercolor[fg]{title}\inserttitle\par
    \alert{\rule{\textwidth}{1pt}}
    \vfill
    \[
    \left\llbracket 
        \;\vcenter{\hbox{\tikzfig{./figures/e-graph-logo}}}\; 
    \right\rrbracket
    \]
    \vfill
    \centering
    \usebeamerfont{author}\insertauthor\par
    \usebeamerfont{date}\insertdate\par
  \end{frame}
}

\maketitle 

% \begin{frame}{Table of contents}
%     \small
%     \setbeamertemplate{section in toc}[sections numbered]
%     \tableofcontents
% \end{frame}

\section{What and Why?}
\bsubsection{Equality Saturation and E-graphs}

\begin{frame}{Equality saturation}
\vfill
Program optimisation technique that obviates the need to worry about \alert{ordering}
\pause
\vfill
Equality information is \alert{added} until the representation is \enquote{saturated} with respect to these equalities\par
\pause
\vfill
Once saturated, the \alert{best} candidate can be extracted from the set\par
\vfill
\pause
The \alert{best} candidate is domain-specific
\begin{itemize}
    \item Best program in compiler optimisation
    \item Smallest test-case in fuzzy testing
    \item Smallest digital circuit
    \item $\ldots$
\end{itemize}


\end{frame}

\begin{frame}{Equality saturation}

\begin{example}
    \vfill
    Let $P \Coloneqq (a * 2) / 2$, $\mathcal{E} \Coloneqq \{\alert<2>{x * 2 = x <\!\!< 1}, \alert<3>{(x * y) / z = x * (y / z)}, \alert<4>{x / x = 1}, \alert<5>{x * 1 = x} \}$
    \vfill
    \pause
    $P \Coloneqq \{(a * 2) / 2, \alert<2>{(a <\!\!< 1) / 2}\}$
    \vfill
    \pause
    $P \Coloneqq \{(a * 2) / 2, (a <\!\!< 1) / 2, \alert<3>{a * (2 / 2)}\}$
    \vfill
    \pause
    $P \Coloneqq \{(a * 2) / 2, (a <\!\!< 1) / 2, a * (2 / 2), \alert<4>{a * 1}\}$
    \vfill
    \pause
    $P \Coloneqq \{(a * 2) / 2, (a <\!\!< 1) / 2, a * (2 / 2), a * 1, \alert<5>{a}\}$ 
    \pause
    \vfill
    $P \Coloneqq \{\alert<6>{(a * 2) / 2}, (a <\!\!< 1) / 2, a * (2 / 2), \alert<6>{a * 1}, a, \alert<6>{((a * 1) * 2) / 2}\}$
    \vfill
    $\ldots$ 
\end{example}
        

\end{frame}

\begin{frame}{E-graphs}
    \vfill
    Data structure to \alert{efficiently} represent an equivalence relation on a set of terms over an \alert{algebraic} signature $\Sigma$
    \pause
    \vfill
    \begin{itemize}
        \item $\Sigma = \{\mathbb{N}, * : 2 \to 1, / : 2 \to 1, <\!\!< : 2 \to 1, a : 0 \to 1\}$
        \item $a$, $a * 2$, $a <\!\!< 1$, $\ldots$
    \end{itemize}
    \vfill
\end{frame}

% \bsubsection{Monoidal theories}

\begin{frame}{E-graphs}
\begin{minipage}{0.45\linewidth}
E-classes $\scalebox{0.4}{\begin{tikzpicture}\begin{pgfonlayer}{nodelayer}\node [style=empty diag red] (0) at (0, 0) {};\end{pgfonlayer}\end{tikzpicture}}$,$\scalebox{0.4}{\begin{tikzpicture}\begin{pgfonlayer}{nodelayer}\node [style=empty diag yellow] (0) at (0, 0) {};\end{pgfonlayer}\end{tikzpicture}}$,$\scalebox{0.4}{\begin{tikzpicture}\begin{pgfonlayer}{nodelayer}\node [style=empty diag blue] (0) at (0, 0) {};\end{pgfonlayer}\end{tikzpicture}}$,$\scalebox{0.4}{\begin{tikzpicture}\begin{pgfonlayer}{nodelayer}\node [style=empty diag green] (0) at (0, 0) {};\end{pgfonlayer}\end{tikzpicture}}$,$\scalebox{0.4}{\begin{tikzpicture}\begin{pgfonlayer}{nodelayer}\node [style=empty diag black] (0) at (0, 0) {};\end{pgfonlayer}\end{tikzpicture}}$,$\ldots$ 
\begin{itemize}
    \item Set of equivalent terms represented by enodes rooted at this particular e-class
\end{itemize}
\vfill
E-nodes $\begin{tikzpicture}\begin{pgfonlayer}{nodelayer}\node [style=round box] (0) at (0, 0) {$f_1$};\node [style=none] (1) at (-0.5, -0.5) {};\node [style=none] (2) at (0.5, -0.5) {};\node [style=none] (3) at (0, -0.5) {$\small\ldots$};\end{pgfonlayer}\begin{pgfonlayer}{edgelayer}\draw [style=new edge style 1] (0.center) to (1.center);\draw [style=new edge style 1] (0.center) to (2.center);\end{pgfonlayer}\end{tikzpicture}$, $\ldots$, $\begin{tikzpicture}\begin{pgfonlayer}{nodelayer}\node [style=round box] (0) at (0, 0) {$f_n$};\node [style=none] (1) at (-0.5, -0.5) {};\node [style=none] (2) at (0.5, -0.5) {};\node [style=none] (3) at (0, -0.5) {$\small\ldots$};\end{pgfonlayer}\begin{pgfonlayer}{edgelayer}\draw [style=new edge style 1] (0.center) to (1.center);\draw [style=new edge style 1] (0.center) to (2.center);\end{pgfonlayer}\end{tikzpicture}$
\begin{itemize}
    \item A representative of an e-class once the root node is fixed
\end{itemize}
\vfill
Machinery to maintain \alert{sharing} and \alert{congruence} ($t_1 \cong t_2 \implies f\;(t_1) \cong f\;(t_2)$)
\begin{itemize}
    \item Union-find
    \item Hashconsing
\end{itemize}
\end{minipage}
\hfill
\begin{minipage}{0.45\linewidth}
\[
\tikzfig{./figures/e-graph-generic-example}
\]
\end{minipage}
\end{frame}

\begin{frame}{E-graphs}
\begin{example}[$(a * 2) / 2$]
    \[
    \tikzfig{./figures/e-graph-example-a-no-label}
    \]
\end{example}
\end{frame}

\begin{frame}{E-graphs}
    \begin{example}[$x * 2 \to x <\!\!< 1$]
        \hspace{-3em}
        \begin{minipage}{0.25\linewidth}
        \[
        \scalebox{0.8}{
        \tikzfig{./figures/e-graph-example-a-no-label}
        }
        \]    
        \end{minipage}
        \pause
        \hspace{-1.5em}
        \begin{minipage}{0.075\linewidth}
        \[
        \xRightarrow{\texttt{add}(a <\!\!< 1)}
        \]
        \end{minipage}
        \hfill
        \begin{minipage}{0.25\linewidth}
            \[
            \scalebox{0.8}{
            \tikzfig{./figures/e-graph-example-b-add}
            }
            \]    
        \end{minipage}
        \hfill
        \pause
        \begin{minipage}{0.1\linewidth}
            \[
            \xRightarrow{\texttt{merge}(\scalebox{0.3}{\begin{tikzpicture}\begin{pgfonlayer}{nodelayer}\node [style=empty diag yellow] (0) at (0, 0) {<\!\!<};\end{pgfonlayer}\end{tikzpicture}},\scalebox{0.3}{\begin{tikzpicture}\begin{pgfonlayer}{nodelayer}\node [style=empty diag black] (0) at (0, 0) {*};\end{pgfonlayer}\end{tikzpicture}})}
            \]
        \end{minipage}
        \hfill
        \begin{minipage}{0.25\linewidth}
            \[
            \scalebox{0.8}{
            \tikzfig{./figures/e-graph-example-b-merged}
            }
            \]
        \end{minipage}
        \hfill
    \end{example}
\end{frame}


\begin{frame}{E-graphs}
    \begin{example}[$(x * y) / z \to x * (y / z)$]
        \hspace{-2.5em}
        \begin{minipage}{0.25\linewidth}
        \[
        \scalebox{0.8}{
        \tikzfig{./figures/e-graph-example-b-merged}
        }
        \]    
        \end{minipage}
        \pause
        \hspace{-1.5em}
        \begin{minipage}{0.075\linewidth}
        \[
        \xRightarrow{\texttt{add}(a * (2/2))}
        \]
        \end{minipage}
        \hfill
        \begin{minipage}{0.25\linewidth}
            \[
            \scalebox{0.8}{
                \tikzfig{./figures/e-graph-example-c-add}
            }
            \]    
        \end{minipage}
        \hfill
        \pause
        \begin{minipage}{0.1\linewidth}
            \[
            \xRightarrow{\texttt{merge}(\scalebox{0.3}{\begin{tikzpicture}\begin{pgfonlayer}{nodelayer}\node [style=empty diag yellow] (0) at (0, 0) {/};\end{pgfonlayer}\end{tikzpicture}},\scalebox{0.3}{\begin{tikzpicture}\begin{pgfonlayer}{nodelayer}\node [style=empty diag black] (0) at (0, 0) {*};\end{pgfonlayer}\end{tikzpicture}})}
            \]
        \end{minipage}
        \hfill
        \begin{minipage}{0.25\linewidth}
            \[
            \scalebox{0.8}{
            \tikzfig{./figures/e-graph-example-c-merged}
            }
            \]
        \end{minipage}
        \hfill
    \end{example}
\end{frame}


\begin{frame}{E-graphs}
    \begin{example}[$(x / x) \to 1$]
        \hspace{-2.5em}
        \begin{minipage}{0.25\linewidth}
        \[
        \scalebox{0.8}{
        \tikzfig{./figures/e-graph-example-c-merged}
        }
        \]    
        \end{minipage}
        \pause
        \hfill
        \begin{minipage}{0.075\linewidth}
        \[
        \xRightarrow{\texttt{add}(1)}
        \]
        \end{minipage}
        \hfill
        \begin{minipage}{0.25\linewidth}
            \[
            \scalebox{0.8}{
            \tikzfig{./figures/e-graph-example-c-merged}
            }
            \]    
        \end{minipage}
        \hfill
        \pause
        \begin{minipage}{0.1\linewidth}
            \[
            \xRightarrow{\texttt{merge}(\scalebox{0.3}{\begin{tikzpicture}\begin{pgfonlayer}{nodelayer}\node [style=empty diag black] (0) at (0, 0) {/};\end{pgfonlayer}\end{tikzpicture}},\scalebox{0.3}{\begin{tikzpicture}\begin{pgfonlayer}{nodelayer}\node [style=empty diag black] (0) at (0, 0) {1};\end{pgfonlayer}\end{tikzpicture}})}
            \]
        \end{minipage}
        \hfill
        \begin{minipage}{0.25\linewidth}
            \[
            \scalebox{0.8}{
            \tikzfig{./figures/e-graph-example-d-merged}
            }
            \]
        \end{minipage}
        \hfill
    \end{example}
\end{frame}


\begin{frame}{E-graphs}
    \begin{example}[$x * 1 \to x$]
        \hspace{-2em}
        % \hfill
        \begin{minipage}{0.2\linewidth}
        \[
        \scalebox{0.8}{
        \tikzfig{./figures/e-graph-example-d-merged}
        }
        \]    
        \end{minipage}
        \pause
        \hspace{2em}
        \begin{minipage}{0.1\linewidth}
        \[
        \xRightarrow{\texttt{add}(a)}
        \]
        \end{minipage}
        \begin{minipage}{0.2\linewidth}
            \[
            \scalebox{0.8}{
            \tikzfig{./figures/e-graph-example-d-merged}
            }
            \]    
        \end{minipage}
        \hspace{1em}
        \pause
        \begin{minipage}{0.25\linewidth}
            \[
            \xRightarrow{\texttt{merge}(\scalebox{0.3}{\begin{tikzpicture}\begin{pgfonlayer}{nodelayer}\node [style=empty diag black] (0) at (0, 0) {/\;,*};\end{pgfonlayer}\end{tikzpicture}},\scalebox{0.3}{\begin{tikzpicture}\begin{pgfonlayer}{nodelayer}\node [style=empty diag black] (0) at (0, 0) {a};\end{pgfonlayer}\end{tikzpicture}})}
            \]
        \end{minipage}
        \hspace{-2em}
        \begin{minipage}{0.2\linewidth}
            \[
            \scalebox{0.8}{
            \tikzfig{./figures/e-graph-example-e-merged}
            }
            \]
        \end{minipage}
    \end{example}
\end{frame}


\begin{frame}{What we do}
We look at e-graphs from a \alert{categorical} perspective which has the following benefits
\begin{itemize}
    \item Provides a new perspective on e-graph operations
    \item Opens up new avenues for applications
\end{itemize}
\end{frame}

\begin{frame}{Categorical semantics}

We generalise from \alert{algebraic} signatures to arbitrary \alert{monoidal signatures}
\begin{itemize}
    \item $\Sigma_{\mathbf{CMon}} = \{\mu : 1 \to 2, i : 1 \to 0\}$
\end{itemize}
These induce corresponding syntactic categories $\mathbf{S}(\Sigma)$ (\textbf{PROP}s)
\begin{itemize}
    \item Morphisms are given by terms freely constructed using functional symbols from $\Sigma$ and $\otimes,\; ;, \textsf{sym}, id$ quotiented by the axioms of symmetric monoidal categories
\end{itemize}

We can also consider equations between $\Sigma$-terms $\mathcal{E}$, i.e. pairs of morphisms $(l,r)$ with matching arities and co-arities (types), which gives rise to $\mathbf{SMT}(\Sigma,\mathcal{E})$ and the corresponding syntactic category $\mathbf{S}(\Sigma, \mathcal{E})$, where morphisms are additionally quotiented by $\mathcal{E}$
\begin{itemize}
    \item $\mathcal{E} = \{(\mu(x);id \otimes i \equiv x), \ldots\}$
\end{itemize}

\end{frame}

\begin{frame}{Categorical semantics}
Such constructions crop up in many domains and make it possible to perform reasoning syntactically using term rewriting or string diagram rewriting
\begin{itemize}
    \item Quantum circuits
    \item Digital circuits
    \item Lambda calculus
    \item $\ldots$
\end{itemize}
\end{frame}

\begin{frame}{Categorical semantics}
\vfill
To capture the notion of equivalence between (sub)terms we freely enrich a given \textbf{PROP} over a category of semilattices ($\mathbf{SLat}$)
\vfill
$\mathbf{SLat}$:
\begin{itemize}
\item Objects are commutative idempotent semigroups ($(\mathbb{S}, +)$)
\item Morphisms are homomorphisms of $\uparrow$
\item $S_1 \otimes S_2$ is given by formal pairs $(s_1, s_2)$ such that $+$ is bilinear
\begin{itemize}
    \item $(s_1 +_{S_1} s_1', s_2) \equiv (s_1,s_2) +_{S_1 \otimes S_2} (s'_1, s_2)$
    \item $(s_1, s_2 +_{S_2} s_2') \equiv (s_1,s_2) +_{S_1 \otimes S_2} (s_1, s_2')$
\end{itemize}
\item The free enrichment is given by free-forgetful adjunction $F \dashv U : \mathbf{SLat} \to \mathbf{Set}$
\item Endows each hom-set with an idempotent binary operation $+$
\item Preserves naturality, monoidal structure, adjunctions etc. 
\end{itemize}
\vfill
\end{frame}

\begin{frame}{Categorical semantics}
\vfill
The above gives rise to $\mathbf{S}^{+}(\Sigma)$ and $\mathbf{S}^{+}(\Sigma,\mathcal{E})$, respectively
\vfill
\pause
\vfill
Which can also be constructed syntactically from $\Sigma^{+}$-terms (terms with additional constructor $+$)
\[
\scalebox{0.8}{
\tikzfig{./figures/egraph-strings}
}
\]

\end{frame}

\begin{frame}{Categorical semantics}

Appropriately quotiented $\ldots$

\[
\scalebox{0.5}{
\tikzfig{./figures/egraph-strings-equations}
}
\]

\end{frame}

\bsubsection{Combinatorial semantics}

\begin{frame}{Combinatorial semantics}

Syntactic reasoning for $\mathbf{S}(\Sigma, \mathcal{E})$ is performed by term (or, string diagram) rewriting
\vfill
\pause
String diagram rewriting is formalised as DPO rewriting in a particular category of \alert{hypergraphs} by a series of works by~\cite{bonchi-lics}
\vfill
\pause
We built on these results and formalise string diagram rewriting for $\mathbf{S}^{+}(\Sigma, \mathcal{E})$ as DPO rewriting in a particular category of \alert{e-hypergraphs}
\end{frame}

\begin{frame}{Hypergraphs vs E-Hypergraphs}
\vspace{-1em}
\begin{minipage}{0.3\linewidth}
    \centering
    $\llbracket - \rrbracket : \mathbf{S}(\Sigma) \to \mathbf{HypI}(\Sigma)$
    \[
    \scalebox{0.7}{
    \tikzfig{./figures/cospan_hypergraphs_example}
    }
    \]
    $\textsf{sym};e_1\otimes e_2 \equiv e_2 \otimes e_1;\textsf{sym}$
\end{minipage}
\hfill
\begin{minipage}{0.3\linewidth}
    \begin{itemize}
    \item Additional relations $\leq$ and $\smile$ to make the dashed edges well-defined
\end{itemize}
    \[
    \begin{array}{ccc}
        e_1 <^{\mu} v_3 & v_3 \consistency v_4 & v_3 \not \consistency v_5\\
        \ldots & \ldots & \ldots\\
        e_1 <^{\mu} w_4 & w_3 \consistency w_4 & v_4 \not \consistency w_4
    \end{array}
    \]
\begin{itemize}
    \item Additional interfaces
    \item Pushouts that we need to exist do \alert{exist}
\end{itemize}
\end{minipage}
\hfill
\begin{minipage}{0.35\linewidth}
    \centering
    $\llbracket - \rrbracket : \mathbf{S}^{+}(\Sigma) \to \mathbf{EHypI}(\Sigma)$
    \vspace{-1em}
    \[
    \scalebox{0.5}{
        \tikzfig{./figures/extended_cospan_example}
    }    
    \]
    \vspace{-1em}
    $\textsf{sym} + f \equiv f + \textsf{sym}$
\end{minipage}
% Compare definitions side by side highlighting what is new
\end{frame}

\begin{frame}{Combinatorial semantics}
Double pushout (DPO) rewriting
\vfill
\begin{minipage}{0.45\linewidth}
    \[
    \scalebox{0.8}{
    \begin{tikzcd}[ampersand replacement=\&]
        {\mathcal{L}} \& {i+j} \& {\mathcal{R}} \\
        {\mathcal{G}} \& {\mathcal{L}^{\bot}} \& {\mathcal{H}} \\
        \& {n+m}
        \arrow["f"', from=1-1, to=2-1]
        \arrow[from=1-2, to=1-1]
        \arrow[from=1-2, to=1-3]
        \arrow[from=1-2, to=2-2]
        \arrow[from=1-3, to=2-3]
        \arrow["{\lrcorner}"{rotate=90,xshift=-0.25em, yshift=1em,font = \Large}, draw=none, from=2-1, to=1-2]
        \arrow[from=2-2, to=2-1]
        \arrow[from=2-2, to=2-3]
        \arrow["{\lrcorner}"{rotate=180,xshift=-0.9em, yshift=2.25em,font = \Large}, draw=none, from=2-3, to=1-2]
        \arrow[from=3-2, to=2-1]
        \arrow[from=3-2, to=2-2]
        \arrow[from=3-2, to=2-3]
    \end{tikzcd}
    }
\]
\end{minipage}
\hfill
\begin{minipage}{0.45\linewidth}
\[
 \scalebox{0.7}{
    \tikzfig{../../figures/combinatorial_semantics/DPOI_square}
 }
\]
\end{minipage}
\end{frame}

\begin{frame}{Combinatorial semantics}
\begin{minipage}{0.45\linewidth}
    \[
    \scalebox{0.8}{
    \begin{tikzcd}[ampersand replacement=\&]
        {\mathcal{L}} \& {i+j} \& {\mathcal{R}} \\
        {\mathcal{G}} \& {\mathcal{L}^{\bot}} \& {\mathcal{H}} \\
        \& {n+m}
        \arrow["f"', from=1-1, to=2-1]
        \arrow[from=1-2, to=1-1]
        \arrow[from=1-2, to=1-3]
        \arrow[from=1-2, to=2-2]
        \arrow[from=1-3, to=2-3]
        \arrow["{\lrcorner}"{rotate=90,xshift=-0.25em, yshift=1em,font = \Large}, draw=none, from=2-1, to=1-2]
        \arrow[from=2-2, to=2-1]
        \arrow[from=2-2, to=2-3]
        \arrow["{\lrcorner}"{rotate=180,xshift=-0.9em, yshift=1.7em,font = \Large}, draw=none, from=2-3, to=1-2]
        \arrow[from=3-2, to=2-1]
        \arrow[from=3-2, to=2-2]
        \arrow[from=3-2, to=2-3]
    \end{tikzcd}
    }
\]
\end{minipage}
\hfill
\begin{minipage}{0.45\linewidth}
    \[
 \scalebox{0.7}{
    \tikzfig{./figures/DPOI_square_colored}
 }
\]
\end{minipage}
\end{frame}

\begin{frame}
    \begin{example}
\[
    \adjustbox{width=0.65\linewidth}{
        \tikzfig{../../figures/appendix/boundary_complement_example}
    }
    \]
    \end{example}
\end{frame}

\begin{frame}{Soundness and completeness}
    \[\mathbf{S}(\Sigma) \cong \mathbf{HypI}(\Sigma)\qquad \text{(\cite{bonchi-lics})}\]
    \vfill
    To make $\textbf{EHypI}(\Sigma)$ sound and complete for $\textbf{S}^{+}(\Sigma)$ we need to turn it into a semilattice-enriched category
    \vfill
    We define $+$ on cospans of e-hypergraphs
    \vfill
    Introduce DPO rewrite rules so that it behaves like a semilattice $+$
    \vfill
    Quotient morphisms of $\textbf{EHypI}(\Sigma)$ by these rules
    \vfill
    \[\mathbf{S}^{+}(\Sigma) \cong \textbf{EHypI}(\Sigma) / \mathcal{S} \qquad \mathbf{S}^{+}(\Sigma, \mathcal{E}) \cong \textbf{EHypI}(\Sigma) / \mathcal{S},\mathcal{E}\]
\end{frame}

\begin{frame}
    \vfill
    We can then interpret e-graphs as cospans of e-hypergraphs for a free PROP presented by Cartesian $\textbf{SMT}(\Sigma,\mathcal{E})$ $\scalebox{0.3}{
        \begin{tikzpicture}
            \begin{pgfonlayer}{nodelayer}
                \node [style=vertex] (0) at (1.5, 3.5) {};
                \node [style=none] (1) at (1.5, 3) {};
                \node [style=none] (2) at (1, 4) {};
                \node [style=none] (3) at (2, 4) {};
            \end{pgfonlayer}
            \begin{pgfonlayer}{edgelayer}
                \draw (0) to (1.center);
                \draw [bend right=45] (0) to (3.center);
                \draw [bend left=45] (0) to (2.center);
            \end{pgfonlayer}
        \end{tikzpicture}
    } : 1 \to 2$
    $\scalebox{0.4}{
        \begin{tikzpicture}
            \begin{pgfonlayer}{nodelayer}
                \node [style=vertex] (0) at (1.5, 3.5) {};
                \node [style=none] (1) at (1.5, 3) {};
            \end{pgfonlayer}
            \begin{pgfonlayer}{edgelayer}
                \draw (0) to (1.center);
            \end{pgfonlayer}
        \end{tikzpicture}
    } : 1 \to 0$
    \vfill
    \[
        \tikzfig{../../figures/categorical-semantics/egraph-translation-1}
    \]
    \end{frame}

\begin{frame}
    E-graph operations are then interpreted as DPO rewrites
    \vfill
    \[
    \adjustbox{scale=1.2}{
        \begin{tikzcd}[ampersand replacement=\&, column sep=small]
            e_1 \arrow[r, "\leadsto"] \arrow[d, shift right=2]                \& e_2 \arrow[d]                                        \\
            \llbracket e_1 \rrbracket \arrow[r, "\Rrightarrow^{*}"] \arrow[u] \& \llbracket e_{2} \rrbracket \arrow[u, shift right=2]
        \end{tikzcd}
    }
    \]
\end{frame}

\begin{frame}
    \begin{example}[$(a * 2) / 2$]
        \[
        \tikzfig{./figures/e-string-example-a}
        \]
    \end{example}
\end{frame}

\begin{frame}
\begin{example}
    \[
    \adjustbox{width=\linewidth}{
    \tikzfig{../../figures/categorical-semantics/egraph-translation-step-by-step-a-b}
    }
    \]
\end{example}
\end{frame}

\begin{frame}

\begin{example}
    \begin{minipage}{0.25\linewidth}
\[
\adjustbox{width=\linewidth}{
\tikzfig{./figures/e-string-example-b}
}
\]
    \end{minipage}
\hfill
\begin{minipage}{0.05\linewidth}
    \[
    \xRightarrow{}
    \]
\end{minipage}
\hfill
\begin{minipage}{0.25\linewidth}
\[
\adjustbox{width=\linewidth}{
    \tikzfig{./figures/e-string-example-c}
}
\]
\end{minipage}
\hfill
\begin{minipage}{0.05\linewidth}
    \[
    \xRightarrow{}
    \]
\end{minipage}
\hfill
\begin{minipage}{0.25\linewidth}
    \[
    \adjustbox{width=\linewidth}{
    \tikzfig{./figures/e-string-example-d}
    }
    \]
    \end{minipage}
\end{example}
\end{frame}

\begin{frame}
\begin{example}
\begin{minipage}{0.35\linewidth}
        \[
        \adjustbox{width=\linewidth}{
        \tikzfig{./figures/e-string-example-d}
        }
        \]
\end{minipage}
\hfill
\begin{minipage}{0.2\linewidth}
    \[
    \xRightarrow{}
    \]
\end{minipage}
\hfill
\begin{minipage}{0.35\linewidth}
    \[
    \adjustbox{width=\linewidth}{
    \tikzfig{./figures/e-string-example-e}
    }
    \]
\end{minipage}
\end{example}
\end{frame}

\begin{frame}{New avenues}
\vfill
What we did above goes beyond just Cartesian $\mathbf{SMT}$s
\vfill
E-graphs struggle with monoidal theories
\begin{itemize}
\item Can encode morphisms
\begin{itemize}
    \item $(\mu;(\sigma_{1,1};i \otimes id_1)) \otimes (\mu ; (id_1 \otimes id_1))$
    \item $x : A, x' : A \vdash \{\text{match}_{\otimes}(\mu(x),zy.\{i,z\}), \text{match}_{\otimes}(\mu(x'),yz.\{y,z\})\}$
\end{itemize}
\item Saturating with respect to the axioms of SMC (in the first case) or universal equalities and susbtitution equalities (in the second case) leads to a blow-up in the number of e-nodes
\end{itemize}
\vfill
Hypergraphs (and therefore e-hypergraphs) absorb these axioms 
\end{frame}

\begin{frame}
    \small
    \begin{example}
        \vspace{1em}
        E-graph for
        \[
            (\mu\;;(\sigma_{1,1}\;;i \otimes id_{1})) \otimes (\mu\;;(id_{1}\otimes id_{1}))
        \]

        Before saturation
        \vspace{-3em}
        \begin{figure}
            \includegraphics[scale=0.4]{figures/egraph_before_saturation_fix.pdf}
        \end{figure}
    \end{example}
\end{frame}

\begin{frame}
    \small
    \begin{example}
        \vspace{1em}
        E-graph for
        \[
            (\mu\;;(\sigma_{1,1}\;;i \otimes id_{1})) \otimes (\mu\;;(id_{1}\otimes id_{1}))
        \]
        After saturation
        \begin{figure}
            \includegraphics[width=0.9\linewidth]{figures/dot_5.jpeg}
        \end{figure}
        
        \end{example}
\end{frame}

\begin{frame}{New avenues}

In some cases the free enrichment helps to justify certain equalities
\vfill
\begin{equation*}
	\begin{aligned}
		\Lambda_{A,B,C}(f + g) & = \eta_A;(B \multimap (f + g))                    \\
		                       & = \eta_A;(B \multimap f + B \multimap g)          \\
		                       & = \eta_A;(B \multimap f) + \eta_A;(B \multimap g) \\
		                       & = \Lambda_{A,B,C}(f) + \Lambda_{A,B,C}(g).
	\end{aligned}%
	\label{law:distributivity}
\end{equation*}
\vfill
That emerged in e-graphs extensions~\cite{slotted-egraphs}

\begin{equation*}%
    \label{eq:cong-bind}
    \inferrule*[right=cong. (bind)]
    {E \vdash tc \cong tc'}
    {E \vdash \text{bind}\; \$x\; tc \cong \text{bind}\; \$x\; tc'},
\end{equation*}
\end{frame}

\begin{frame}{Linear substitution $\lambda$-calculus}

Hypergraphs absorb certain equations of these calculi, make the support for variables and substitution more natural
\vfill
\small
\begin{tabular}{lcc}
    \texttt{beta}                  & $(\lambda x . t) u = t[x / u]$                       &                                                                    \\
    \texttt{subst\_same\_var}      & $x[x / u] = u$                                       &                                                                    \\
    \texttt{subst\_diff\_var}      & $y[x / u] = y$                                       & if $x \not \in \mathcal{F}(u)$                                     \\
    \texttt{subst\_application}    & $(uu')[x / v] = u[x / v]u'[x / v]$                   &                                                                    \\
    \texttt{subst\_lambda}         & $(\lambda x . u)[y / v] = \lambda x . (u[y / v])$    & if $x \not \in \mathcal{F}(v)$                                     \\
    \texttt{subst\_interchange\_1} & $(u[x / v])[x' / v'] = (u[x' / v'])[x / v[x' / v']]$ & if $x' \in \mathcal{F}(v)$ and $x \in \mathcal{F}(v')$             \\
    \texttt{subst\_interchange\_2} & $(u[x / v])[x' / v'] = (u[x / v[x' / v']])$          & if $x' \not \in \mathcal{F}(u)$                                    \\
    \texttt{subst\_comm}           & $(u[x / v])[x' / v'] = (u[x' / v'])[x / v]$          & if $x \not \in \mathcal{F}(v')$ and $x' \not \in \mathcal{F}(v)~.$ \\
\end{tabular}

\end{frame}

\begin{frame}
$\texttt{plus}(2,2)$ using Church encoding in slotted e-graphs
\vfill
\begin{tabular}{lcccc}
    Set of equations & explicit subst & explicit subst & explicit subst & explicit subst \\
                     &                & + comm         & + interchange  & + interchange  \\
                     &                &                &                & + comm         \\
    \# of e-nodes    & 161            & 288            & 289            & 356
\end{tabular}

\end{frame}   

\begin{frame}{Conclusion}
We introduced a generalisation of e-graphs for symmetric monoidal theories
\vfill
Introduced DPO rewriting semantics for e-graphs
\vfill
Showed some applications
\vfill
Future work includes seeking other applications, making a more concrete comparison with existing e-graph frameworks, designing term calculus
\end{frame}

\end{document}