%!LW recipe=pdflatex -> bibtex -> pdflatex * 2
\documentclass[acmsmall, screen, nonacm]{acmart}



\newcommand*{\ONECOLUMN}{}

% % !TEX root = ../main.tex

\usepackage{docmute}
\usepackage{hyperref}
\usepackage{proof}

\usepackage{mathtools}
\usepackage{tikz}
\usepackage{tikz-network}
\usepackage{xspace}
\usepackage{xargs}
\usepackage{fontenc}

%\usepackage{minted}

\usepackage{subcaption}
\usepackage{caption}

\usepackage{tikz-cd}
\usepackage{tikzit}
% to reduce tikz picture white space
\usepackage{adjustbox}

% for quotienting
\usepackage{xfrac} 

\usepackage{import}

% category names
%\newcommand{\MdaCospans}{{\catname{MACsp_{D}(Hyp_{\Sigma})}}}

\newcommand\HypI[1]{\textbf{HypI}(#1)}
\newcommand\MdaCospans{\textbf{MHypI}(\Sigma)}
\newcommand{\Ecospans}{{\catname{EHypI(\Sigma)}}}
\newcommand{\MdaEcospans}{{\catname{MEHypI(\Sigma)}}}
\newcommand{\WellTypedMdaEcospans}{{\catname{MEHypI(\Sigma)}}}
% conflict
\newcommand{\hashtag}{{\#}}
\newcommand{\consistency}{{\smile}}
% There is already a remark in the preamble, but it does
% not work for some reason
\newtheorem{remark}{Remark}


\usepackage{xifthen}
\ifthenelse{\isundefined{\ismain}}{
  \newcommand{\ismain}{0}
}{}

\usepackage{enumitem}
 
\newcommand{\ignora}[1]{ }

\ignora{
\usepackage{aliascnt}
% alias-counters
\newaliascnt{definition}{thm}
\newaliascnt{proposition}{thm} 
\newaliascnt{lemma}{thm}
\newaliascnt{corollary}{thm}
\newaliascnt{conjecture}{thm}
\newaliascnt{remark}{thm}
\newaliascnt{example}{thm}
\newaliascnt{examples}{thm}
\newaliascnt{assumption}{thm}
\newaliascnt{construction}{thm}
\newaliascnt{claim}{thm}

% \theoremstyle{plain}
\theoremstyle{definition}

\newtheorem{theorem}[thm]{Theorem}
\newtheorem{proposition}[proposition]{Proposition} 
\aliascntresetthe{proposition}
\newtheorem{lemma}[lemma]{Lemma}
\aliascntresetthe{lemma}
\newtheorem{corollary}[corollary]{Corollary}
\aliascntresetthe{corollary}
\newtheorem{conjecture}[conjecture]{Conjecture}
\aliascntresetthe{conjecture}
\newtheorem{remark}[remark]{Remark}
\aliascntresetthe{remark}
\newtheorem{assumption}[assumption]{Assumption}
\aliascntresetthe{assumption}
\newtheorem{construction}[construction]{Construction}
\aliascntresetthe{construction}
\newtheorem{claim}[claim]{Claim}
\aliascntresetthe{claim}

\theoremstyle{definition}
\newtheorem{definition}[definition]{Definition}
\aliascntresetthe{definition}
\newtheorem{example}[example]{Example}
\aliascntresetthe{example}
\newtheorem{examples}[examples]{Examples}
\aliascntresetthe{examples}
}

\usepackage{cleveref} % after hyperref!
% possibly new thm environment
% \newtheorem{satz}[thm]{Satz}


% \newcommand{\TODO}[1]{{{\color{red}[\textbf{TODO:} #1]}}}
% \newcommand{\TODO}[1]{}


\usepackage{hyperref}

\usepackage{amsthm}

\usepackage{mathtools}
\usepackage{tikz}
\usepackage{tikz-network}
\usepackage{xspace}
\usepackage{xargs}
\usepackage{fontenc}

\usepackage{subcaption}
\usepackage{caption}

\usepackage{tikz-cd}
\usepackage{tikzit}
% to reduce tikz picture white space
\usepackage{adjustbox}

% for quotienting
\usepackage{xfrac} 

\usepackage{import}



\newcommand\HypI[1]{\textbf{HypI}(#1)}
\newcommand\MdaCospans{\textbf{MHypI}(\Sigma)}
\newcommand{\Ecospans}{{\catname{EHypI(\Sigma)}}}
\newcommand{\MdaEcospans}{{\catname{MEHypI(\Sigma)}}}
\newcommand{\WellTypedMdaEcospans}{{\catname{MEHypI(\Sigma)}}}
% conflict
\newcommand{\hashtag}{{\#}}
\newcommand{\consistency}{{\smile}}
% There is already a remark in the preamble, but it does
% not work for some reason
\newcommand\obj[1]{{\textbf{obj}(#1)}}


\usepackage{xifthen}
\ifthenelse{\isundefined{\ismain}}{
  \newcommand{\ismain}{0}
}{}

\usepackage{enumitem}
 
\newcommand{\ignora}[1]{ }

% \newcounter{thm}

% % \ignora{
% \usepackage{aliascnt}
% % alias-counters
% \newaliascnt{definition}{thm}
% \newaliascnt{proposition}{thm} 
% \newaliascnt{lemma}{thm}
% \newaliascnt{corollary}{thm}
% \newaliascnt{conjecture}{thm}
% \newaliascnt{remark}{thm}
% \newaliascnt{example}{thm}
% \newaliascnt{examples}{thm}
% \newaliascnt{assumption}{thm}
% \newaliascnt{construction}{thm}
% \newaliascnt{claim}{thm}

% % \theoremstyle{plain}
% \theoremstyle{definition}

% \newtheorem{theorem}[thm]{Theorem}
% \newtheorem{proposition}[proposition]{Proposition} 
% \aliascntresetthe{proposition}
% \newtheorem{lemma}[lemma]{Lemma}
% \aliascntresetthe{lemma}
% \newtheorem{corollary}[corollary]{Corollary}
% \aliascntresetthe{corollary}
% \newtheorem{conjecture}[conjecture]{Conjecture}
% \aliascntresetthe{conjecture}
% \newtheorem{remark}[remark]{Remark}
% \aliascntresetthe{remark}
% \newtheorem{assumption}[assumption]{Assumption}
% \aliascntresetthe{assumption}
% \newtheorem{construction}[construction]{Construction}
% \aliascntresetthe{construction}
% \newtheorem{claim}[claim]{Claim}
% \aliascntresetthe{claim}


% \theoremstyle{definition}
% \newtheorem{definition}[definition]{Definition}
% \aliascntresetthe{definition}
% \newtheorem{example}[example]{Example}
% \aliascntresetthe{example}
% \newtheorem{examples}[examples]{Examples}
% \aliascntresetthe{examples}
% }

\newtheorem{theorem}{Theorem}[section]
\newtheorem{definition}[theorem]{Definition}
\newtheorem{example}[theorem]{Example}
\newtheorem{proposition}[theorem]{Proposition}
\newtheorem{remark}[theorem]{Remark}

\usepackage{cleveref} % after hyperref!

% !TEX root = ../main.tex

\newcommand{\missing}[1]{{\color{red}\bfseries [TODO]}}

\newcommand{\egg}{\texorpdfstring{\MakeLowercase{\texttt{egg}}}{\texttt{egg}}\xspace}
\newcommand{\Egg}{\texorpdfstring{\MakeLowercase{\texttt{egg}}}{\texttt{egg}}\xspace}
% \newcommand{\egg}{Lego\xspace}
% \newcommand{\Egg}{Lego\xspace}
\newcommand{\egraphs}{\mbox{e-graphs}\xspace}
\newcommand{\egraph}{\mbox{e-graph}\xspace}
\newcommand{\Egraph}{\mbox{E-graph}\xspace}
\newcommand{\Egraphs}{\mbox{E-graphs}\xspace}
\newcommand{\eclass}{\mbox{e-class}\xspace}
\newcommand{\Eclass}{\mbox{E-class}\xspace}
\newcommand{\enode}{\mbox{e-node}\xspace}
\newcommand{\eclasses}{\mbox{e-classes}\xspace}
\newcommand{\enodes}{\mbox{e-nodes}\xspace}
\newcommand{\Enodes}{\mbox{E-nodes}\xspace}
% \newcommand{\regraph}{Regraph\xspace}
% \newcommand{\Regraph}{Regraph\xspace}
\newcommand{\sz}{Szalinski\xspace}
\newcommand{\find}{\texttt{find}\xspace}

\newcommand{\equivid}{\equiv_{\sf id}}
\newcommand{\equivnode}{\equiv_{\sf node}}
\newcommand{\equivterm}{\equiv_{\sf term}}

\newcommand{\congrinv}{$\mathcal{I}_c$\xspace}

\newcommand{\egglogo}[1][]{\protect\includegraphics[height=1em, #1]{egg.png} }
\newcommand{\eggurl}{\url{https://github.com/mwillsey/egg}}

% TODOs
% https://tex.stackexchange.com/questions/9796/how-to-add-todo-notes

\newcommand{\eggtodo}[1]{\textcolor{Red}{\textbf{[CITE: #1]}}}

\newcommand{\defTodo}[2]{%
  \expandafter\newcommand\csname #1\endcsname[1]{%
    \todo[linecolor=#2,backgroundcolor=#2!25,bordercolor=#2,inline,size=\tiny]{\textbf{#1}: ##1}}}

\newcommand{\defTODO}[2]{%
  \expandafter\newcommand\csname #1\endcsname[1]{%
    \todo[linecolor=#2,backgroundcolor=#2!25,bordercolor=#2,inline,size=\tiny,caption={\textbf{(#1 LONG TODO)}}]{##1}}}

\newcommand{\AlekseiColor}{RoyalBlue}

% examples
% \newcommand{\RemyColor}{Red}
% \newcommand{\OliverColor}{OliveGreen}
% \newcommand{\ChandraColor}{Magenta}
% \newcommand{\PavelColor}{Cerulean}
% \newcommand{\ZachColor}{Plum}
% \newcommand{\BenColor}{Orchid}
% \newcommand{\JamesColor}{Salmon}


\defTodo{Aleksei}{\AlekseiColor}
% \defTodo{Max}{\MaxColor}

\defTODO{ALEKSEI}{\AlekseiColor}
% \defTODO{MAX}{\MaxColor}


% categorical stuff

\newcommand{\catname}[1]{\mathbf{#1}}

\newcommand{\Graph}{\catname{Graph}}
\newcommand{\Set}{\catname{Set}}

\newcommand\id{\textsf{id}}
\newcommand\sym{\textsf{sym}}

% tiny inline string diagrams
\newcommand{\comonoid}{%
	\begin{tikzpicture}[baseline=0.4ex, scale=0.4, line width=0.8pt]
		\path [use as bounding box] (0,0) rectangle (0.6,0.8);
		\draw (0.2,0) -- (0.2,0.4);
		\filldraw[black] (0.2,0.4) circle (0.08);
		\draw (0,0.8) .. controls (0,0.65) and (0.1,0.45) .. (0.2,0.4);
		\draw (0.4,0.8) .. controls (0.4,0.65) and (0.3,0.45) .. (0.2,0.4);
	\end{tikzpicture}%
}
\newcommand{\counit}{%
	\begin{tikzpicture}[baseline=0.4ex, scale=0.4, line width=0.8pt]
		\path [use as bounding box] (0,0) rectangle (0.4,0.8);
		\draw (0.2,0) -- (0.2,0.4);
		\filldraw[black] (0.2,0.4) circle (0.08);
	\end{tikzpicture}%
}

% camera-ready
\newcommand{\ifcameraready}[2]{\ifdefined\cameraready #2 \else #1 \fi}

\input{../sample.tikzstyles}
\input{../hypergraph.tikzstyles}
\input{../hypergraph.tikzdefs}


\definecolor{applegreen}{rgb}{0.55, 0.71, 0.0}
\definecolor{americanrose}{rgb}{1.0, 0.01, 0.24}
\definecolor{atomictangerine}{rgb}{1.0, 0.6, 0.4}
\definecolor{azure}{rgb}{0.0, 0.5, 1.0}

% These are for comments
\newcommand\question[1]{{\color{azure}#1}}
\newcommand\update[1]{{\color{americanrose}#1}}

\newcommand\enriched[1]{{\overline{\mathcal{#1}}}}


\makeatletter
\newsavebox{\@brx}
\newcommand{\llangle}[1][]{\savebox{\@brx}{\(\m@th{#1\langle}\)}%
  \mathopen{\copy\@brx\kern-0.5\wd\@brx\usebox{\@brx}}}
\newcommand{\rrangle}[1][]{\savebox{\@brx}{\(\m@th{#1\rangle}\)}%
  \mathclose{\copy\@brx\kern-0.5\wd\@brx\usebox{\@brx}}}
\makeatother


%%
%% \BibTeX command to typeset BibTeX logo in the docs
\AtBeginDocument{%
  \providecommand\BibTeX{{%
    Bib\TeX}}}

%% Rights management information.  This information is sent to you
%% when you complete the rights form.  These commands have SAMPLE
%% values in them; it is your responsibility as an author to replace
%% the commands and values with those provided to you when you
%% complete the rights form.
% \setcopyright{acmlicensed}
% \copyrightyear{2024}
% \acmYear{2024}
% \acmDOI{XXXXXXX.XXXXXXX}

%% These commands are for a PROCEEDINGS abstract or paper.

%%
%%  Uncomment \acmBooktitle if the title of the proceedings is different
%%  from ``Proceedings of ...''!
%%
%%\acmBooktitle{Woodstock '18: ACM Symposium on Neural Gaze Detection,
%%  June 03--05, 2018, Woodstock, NY}
% \acmISBN{978-1-4503-XXXX-X/18/06}


%%
%% Submission ID.
%% Use this when submitting an article to a sponsored event. You'll
%% receive a unique submission ID from the organizers
%% of the event, and this ID should be used as the parameter to this command.
%%\acmSubmissionID{123-A56-BU3}

%%
%% For managing citations, it is recommended to use bibliography
%% files in BibTeX format.
%%
%% You can then either use BibTeX with the ACM-Reference-Format style,
%% or BibLaTeX with the acmnumeric or acmauthoryear sytles, that include
%% support for advanced citation of software artefact from the
%% biblatex-software package, also separately available on CTAN.
%%
%% Look at the sample-*-biblatex.tex files for templates showcasing
%% the biblatex styles.
%%

%%
%% The majority of ACM publications use numbered citations and
%% references.  The command \citestyle{authoryear} switches to the
%% "author year" style.
%%
%% If you are preparing content for an event
%% sponsored by ACM SIGGRAPH, you must use the "author year" style of
%% citations and references.
%% Uncommenting
%% the next command will enable that style.
%%\citestyle{acmauthoryear}

\begin{abstract}
  These notes contain the foundations of categories enriched over a category of semilattices and the respective string diagrams for such categories.
  It introduces the required notions and carefully checks that all properties are satisfied.
\end{abstract}

\begin{document}

\title{Equivalence Hypergraphs: E-Graphs for Monoidal Theories}

\maketitle



\begin{section}{Enriched Categories}

\begin{definition}[Definition 6.2.1~\cite{Borceux_1994}]
  Let $\mathcal{V} = (\mathcal{V}_{0}, I_{\mathcal{V}}, \otimes, l, r, a)$ be a monoidal category where $l$ and $r$ are left and right unitors.
  We define a $\mathcal{V}$-enriched category $\enriched{C}$ consisting of the following data
  \begin{enumerate}
    \item a class of objects $\obj{\enriched{C}}$
    \item for every pair of objects $A,B \in \obj{\enriched{C}}$ --- an (hom-)object $\obj{\enriched{C}}(A,B) \in \mathcal{V}$
    \item for every triple of objects $A,B,C \in \obj{\enriched{C}}$ --- a composition morphism in $\mathcal{V}$
    \[
    c_{A,B,C} : \enriched{C}(A,B) \otimes \enriched{C}(B,C) \to \enriched{C}(A,C)
    \]
    \item for every object $A \in \obj{\enriched{C}}$ --- a unit morphism
    \[
      u_{A} : I_{\mathcal{V}} \to \enriched{C}(A,A)
    \]
  \end{enumerate}
  such that the following diagrams commute (associativity and unitality)
  \[
  \begin{tikzcd}
    {(\enriched{C}(A,B) \otimes \enriched{C}(B,C)) \otimes \enriched{C}(C,D)} \arrow[rr, "{c_{A,B,C} \otimes 1}"] \arrow[d, "{a_{\enriched{C}(A,B),\enriched{C}(B,C),\enriched{C}(C,D)}}"']   &  & {\enriched{C}(A,C) \otimes \enriched{C}(C,D)} \arrow[dd, "{c_{A,C,D}}"]                                                                                     \\
    {\enriched{C}(A,B) \otimes (\enriched{C}(B,C) \otimes \enriched{C}(C,D))} \arrow[d, "{1\otimes c_{B,C,D}}"']                                             &  &                                                                                                                                       \\
    {\enriched{C}(A,B) \otimes \enriched{C}(B,D)} \arrow[rr, "{c_{A,B,D}}"]                                                                       &  & {\enriched{C}(A,D)}                                                                                                                             
  \end{tikzcd}
  \]

  \[
  \begin{tikzcd}
    {I_{\mathcal{V}} \otimes \enriched{C}(A,B)} \arrow[dd, "u_{A} \otimes 1"'] \arrow[r, "{l_{\enriched{C}(A,B)}}"]                                             & {\enriched{C}(A,B)} \arrow[dd, "{1_{\enriched{C}(A,B)}}"] & {\enriched{C}(A,B) \otimes I_{\mathcal{V}}} \arrow[l, "{r_{\enriched{C}(A,B)}}"'] \arrow[dd, "1 \otimes u_{B}"]                                                           \\
    &                                     &                                                                                                                                       \\
{\enriched{C}(A,A) \otimes \enriched{C}(A,B)} \arrow[r, "{c_{A,A,B}}"]                                                                        & {\enriched{C}(A,B)}                            & {\enriched{C}(A,B) \otimes \enriched{C}(B,B)} \arrow[l, "{c_{A,B,B}}"']                                                                                    
\end{tikzcd}
  \]
\end{definition}

\begin{remark}
  We will occasionally use $c_{A,B,C}$ as $;$, \textit{i.e.}, instead of $c_{A,B,C}(f,g)$ we will write $f ; g$ when it makes sense to talk about components of hom-objects.
\end{remark}

\begin{remark}

Requiring that a category $\mathcal{V}$ is monoidal is the most basic requirement for enriched category theory.
To make the notion of enriched category more useful, in particular to be able to talk about a free enrichment, we will further require that $\mathcal{V}$ is a closed symmetric monoidal category.
\end{remark}

\begin{example}
When $\mathcal{V} = \catname{Set}$ we end up with the usual definition of a category.
$c_{A,B,C}$ becomes the usual morphism composition ($f;g$ rather than $g \circ f$) and $u_{A} : I_{\mathcal{V}} \to \enriched{C}(A,A)$ picks out the identity morphism from the homset $\enriched{C}(A,A)$ by matching the only element of a singleton set $I \cong \{*\} \mapsto 1_{A}$ and component-wise the diagrams above commute.
\end{example}

TODO: another example

\begin{definition}[Semilattice]
  A \textit{semilattice} is a set equipped with an operation that we denote as $+$ which is associative, commutative and idempotent.
\end{definition}

Note that we do not require the existence of a unit for $+$. 
Semilattices that satisfy this extra requirement are sometimes called \textit{bounded}, i.e., they are idempotent commutative monoids.

\begin{definition}[Semilattice homomorphism]

A homomorphism between two semilattices $S_{1}$ and $S_{2}$ is a map $h$ that respects $+$.
That is, for all $s,s' \in S_{1}$, $h(s +_{S_{1}} s') = h(s) +_{S_{2}} h(s')$.
\end{definition}

\begin{definition}[Category of semilattices]
  
Semilattices with their respective homomorphisms form a category that we denote $\catname{SLat}$.
\end{definition}

\begin{proposition}
  $\catname{SLat}$ is a closed symmetric monoidal category.
\end{proposition}
\begin{proof}
  The tensor product of two semilattices $S_{1}$ and $S_{2}$ is defined as follows.
  $S_{1} \otimes S_{2}$ consists of pairs $(s_1,s_2)$ $s_{1} \in S_{1}$, $s_{2} \in S_{2}$ quotiented by commutativity, idempotence and associativity and additionally by the following relations
  \begin{itemize}
    \item $(s_{1} +_{S_{1}} s_{1}',s_{2}) \equiv (s_{1},s_{2}) +_{S_{1} \otimes S_{2}} (s_{1}',s_{2})$
    \item $(s_{1}, s_{2} +_{S_{2}} s_{2}') \equiv (s_{1},s_{2}) +_{S_{1} \otimes S_{2}} (s_{1}',s_{2})$
  \end{itemize}

  The unit for this tensor product is $I = \{*\}$ --- a one-element semilattice.
  Clearly $S \otimes I \cong S$ by mapping $(s,*) \mapsto s$ for all $s \in S$ and vice versa.
  The symmetry, associators and unitors are then obvious morphisms.
  Finally, the category is closed since the set of homomorpisms between two semilattices is a semilattice by defining $(f + g)(x)$ as $f(x) + g(x)$.
  $f + g$ is a homomorphism since $(f + g)(x+y) = f(x+y) + g(x+y) = f(x) + f(y) + g(x) + g(y) = f(x) + g(x) + f(y) + g(y) = f(x+y) + g(x+y)$.
\end{proof}

This makes $\catname{SLat}$ a suitable base for enrichment and allows us to take $\mathcal{V} = \catname{SLat}$.
Before we proceed we need to define what an enriched functor is and define what it means to freely enrich a category over $\catname{SLat}$.

\begin{definition}[Isomorphism in $\enriched{C}$. Lemma 3.5.12~\cite{Riehl_2014}]

Let $\enriched{C}$ be a $\mathcal{V}$-enriched category.
Given two objects $A,B \in \enriched{C}$ and a morphism $f : I_{\mathcal{V}} \to \enriched{C}(A,B)$, we call $f$ an isomorphism if there is a $g : I_{\mathcal{V}} \to \enriched{C}(B,A)$ such that
\[
  I_{\mathcal{V}} \cong I_{\mathcal{V}} \otimes I_{\mathcal{V}} \xrightarrow{f \otimes g} \enriched{C}(A,B) \otimes \enriched{C}(B,A) \xrightarrow{c_{A,B,A}} \enriched{C}(A,A)
\]
equals $u_{A} : I_{\mathcal{V}} \to \enriched{C}(A,A)$.
\end{definition}

% TODO: I took this definition from \url{https://mathoverflow.net/questions/326575/isomorphisms-in-enriched-categories#:~:text=equals%20id%3A1%E2%86%92,C(X%2CY).} 
% Need a proper reference. Riehl, for example, only defines isomorphism of $A$ $B$ in terms of objects of the underlying category.

\begin{definition}
  Let $\enriched{C}$ and $\enriched{D}$ be two $\mathcal{V}$-categories.
  A $\mathcal{V}$-enriched functor $F : \enriched{C} \to \enriched{D}$ is defined by the following data.
  \begin{enumerate}
    \item A mapping $F : \obj{\enriched{C}} \to \obj{\enriched{D}}$
    \item An object-indexed family of morphisms in $\mathcal{V}$ $F_{A,B} : \enriched{C}(A,B) \to \enriched{D}(FA,FB)$
  \end{enumerate}
  such that the following diagrams commute

  \[
  \begin{tikzcd}
	{\enriched{C}(A,A') \otimes \enriched{C}(A',A'')} && {\enriched{C}(A,A'')} \\
	\\
	{\enriched{D}(FA,FA') \otimes \enriched{D}(FA',FA'')} && {\enriched{D}(FA,FA'')}
	\arrow["{c_{A,A',A''}}"', from=1-1, to=1-3]
	\arrow["{F_{A,A'} \otimes F_{A',A''}}", from=1-1, to=3-1]
	\arrow["{F_{A,A''}}"', from=1-3, to=3-3]
	\arrow["{c_{FA,FA',FA''}}"', from=3-1, to=3-3]
\end{tikzcd}
\]

\[
\begin{tikzcd}
	I_{\mathcal{V}} && {\enriched{C}(A,A)} \\
	\\
	&& {\enriched{D}(FA,FA)}
	\arrow["{u_{A}}", from=1-1, to=1-3]
	\arrow["{u_{FA}}"', from=1-1, to=3-3]
	\arrow["{F_{A,A}}", from=1-3, to=3-3]
\end{tikzcd}
\]
\end{definition}

\begin{definition}
  Given two $\mathcal{V}$-functors $F,G : \enriched{C} \to \enriched{D}$ a $\mathcal{V}$-natural transformation is family of morphisms indexed by objects in $\enriched{C}$, $\alpha_{A} : I_{\mathcal{V}} \to \enriched{D}(FA,GA)$ in $\mathcal{V}$ such that the following diagram commutes
  \[
  \begin{tikzcd}
	& {\enriched{C}(A,A')\otimes I_{\mathcal{V}}} && {\enriched{D}(FA,FA') \otimes \enriched{D}(FA',GA')} \\
	{\enriched{C}(A,A')} &&&& {\enriched{D}(FA,GA')} \\
	& {I_{\mathcal{V}} \otimes \enriched{C}(A,A')} && {\enriched{D}(FA,GA) \otimes \enriched{D}(GA,GA')}
	\arrow["{G_{A,A'}\otimes \alpha_{A'}}"', from=1-2, to=1-4]
	\arrow["{c_{FA,FA',GA'}}"', from=1-4, to=2-5]
	\arrow["{r^{-1}}", from=2-1, to=1-2]
	\arrow["{l^{-1}}", from=2-1, to=3-2]
	\arrow["{\alpha_{A} \otimes F_{A,A'}}", from=3-2, to=3-4]
	\arrow["{c_{FA,GA,GA'}}", from=3-4, to=2-5]
\end{tikzcd}
\]
\end{definition}

\begin{remark}
  When $\mathcal{V}$ is $\catname{Set}$, a family of morphisms $\alpha_{A} : I_{\mathcal{V}} \to \enriched{C}(FA,GA)$ is defined by mapping $* \mapsto \alpha'_{A}$ where $\alpha'_{A} : FA \to GA$ is usual natural transformation.
  We can then trace the diagram to get

\[\begin{tikzcd}
	& {f \otimes *} && {Gf \otimes \alpha'_{A'}} \\
	f &&&& {Gf;\alpha'_{A} = \alpha'_{A};Ff} \\
	& {* \otimes f} && {\alpha'_{A} \otimes Ff}
	\arrow[from=1-2, to=1-4]
	\arrow[from=1-4, to=2-5]
	\arrow[from=2-1, to=1-2]
	\arrow[from=2-1, to=3-2]
	\arrow[from=3-2, to=3-4]
	\arrow[from=3-4, to=2-5]
\end{tikzcd}\]

which is the usual naturality condition.

\end{remark}

\begin{definition}
Forgetful functor $U : \catname{SLat} \to \catname{Set}$ is given by $\catname{SLat}(I_{\catname{SLat}, -}) : \catname{SLat} \to \catname{Set}$.
\end{definition}

Intuitively, the above functor returns the underlying set of a given semilattice $S$ as each morphism from $\{*\} \to S$ picks out an element of $S$.

\begin{proposition}
  The forgetful functor $U : \catname{SLat} \to \catname{Set}$ has a left adjoint free functor $F : \catname{Set} \to \catname{SLat}$.
\end{proposition}
\begin{proof}
  The functor $F$ is defined by letting $F(A) = \coprod_{A} I_{\catname{SLat}}$ where the latter is a coproduct which is a `free' product of semilattices defined as follows.
  The elements of $S_{1} \coprod S_{2}$ are sequences $s_{1} + s_{2} + \ldots + s_{n}$ where each $s_{i}$ is either from $S_{1}$ or $S_{2}$ quotiented by all relations in $S_{1}$ and $S_{2}$.
  The adjunction is then given by the following natural isomorphisms
  \[
  \catname{SLat}(\coprod_{A}(I_{\catname{SLat}}), B) \cong \prod_{A}(\catname{SLat}(I_{\catname{SLat}}, B)) \cong \catname{Set}(A,\catname{SLat}(I_{\catname{SLat}}, B)) \cong \catname{Set}(A, U(B))
  \]
  The fist isomorphism is given by the fact that hom-functor makes limits into colimits in its first argument, the second being given by the property that $|\catname{Set}(A,B)| = |B|^{|A|} = |\underbrace{B \times \ldots \times B}_{|A|}|$, and the last isomorphism is given by the definition of $U$.
  Furthermore, we have, 
  \[
  F(I) \cong I_{\catname{SLat}}
  \]
  and 
  \begin{align*}
  F(A) \otimes F(B) &\cong (\coprod_{A} I_{\catname{SLat}}) \otimes (\coprod_{B} I_{\catname{SLat}})\\
        &\cong \coprod_{A} (I_{\catname{SLat}} \otimes \coprod_{B} I_{\catname{SLat}})\\
        &\cong \coprod_{A} (\coprod_{B} (I_{\catname{SLat}} \otimes I_{\catname{SLat}}))\\
        &\cong \coprod_{A} (\coprod_{B} I_{\catname{SLat}})\\
        &\cong \coprod_{A \times B} I_{\catname{SLat}}\\
        &\cong F(A \times B)
  \end{align*}
In the above we used the fact that functors $- \otimes X$ and $X \otimes -$ preserve colimits as they are both left-adjoint by symmetry and monoidal closedness of $\catname{SLat}$.
\end{proof}

\begin{remark}
  The above proposition in fact works for any symmetric monoidal closed category $\mathcal{V}$ picked as a base for enrichment (Proposition 6.4.6~\cite{Borceux_1994}) 
\end{remark}

\begin{remark}

  Free semilattice on a set $S$ can be though of as a set of non-empty subsets of $S$.
\end{remark}

\begin{definition}[Proposition 6.4.7~\cite{Borceux_1994}]
\label{def:free_slat}
  Let $\mathcal{C}$ be a small category.
  Freely enriched $\catname{SLat}$-category $\enriched{C}$ is defined as follows
  \begin{enumerate}
    \item $\obj{\enriched{C}} = \obj{\mathcal{C}}$;
    \item $\enriched{C}(A,B) = F(\mathcal{C}(A,B))$, where $F$ is the left adjoint to $U : \catname{SLat} \to \catname{Set}$;
    \item $c_{A,B,C} : \enriched{C}(A,B) \otimes \enriched{C}(B,C) \to \enriched{C}(A,C)$ is the composite $\cong ; F(;)$ below
    \[
    F(\mathcal{C}(A,B)) \otimes F(\mathcal{C}(B,C)) \xrightarrow{\cong} F(\mathcal{C}(A,B) \otimes \mathcal{C}(B,C)) \xrightarrow{F(;)} F(\mathcal{C}(A,C))
    \] 
    where the rightmost arrow is the image of the ordinary composition;
    \item $u_{A} : I_{\catname{SLat}} \to \enriched{C}(A,A)$ is the composite
    \[
    I_{\catname{SLat}} \xrightarrow{\cong} F(I) \to F(\mathcal{C}(A,A))
    \]
    where the rightmost arrow is $* \mapsto 1_{A}$.
  \end{enumerate}
\end{definition}

This is again a specialisation of a more general definition of a free $\mathcal{V}$-category (Proposition 6.4.7~\cite{Borceux_1994}).

We will then consider \textit{only} such freely-enriched categories.

\begin{remark}
  Suppose $\mathcal{C}(A,B) = \{f,g\}$ and $\mathcal{C}(B,C) = \{h\}$, then $f + g \in \enriched{C}(A,B)$ and in particular $(f+g);h = f;h + g;h$ since $c_{A,B,C}$ is a semilattice homomorphism.
\end{remark}

\begin{proposition}
\label{prop:functoriality_preserved}
  A functor $G : \mathcal{C} \to \mathcal{D}$ induces an enriched functor $\overline{G} : \enriched{C} \to \enriched{D}$.
\end{proposition}
\begin{proof}
  We define
  \begin{itemize}
    \item $\overline{G}(A) = G(A)$ for all $A \in \enriched{C}$
    \item $\overline{G}_{A,B}(\enriched{C}(A,B)) = F(G_{A,B}(\mathcal{C}(A,B)))$
  \end{itemize}
  The functoriality follows from the following diagrams
\[
\adjustbox{width=\linewidth}{
\begin{tikzcd}
	{\enriched{C}(A,A')\otimes \enriched{C}(A',A'')} && {\enriched{C}(A,A'')} && {(f, g)} && {f;g} \\
	\\
	{\enriched{D}(GA,GA') \otimes \enriched{D}(GA',GA'')} && {\enriched{D}(GA,GA'')} && {(Gf,Gg)} && {Gf;Gg = G(f;g)}
	\arrow[from=1-1, to=1-3]
	\arrow[from=1-1, to=3-1]
	\arrow[from=1-3, to=3-3]
	\arrow[from=1-5, to=1-7]
	\arrow[from=1-5, to=3-5]
	\arrow[from=1-7, to=3-7]
	\arrow[from=3-1, to=3-3]
	\arrow[from=3-5, to=3-7]
\end{tikzcd}}
\]
where the right-hand diagram is a component render of the left-hand diagram.
It commutes because $G$ is a functor. 
By freeness, the commutativity of more complex components follows from the commutativity for generators.
The following diagrams commute for the same reason.
\[\begin{tikzcd}
	I_{\catname{SLat}} && {\enriched{C}(A,A)} && {*} && {1_{A}} \\
	\\
	&& {\enriched{D}(GA,GA)} &&&& {G(1_{A}) = 1_{G(A)}}
	\arrow[from=1-1, to=1-3]
	\arrow[from=1-1, to=3-3]
	\arrow[from=1-3, to=3-3]
	\arrow[from=1-5, to=1-7]
	\arrow[from=1-5, to=3-7]
	\arrow[from=1-7, to=3-7]
\end{tikzcd}\]
\end{proof}

\begin{remark}
  Again, because $\overline{G}_{A,B}$ is a semilattice homomorphism, we have $\overline{G}(f+g) = \overline{G}(f) + \overline{G}(g)$ for any $\catname{SLat}$-enriched functor $\overline{G}$.
\end{remark}

\begin{example}
  Let $\mathcal{C} = (\mathcal{C}_{0},I,\otimes, l, r, \alpha)$ be a monoidal category and consider the bifunctor $\otimes : \mathcal{C} \times \mathcal{C} \to \mathcal{C}$ on $\mathcal{C}$.
  It induces an enriched monoidal category $\enriched{C} = (\enriched{C}_{0}, I, \enriched{\otimes}, \overline{l}, \overline{r}, \enriched{\alpha})$ such that
  \begin{enumerate}
    \item $A \overline{\otimes} B \coloneq A \otimes B$;
    \item $\overline{l} : A \overline{\otimes} I \to A \coloneq I \to \enriched{C}(A \overline{\otimes} I, I)$, i.e. $\overline{l}$ picks out $l$ out of $\enriched{C}(A \overline{\otimes} I, I)$;
    \item similarly for $\overline{r}$, $\overline{\alpha}$.
  \end{enumerate}
  $\otimes_{A,B,C,D} : \mathcal{C}(A,B) \times \mathcal{C}(C,D) \to \mathcal{C}(A \otimes C, B \otimes D)$ becomes $\overline{\otimes}_{A,B,C,D} : \enriched{C}(A,B) \times \enriched{C}(C,D) \to \enriched{C}(A \overline{\otimes} C, B \overline{\otimes} D)$.
  In particular, $f + g \;\overline{\otimes}\; h + k = (f \;\overline{\otimes}\; h + k) + (g \;\overline{\otimes}\; h + k) = (f \;\overline{\otimes}\; k) + (f \;\overline{\otimes}\; h) + (g \;\overline{\otimes}\; h) + (g \;\overline{\otimes}\; k)$, where $+$ binds tighter than $\overline{\otimes}$.
\end{example}

\begin{proposition}[Proposition 6.2.8~\cite{Borceux_1994}]
  When $F,G$ are $\mathcal{V}$-valued $\mathcal{V}$-functors, the notion of $\mathcal{V}$-natural transformation reduces to
  $\alpha_{A} : FA \to GA$ and the following diagram
  \[\begin{tikzcd}
	{C(A,B)} && {[FA,FB]} \\
	\\
	{[GA,GB]} && {[FA,GB]}
	\arrow["{F_{A,B}}", from=1-1, to=1-3]
	\arrow["{G_{A,B}}"', from=1-1, to=3-1]
	\arrow["{[1,\alpha_{B}]}", from=1-3, to=3-3]
	\arrow["{[\alpha_{A},1]}"', from=3-1, to=3-3]
\end{tikzcd}\]
\end{proposition}

Note the flip of the domain and co-domain of the bottom arrow.
This is because the functor $[-,-] : \enriched{C}^{\text{op}} \otimes \enriched{C} \to \mathcal{V}$ is contravariant in its first argument.

\begin{definition}[$\mathcal{V}$-enriched adjunction]
  Let $\mathcal{V}$ be a closed symmetric monoidal category and $\enriched{C}$, $\enriched{D}$ two $\mathcal{V}$-categories.
  Two $\mathcal{V}$-functors $\overline{L} : \enriched{D} \to \enriched{C}$ and $\overline{R} : \enriched{C} \to \enriched{D}$ are $\mathcal{V}$-adjoint, $\overline{L}$ left adjoint to $\overline{R}$ and $\overline{R}$ right adjoint to $\overline{L}$, when there exist isomorphisms in $\mathcal{V}$
  \[
  \enriched{C}(\overline{L}A,B) \cong \enriched{D}(A, \overline{R}B) \;,
  \]
  which are $\mathcal{V}$-natural in $A$ and $B$.
\end{definition}

Equivalently, one can give a definition of an adjunction in terms of unit and co-unit natural transformations.

\begin{definition}[Section 1.11 in~\cite{Kelly2022BASICCO}]
  A $\mathcal{V}$-adjunction $\eta, \epsilon: \overline{L} \;\reflectbox{$\vdash$}\; \overline{R} : \enriched{C} \to \enriched{D}$ between functors $L : \enriched{D} \to \enriched{C}$ and $\enriched{R} : \enriched{C} \to \enriched{D}$ consists of $\mathcal{V}$-natural $\eta : 1 \to RL$ (the \textit{unit}) and $\epsilon : LR \to 1$ (the \textit{co-unit}) satisfying the triangular equations
  \[
  \overline{R}\epsilon . \eta_{\overline{L}} = 1 \qquad \epsilon_{\overline{L}} . \overline{R} \eta = 1
  \]
\end{definition}
Furthermore we have that the isomorphism $\alpha : \enriched{C}(\overline{L}A,B) \to \enriched{D}(A, \overline{R}B)$ has the form $\enriched{D}(\eta, 1)\overline{R}$:
\[
  \alpha = \enriched{C}(\overline{L}A, B) \xrightarrow{\overline{R}} \enriched{D}(\overline{RL}A, \overline{R}B) \xrightarrow{\enriched{D}(\eta, 1)} \enriched{D}(A, \enriched{R}B)
\]
and $\eta = u_{\overline{L}A};\alpha_{\overline{L}A} = I \xrightarrow{u_{\overline{L}A}}\enriched{C}(\overline{L}A, \overline{L}A) \xrightarrow{\alpha_{\overline{L}A}} \enriched{D}(A,\overline{RL}A)$.
Similarly, $\alpha^{-1} = \enriched{C}(1,\epsilon)\overline{L}$ and $\epsilon = u_{\overline{R}B};\alpha^{-1}_{\overline{R}B}$.

\begin{proposition}
\label{prop:naturality_preserved}
  For a natural transformation $\alpha : T \to S$ for $T,S : \mathcal{C} \to \mathcal{D}$ there is a $\catname{SLat}$-natural transformation $\overline{\alpha} : \overline{T} \to \overline{S}$ where $\overline{T},\overline{S} : \enriched{C} \to \enriched{D}$.
\end{proposition}
\begin{proof}
  Recall that an ordinary natural transformation is in fact a $\catname{Set}$-enriched natural transformation and has the form $\alpha_{A} : I \to \mathcal{D}(TA,SA)$.
  We then obtain an $\catname{SLat}$-natural transformation $\overline{\alpha_{A}} = F(\alpha_{A}) : F(I) \to F(\mathcal{D}(TA,SA))$, or $\overline{\alpha_{A}} : I_{\catname{SLat}} \to \enriched{D}(TA,SA)$ given as $* \mapsto F\alpha_{A}$.
  Again, we can use $TA$ or $\overline{T}A$ interchangeably as their action on objects is the same.
  % The component of a natural transformation $\overline{\alpha}_{A}$ is given by $I_{\catname{SLat}} \to \enriched{D}(\overline{T}A, \overline{S}A)$, i.e. a morphism that picks out the original $\alpha_{A}$.
  We need to check if so defined $\overline{\alpha}$ is $\catname{SLat}$-natural.
  This follows from the following commutativity diagrams
  \[
  \adjustbox{width=\linewidth}{
  \begin{tikzcd}
	&& {\enriched{C}(A,B) \otimes I_{\catname{SLat}}} && {\enriched{D}(\overline{T}A,\overline{T}B) \otimes \enriched{D}(\overline{T}B, \overline{S}B)} &&&&& {f \otimes *} && {\overline{T}f \otimes F\alpha_{B}} \\
	{\enriched{C}(A,B)} &&&&&& {\enriched{D}(\overline{T}A,\overline{S}B)} && f &&&& {F\alpha_{A};\overline{S}f = \overline{T}f ;F\alpha_{B}} \\
	&& {I_{\catname{SLat}} \otimes \enriched{C}(A,B)} && {\enriched{D}(\overline{T}A,\overline{S}A) \otimes \enriched{D}(\overline{S}A,\overline{S}B)} &&&&& {* \otimes f} && {F\alpha_{A} \otimes \overline{S}f}
	\arrow["{\overline{T}_{A,B} \otimes \overline{\alpha_{B}}}", from=1-3, to=1-5]
	\arrow["{c_{\overline{T}A,\overline{T}B,\overline{S}B}}", from=1-5, to=2-7]
	\arrow[from=1-10, to=1-12]
	\arrow[from=1-12, to=2-13]
	\arrow[from=2-1, to=1-3]
	\arrow[from=2-1, to=3-3]
	\arrow[from=2-9, to=1-10]
	\arrow[from=2-9, to=3-10]
	\arrow["{\overline{\alpha_{A}} \otimes \overline{S}(A,B)}", from=3-3, to=3-5]
	\arrow["{c_{\overline{T}A,\overline{S}A,\overline{T}B}}"', from=3-5, to=2-7]
	\arrow[from=3-10, to=3-12]
	\arrow[from=3-12, to=2-13]
\end{tikzcd}}
\]
The diagram on the right is the component-wise rendering of the diagram on the left.
It commutes because it is essentially a lifting of the following diagram (naturality of $\alpha$) using free enrichment functor $F$.

\[
\begin{tikzcd}
	TA && SA && {\overline{T}A} && {\overline{S}A} \\
	\\
	TB && SB && {\overline{T}B} && {\overline{S}B}
	\arrow["{\alpha_{A}}"', from=1-1, to=1-3]
	\arrow["Tf", from=1-1, to=3-1]
	\arrow["Sf"', from=1-3, to=3-3]
	\arrow["{F\alpha_{A}}"', from=1-5, to=1-7]
	\arrow["{\overline{T}f}", from=1-5, to=3-5]
	\arrow["{\overline{S}A}"', from=1-7, to=3-7]
	\arrow["{\alpha_{B}}", from=3-1, to=3-3]
	\arrow["{F\alpha_{B}}", from=3-5, to=3-7]
\end{tikzcd}
\]

The commutativity for more complex elements of a semilattice like $f + g$ follows component-wise.

\end{proof}

\begin{example}

Consider a symmetry natural transformation for a given symmetric monoidal category $\mathcal{C}$ $\sigma_{A,B} : A \otimes B \to B \otimes A$ and its freely $\catname{SLat}$ enriched version $\overline{\sigma_{A,B}} : I_{\catname{SLat}} \to \enriched{C}(A \overline{\otimes} B, B \overline{\otimes} A)$.
$\catname{SLat}$-naturality takes the following form

\[
\adjustbox{width=\linewidth}{
\begin{tikzcd}
	&& {(\enriched{C}(A,B) \times \enriched{C}(C,D)) \otimes I} && {\enriched{C}(A \overline{\otimes} C, B \overline{\otimes} D) \otimes \enriched{C}(B \overline{\otimes} D, D \overline{\otimes} B)} \\
	{\enriched{C}(A,B) \times \enriched{C}(C,D)} &&&&&& {\enriched{C}(A \overline{\otimes} C, D \overline{\otimes} B)} \\
	&& {I \otimes (\enriched{C}(A,B) \times \enriched{C}(C,D))} && {\enriched{C}(A \overline{\otimes} C, C \overline{\otimes} A) \otimes \enriched{C}(C \overline{\otimes} A, D \overline{\otimes} B)} \\
	\\
	&& {(f,g) \otimes *} && {(f \overline{\otimes} g) \otimes \sigma} \\
	{(f,g)} &&&&&& {\sigma;(g \overline{\otimes}f) = (f\overline{\otimes}g);\sigma} \\
	&& {* \otimes (f,g)} && {\sigma \otimes (g \overline{\otimes} f)}
	\arrow["{\overline{\otimes} \otimes \overline{\sigma}_{B,D}}", from=1-3, to=1-5]
	\arrow[from=1-5, to=2-7]
	\arrow[from=2-1, to=1-3]
	\arrow[from=2-1, to=3-3]
	\arrow["{\overline{\sigma}_{C,A} \otimes \overline{\otimes}}"', from=3-3, to=3-5]
	\arrow[from=3-5, to=2-7]
	\arrow[from=5-3, to=5-5]
	\arrow[from=5-5, to=6-7]
	\arrow[from=6-1, to=5-3]
	\arrow[from=6-1, to=7-3]
	\arrow[from=7-3, to=7-5]
	\arrow[from=7-5, to=6-7]
\end{tikzcd}}\]

Given $(f_1 + f_2, g)$ we have $((f_1 + f_2) \;\overline{\otimes}\; g);\sigma = (f_1 \;\overline{\otimes}\; g) + (f_{2} \;\overline{\otimes}\; g);\sigma = (f_{1} \;\overline{\otimes}\; g);\sigma + (f_{2} \;\overline{\otimes}\; g);\sigma = \sigma;(g \;\overline{\otimes}\; f_{1}) + \sigma;(g \;\overline{\otimes}\; f_{2})$

\end{example}


\begin{proposition}
Functor $F : \catname{Set} \to \catname{SLat}$ induces a 2-functor $F^{2}: \catname{Cat} \to \catname{SLat}-\catname{Cat}$ from the category of small categories to the category of small categories enriched over a category of semilattices.
\end{proposition}
\begin{proof}
  The functoriality of $F^{2}$ is given by Definition~\ref{def:free_slat} and 2-functoriality is implied by Propositions~\ref{prop:functoriality_preserved} and \ref{prop:naturality_preserved}.
\end{proof}

The above proposition means, in particular, that an ordinary adjunction gives rise to a corresponding $\catname{SLat}$-enriched adjuntion.
The latter, however, could be also shown equivalently as below.

\begin{proposition}
  Given an adjunction $\mathcal{C}(LA,B) \cong \mathcal{D}(A,RB)$ of plain functors $L : \mathcal{D} \to \mathcal{C}$ and $R : \mathcal{C} \to \mathcal{D}$, we also get an $\catname{SLat}$-enriched adjunction $\enriched{C}(\overline{L}A,B) \cong \enriched{D}(A, \overline{R}B)$ of $\catname{SLat}$-functors $\overline{L} : \enriched{D} \to \enriched{C}$ and $\overline{R} : \enriched{C} \to \enriched{D}$.
  Because $\overline{L}$ acts on objects as $L$ we will further omit the bar.
\end{proposition}
\begin{proof}
  The first adjunction implies the existence of natural isomorphisms (in $A$ and $B$) between $\mathcal{C}(L-,B)$ and $\mathcal{D}(-,RB)$ and $\mathcal{C}(LA,-)$ and $\mathcal{D}(A,R-)$ respectively.
  Naturality in $B$ is by definition the commutativity of the following diagram
  \begin{equation}
    \label{diag:naturality_plain}
  \begin{tikzcd}
	{\mathcal{C}(LA,B)} && {\mathcal{D}(A,RB)} \\
	\\
	{\mathcal{C}(LA,B')} && {\mathcal{D}(A,RB')}
	\arrow["{\alpha_{B}}"', from=1-1, to=1-3]
	\arrow["{\mathcal{C}(LA,f)}"', from=1-1, to=3-1]
	\arrow["{\mathcal{D}(A,Rf)}", from=1-3, to=3-3]
	\arrow["{\alpha_{B'}}", from=3-1, to=3-3]
  \end{tikzcd}
\end{equation}
  and similarly for the naturality in $A$.
  The claim is that $\enriched{C}(LA,B) \cong \enriched{D}(A,RB)$ is an $\catname{SLat}$-adjunction.
  Recall that $\enriched{C}(LA,B) = F(\mathcal{C}(LA,B))$ and similarly for another hom-set and clearly the isomorphism is preserved: given $\alpha$, an isomorphism from $\mathcal{C}(LA,B) \to \mathcal{D}(A,RB)$, the isomorphism from $\enriched{C}(LA,B) \to \enriched{D}(A,RB)$ is $\enriched{\alpha} = F(\alpha)$.
  We need to check that this $\enriched{\alpha}$ is natural.
  This is equivalent to checking if the following diagram commutes
  \[\begin{tikzcd}
	{\enriched{C}(B,B')} && {[\enriched{C}(LA,B),\enriched{C}(LA,B')]} \\
	\\
	{[\enriched{D}(A,RB),\enriched{D}(A,RB')]} && {[\enriched{C}(A,RB),\enriched{D}(A,RB')]}
	\arrow["{\enriched{C}(LA,-)}", from=1-1, to=1-3]
	\arrow["{\enriched{D}(A,R-)}"', from=1-1, to=3-1]
	\arrow["{[1,\overline{\alpha}_{B'}]}", from=1-3, to=3-3]
	\arrow["{[\overline{\alpha_{B}},1]}"', from=3-1, to=3-3]
\end{tikzcd}\]
which component-wise becomes

\[
\begin{tikzcd}
	f & {\enriched{C}(LA,f)} \\
	{\enriched{D}(A,Rf)} & {\overline{\alpha}_{B};\enriched{D}(A,Rf);1 = 1;\enriched{C}(LA,f);\overline{\alpha}_{B}'}
	\arrow[from=1-1, to=1-2]
	\arrow[from=1-1, to=2-1]
	\arrow[from=1-2, to=2-2]
	\arrow[from=2-1, to=2-2]
\end{tikzcd}
\]

The commutativity of the diagram above follows from applying $F$ to the diagram \ref{diag:naturality_plain}:
\[\begin{tikzcd}
	{\enriched{C}(LA,B)} && {\enriched{D}(A,RB)} \\
	\\
	{\enriched{C}(LA,B')} && {\enriched{D}(A,RB')}
	\arrow["{\overline{\alpha}_{B}}", from=1-1, to=1-3]
	\arrow["{\enriched{C}(LA,f)}"', from=1-1, to=3-1]
	\arrow["{\enriched{D}(A,Rf)}", from=1-3, to=3-3]
	\arrow["{\overline{\alpha}_{B'}}", from=3-1, to=3-3]
\end{tikzcd}
\]
The commutativity for more complex elements of $\enriched{C}(B,B')$, \textit{e.g.} $f + g$, follows component-wise.
Similarly for naturality in $A$.
Then, $\catname{Slat}$-naturality separately in $A$ and $B$ implies $\catname{SLat}$-naturality in both $A$ and $B$, a consequence of a more general statement about $\mathcal{V}$-naturality ((1.21) in~\cite{Kelly2022BASICCO})

\end{proof}


\end{section}

\begin{section}{String diagrams for freely $\catname{SLat}$-enriched symmetric monoidal categories}
  
String diagrams are a two-dimensional syntax for describing morphisms in symmetric monoidal categories which is typically given by the following grammar parameterised over a monoidal signature $\Sigma$ of symbols.
The labels on wires specify the types of the morphisms.

\[
\adjustbox{scale=0.7}{
\tikzfig{./figures/string_diagrams}
}
\]

The diagrams constructed by using the rules above are then subject to SMC equations.

By freely enriching home-sets to hom-objects in $\catname{SLat}$ we still can access the morphisms within these objects.
To support `new' morphisms we extend the grammar by adding a production for $f + g$

\[
\adjustbox{scale=0.8}{
\tikzfig{./figures/f_plus_g}
}
\]

Once we consider string diagrams for a $\catname{SLat}$-enriched category $\enriched{C}$, $;$ and $\otimes$ become $c_{A,B,C}$ and $\overline{\otimes}$ respectively.
Since the latter are semilattice homomorphisms, we have the following equations (with labels on wires omitted) for the newly introduced dashed boxes.

\[
\adjustbox{width=\linewidth}{
\tikzfig{./figures/egraph-strings-equations}
}
\]

The fact of $+$ being associative and commutative justifies the usage of $\ldots$ in the equations.

Then we can reuse the diagrams for functorial boxes~\cite{GhicaZanassiStringLambda}.
Given a functor $F : \mathcal{C} \to \mathcal{D}$ and a morphism $f \in C(A,B)$ we can depict the application of $Ff$ as below (left) together with the functoriality equations in Figure~\ref{fig:functoriality_laws}

\[
  \adjustbox{scale=0.7}{
  \tikzfig{./figures/functorial_string}
  }
\]

\begin{figure}
  \begin{subfigure}[c]{0.4\linewidth}
    \[
    \adjustbox{scale=0.7}{
    \tikzfig{./figures/functoriality_eq_1_string}
    }
    \]
  \end{subfigure}
  \hfill
  \begin{subfigure}[c]{0.4\linewidth}
    \[
    \adjustbox{scale=0.7}{
    \tikzfig{./figures/functoriality_eq_2_string}
    }
    \]
  \end{subfigure}
  \caption{String diagrammatic equations for functoriality}
  \label{fig:functoriality_laws}
\end{figure}
\end{section}

A functor $\overline{F}$ induced by the functor $F$ has the same diagrammatic rendering with the addition of an extra equation.

\[
\adjustbox{scale=0.7}{
\tikzfig{./figures/functoriality_enriched_eq}
}
\]

Since a component of a natural transformation $\overline{\alpha}$ between two $\catname{SLat}$-enriched functors $\overline{F} : \enriched{C} \to \enriched{D}$ and $\overline{G}: \enriched{C} \to \enriched{D}$ has the form $\overline{\alpha_{A}} : I_{\catname{SLat}} \to \enriched{D}(FA, GA)$, it can be identified with the element of $\enriched{D}(FA,GA)$, i.e. we can use the syntax for plain morphism to depict components of natural transformations.
This allows us to reuse the whole syntax machinery for Cartesian closed categories for their free $\catname{SLat}$-enriched version which induce an additional equation 

\[
\adjustbox{scale=0.7}{
  \tikzfig{./figures/lambda_enriched_eq}
}
\]

We can similarly lift a trace from a traced monoidal category $\mathcal{C}$ to $\enriched{C}$.
Recall that a trace is an operator
\[
\text{tr}^{X} : \mathcal{C}(A \otimes X, B \otimes X) \to \mathcal{C}(A,B)
\]
\textit{i.e.}, a mapping between hom-sets, such that it satisfies the equations
\begin{enumerate}
  \item \textit{Tightening} (naturality in $A,B$) For all $A, B, C, D, X \in \mathcal{C}$, $g : A \to B$, $f : B \otimes X \to C \otimes X$, $h : C \to D$
  \[
  \text{tr}^{X}((h \otimes id_{X});f;(g \otimes id_{X})) = h;\text{tr}^{X}(f);g
  \]
  \item \textit{Sliding} (naturality in $X$) For all $A,B,X,Y \in \mathcal{C}$, $f : A \otimes X \to B \otimes Y$, $g : Y \to X$
  \[
   \text{tr}^{X}(f;(id_{B} \otimes g)) = \text{tr}^{Y}((id_{A} \otimes g); f) 
  \]
  \item \textit{Vanishing} For all $A,B,X,Y \in \mathcal{C}$,
  \[
  \text{tr}^{I}(f) = f,\; \forall f : A \to B \qquad \text{and} \qquad \text{tr}^{X \otimes Y}(f) = \text{tr}^{X}(\text{tr}^{Y}(f)), \; \forall f : A \otimes X \otimes Y \to B \otimes X \otimes Y
  \]
  \item \textit{Strength} For all $A,B,C,D,X \in \mathcal{C}$, $f : C \otimes X \to D \otimes X$, $g : A \to B$
    \[
    \text{tr}^{X}(g \otimes f) = g \otimes \text{tr}^{X}(f)
    \] 
  \item \textit{Yanking} (when $\mathcal{C}$ is symmetric)
    \[
    \text{tr}^{X}(\sigma_{X,X}) = id_{X}
    \]
\end{enumerate}

By freely enriching $\mathcal{C}$ we obtain an enriched trace
\[
\text{tr}^{X} : \enriched{C}(A \overline{\otimes} X, B \overline{\otimes} X) \to \enriched{C}(A,B)
\]
and all the equations are preserved with the addition of a new equation
\[
  \text{tr}^{X}(f + g) = \text{tr}^{X}(f) + \text{tr}^{X}(g)
\]
for $f,g : A \overline{\otimes} X \to B \overline{\otimes} X$
TODO: trace?

% \[
% \adjustbox{scale=0.95, lap={\width}{-12em}}{
%     \begin{tikzcd}
%         & {\enriched{C}(A_1,A_2)} \arrow[ld] \arrow[rd]                       &                                                                                                                                                                                                            \\
% {I \otimes \enriched{C}(A_1,A_2)} \arrow[dd, "{\alpha'_{A_1} \otimes \overline{\mathcal{D}}(\overline{L}A,-)}"]                                                                    &                                                                               & {\enriched{C}(A_1,A_2) \otimes I} \arrow[dd, "{\enriched{C}(\overline{L}A,-) \otimes \alpha'_{A_2}}"]                                                                                  \\
%         &                                                                               &                                                                                                                                                                                                            \\
% {\mathcal{V}(\enriched{C}(\overline{L}A,A_1),\overline{\mathcal{D}}(A,A_1)) \otimes \mathcal{V}(\overline{\mathcal{D}}(A,A_1),\overline{\mathcal{D}}(A,A_2)))} \arrow[rd, "\circ"] &                                                                               & {\mathcal{V}(\enriched{C}(\overline{L}A,A_1),\enriched{C}(\overline{L}A,A_2))) \otimes \mathcal{V}(\enriched{C}(LA,A_2),\overline{\mathcal{D}}(A,A_2))} \arrow[ld, "\circ"'] \\
%         & {\mathcal{V}(\enriched{C}(\overline{L}A,A_1), \overline{D}(A,A_2))} &                                                                                                                                                                                                           
% \end{tikzcd}
% }
% \]

% \[
%   \begin{tikzcd}
%     {\enriched{C}(B,B')} \arrow[rr, "{\enriched{C}(LA,-)}"] \arrow[d, "{\overline{\mathcal{D}}(A,R-)}"] &  & {[\enriched{C}(LA,B), \enriched{C}(LA,B')]} \arrow[d, "{[1,\overline{\alpha_{B'}}]}"] \\
%     {[\overline{\mathcal{D}}(A,RB),\overline{\mathcal{D}}(A,RB')]} \arrow[rr, "{[\overline{\alpha_{B}},1]}"]                &  & {[\enriched{C}(LA,B),\overline{\mathcal{D}}(A,RB')]}                                           
%     \end{tikzcd}
% \]

% \vspace{4em}

% \[
%   \begin{tikzcd}
%     {\enriched{C}(B,B')} \arrow[rr, "{\enriched{C}(LA,-)}"] \arrow[d, "{\overline{\mathcal{D}}(A,R-)}"] &  & {[\enriched{C}(LA,B), \enriched{C}(LA,B')]} \arrow[d, "{[1,\overline{\alpha_{B'}}]}"]                             \\
%     {[\overline{\mathcal{D}}(A,RB),\overline{\mathcal{D}}(A,RB')]} \arrow[rr, "{[\overline{\alpha_{B}},1]}"]                &  & {[\enriched{C}(LA,B),\overline{\mathcal{D}}(A,RB')]}                                                                        \\
%     f \arrow[rr] \arrow[d]                                                                                                  &  & {\enriched{C}(LA,f)} \arrow[d]                                                                                              \\
%     {\overline{\mathcal{D}}(A,Rf)} \arrow[rr]                                                                               &  & {1;\enriched{C}(LA,f);\overline{\alpha_{B'}} = \overline{\alpha_{B}};\overline{\mathcal{D}}(A,Rf);1}
%     \end{tikzcd}
% \]

\bibliographystyle{acm}
\bibliography{bibliography}

\end{document}