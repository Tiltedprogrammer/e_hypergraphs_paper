%!LW recipe=pdflatex
\documentclass{article}

\usepackage{tikz-cd}
\usepackage{xargs}
\usepackage{amsmath}

% !TEX root = ../main.tex

\usepackage{docmute}
\usepackage{hyperref}
\usepackage{proof}

\usepackage{mathtools}
\usepackage{tikz}
\usepackage{tikz-network}
\usepackage{xspace}
\usepackage{xargs}
\usepackage{fontenc}

%\usepackage{minted}

\usepackage{subcaption}
\usepackage{caption}

\usepackage{tikz-cd}
\usepackage{tikzit}
% to reduce tikz picture white space
\usepackage{adjustbox}

% for quotienting
\usepackage{xfrac} 

\usepackage{import}

% pseudocode
\usepackage{algorithm}
\usepackage{algpseudocode}

% category names
%\newcommand{\MdaCospans}{{\catname{MACsp_{D}(Hyp_{\Sigma})}}}

\newcommand\HypI[1]{\textbf{HypI}(#1)}
\newcommand\MdaCospans{\textbf{MHypI}(\Sigma)}
\newcommand{\Ecospans}{{\catname{EHypI(\Sigma)}}}
\newcommand{\MdaEcospans}{{\catname{MEHypI(\Sigma)}}}
\newcommand{\WellTypedMdaEcospans}{{\catname{MEHypI(\Sigma)}}}
% conflict
\newcommand{\hashtag}{{\#}}
\newcommand{\consistency}{{\smile}}
% There is already a remark in the preamble, but it does
% not work for some reason
\newtheorem{remark}{Remark}


\usepackage{xifthen}
\ifthenelse{\isundefined{\ismain}}{
  \newcommand{\ismain}{0}
}{}

\usepackage{enumitem}
 
\newcommand{\ignora}[1]{ }

\ignora{
\usepackage{aliascnt}
% alias-counters
\newaliascnt{definition}{thm}
\newaliascnt{proposition}{thm} 
\newaliascnt{lemma}{thm}
\newaliascnt{corollary}{thm}
\newaliascnt{conjecture}{thm}
\newaliascnt{remark}{thm}
\newaliascnt{example}{thm}
\newaliascnt{examples}{thm}
\newaliascnt{assumption}{thm}
\newaliascnt{construction}{thm}
\newaliascnt{claim}{thm}

% \theoremstyle{plain}
\theoremstyle{definition}

\newtheorem{theorem}[thm]{Theorem}
\newtheorem{proposition}[proposition]{Proposition} 
\aliascntresetthe{proposition}
\newtheorem{lemma}[lemma]{Lemma}
\aliascntresetthe{lemma}
\newtheorem{corollary}[corollary]{Corollary}
\aliascntresetthe{corollary}
\newtheorem{conjecture}[conjecture]{Conjecture}
\aliascntresetthe{conjecture}
\newtheorem{remark}[remark]{Remark}
\aliascntresetthe{remark}
\newtheorem{assumption}[assumption]{Assumption}
\aliascntresetthe{assumption}
\newtheorem{construction}[construction]{Construction}
\aliascntresetthe{construction}
\newtheorem{claim}[claim]{Claim}
\aliascntresetthe{claim}

\theoremstyle{definition}
\newtheorem{definition}[definition]{Definition}
\aliascntresetthe{definition}
\newtheorem{example}[example]{Example}
\aliascntresetthe{example}
\newtheorem{examples}[examples]{Examples}
\aliascntresetthe{examples}
}

\usepackage{cleveref} % after hyperref!
% possibly new thm environment
% \newtheorem{satz}[thm]{Satz}


% \newcommand{\TODO}[1]{{{\color{red}[\textbf{TODO:} #1]}}}
% \newcommand{\TODO}[1]{}

% !TEX root = ../main.tex

\newcommand{\missing}[1]{{\color{red}\bfseries [TODO]}}

\newcommand{\egg}{\texorpdfstring{\MakeLowercase{\texttt{egg}}}{\texttt{egg}}\xspace}
\newcommand{\Egg}{\texorpdfstring{\MakeLowercase{\texttt{egg}}}{\texttt{egg}}\xspace}
% \newcommand{\egg}{Lego\xspace}
% \newcommand{\Egg}{Lego\xspace}
\newcommand{\egraphs}{\mbox{e-graphs}\xspace}
\newcommand{\egraph}{\mbox{e-graph}\xspace}
\newcommand{\Egraph}{\mbox{E-graph}\xspace}
\newcommand{\Egraphs}{\mbox{E-graphs}\xspace}
\newcommand{\eclass}{\mbox{e-class}\xspace}
\newcommand{\Eclass}{\mbox{E-class}\xspace}
\newcommand{\enode}{\mbox{e-node}\xspace}
\newcommand{\eclasses}{\mbox{e-classes}\xspace}
\newcommand{\enodes}{\mbox{e-nodes}\xspace}
\newcommand{\Enodes}{\mbox{E-nodes}\xspace}
% \newcommand{\regraph}{Regraph\xspace}
% \newcommand{\Regraph}{Regraph\xspace}
\newcommand{\sz}{Szalinski\xspace}
\newcommand{\find}{\texttt{find}\xspace}

\newcommand{\equivid}{\equiv_{\sf id}}
\newcommand{\equivnode}{\equiv_{\sf node}}
\newcommand{\equivterm}{\equiv_{\sf term}}

\newcommand{\congrinv}{$\mathcal{I}_c$\xspace}

\newcommand{\egglogo}[1][]{\protect\includegraphics[height=1em, #1]{egg.png} }
\newcommand{\eggurl}{\url{https://github.com/mwillsey/egg}}

% TODOs
% https://tex.stackexchange.com/questions/9796/how-to-add-todo-notes

\newcommand{\eggtodo}[1]{\textcolor{Red}{\textbf{[CITE: #1]}}}

\newcommand{\defTodo}[2]{%
  \expandafter\newcommand\csname #1\endcsname[1]{%
    \todo[linecolor=#2,backgroundcolor=#2!25,bordercolor=#2,inline,size=\tiny]{\textbf{#1}: ##1}}}

\newcommand{\defTODO}[2]{%
  \expandafter\newcommand\csname #1\endcsname[1]{%
    \todo[linecolor=#2,backgroundcolor=#2!25,bordercolor=#2,inline,size=\tiny,caption={\textbf{(#1 LONG TODO)}}]{##1}}}

\newcommand{\AlekseiColor}{RoyalBlue}

% examples
% \newcommand{\RemyColor}{Red}
% \newcommand{\OliverColor}{OliveGreen}
% \newcommand{\ChandraColor}{Magenta}
% \newcommand{\PavelColor}{Cerulean}
% \newcommand{\ZachColor}{Plum}
% \newcommand{\BenColor}{Orchid}
% \newcommand{\JamesColor}{Salmon}


\defTodo{Aleksei}{\AlekseiColor}
% \defTodo{Max}{\MaxColor}

\defTODO{ALEKSEI}{\AlekseiColor}
% \defTODO{MAX}{\MaxColor}


% categorical stuff

\newcommand{\catname}[1]{\mathbf{#1}}

\newcommand{\Graph}{\catname{Graph}}
\newcommand{\Set}{\catname{Set}}

\newcommand\id{\textsf{id}}
\newcommand\sym{\textsf{sym}}

% tiny inline string diagrams
\newcommand{\comonoid}{%
	\begin{tikzpicture}[baseline=0.4ex, scale=0.4, line width=0.8pt]
		\path [use as bounding box] (0,0) rectangle (0.6,0.8);
		\draw (0.2,0) -- (0.2,0.4);
		\filldraw[black] (0.2,0.4) circle (0.08);
		\draw (0,0.8) .. controls (0,0.65) and (0.1,0.45) .. (0.2,0.4);
		\draw (0.4,0.8) .. controls (0.4,0.65) and (0.3,0.45) .. (0.2,0.4);
	\end{tikzpicture}%
}
\newcommand{\counit}{%
	\begin{tikzpicture}[baseline=0.4ex, scale=0.4, line width=0.8pt]
		\path [use as bounding box] (0,0) rectangle (0.4,0.8);
		\draw (0.2,0) -- (0.2,0.4);
		\filldraw[black] (0.2,0.4) circle (0.08);
	\end{tikzpicture}%
}

% camera-ready
\newcommand{\ifcameraready}[2]{\ifdefined\cameraready #2 \else #1 \fi}

\input{sample.tikzstyles}
\input{hypergraph.tikzstyles}
\input{hypergraph.tikzdefs}

\newcommand\id{\textsf{id}}
\newcommand\sym{\textsf{sym}}

\begin{document}

\title{LICS24 Reviews Response}
\author{}

\maketitle

\section*{Review 1}
\paragraph{About theorem 6.5 (the central result of the paper: full completeness)}
\textit{I am doubtful about the statement because it seems to me that $[sym] = [id]$,
hence the functor $[-]$ is not faithful.
Here is a sketch of the argument: $[\sym]$ is a e-hypergraph with two vertices
where the inputs and the outputs are inverted. Now we embed it into a
hierarchical edge by applying the bottom left equality of Figure 3.
We get a cospan with 2 external inputs, 2 external ouputs, 2 strict internal
inputs, 2 strict internal outputs (the sets of strict internal inputs and
outputs are the same, but they are labelled differently by $f_{int}$ and $g_{int}$). We
do the same thing with $[\id]$ and notice that the two resulting cospans are
isomorphic. Thus they are equal in the category where cospans are quotiented by
isos.}

We assume that in $[\sym] = [\id]$ it was meant that $[\sym] = [\id \otimes \id]$, otherwise the equality does not typecheck.

By the definition $[\sym]$ is 
\[
    n \xrightarrow{f_{ext}} n' \xrightarrow{f_{int}} \mathcal{G} \xleftarrow{g_{int}} m' \xleftarrow{g_{ext}} m
\]
represented as
\begin{figure}[h!]
    \[
    \scalebox{0.5}{
        \tikzfig{reviews/sym}
    }
    \]
\end{figure}

And $[\id \otimes \id]$ is \[
    n \xrightarrow{f'_{ext}} n'' \xrightarrow{f'_{int}} \mathcal{G} \xleftarrow{g'_{int}} m'' \xleftarrow{g'_{ext}} m
\]
represented as 
\begin{figure}[h!]
    \[
    \scalebox{0.5}{
        \tikzfig{reviews/id}
    }
    \]
\end{figure}

Clearly, it should be the case that $\{u_1 \mapsto u_1, \ldots u_4 \mapsto u_4\}$ and similarly for $w_i$. I think this does not follow from our definition of isomorphic cospans, so perhaps we should've required that $\alpha$ and $\gamma$ are identities (below).
\[
    \scalebox{0.75}{
        \tikzfig{combinatorial_semantics/isomorphic_e_cospans}
    }
\]
Then, the cospans above are not isomorphic. Consider the cases.
\begin{enumerate}
    \item ${v_1 \mapsto v_1}$ and ${v_2 \mapsto v_2}$ under $\beta$. Then $g_{int};\beta(w_4) \not = g'_{int}(w_4)$
    \item ${v_1 \mapsto v_2}$ and ${v_2 \mapsto v_1}$ under $\beta$. Then $f_{int};\beta(w_4) \not = f'_{int}(w_4)$
\end{enumerate}
This means that the corresponding diagram does not commute and therefore the cospans are not isomorphic.

\paragraph{In the proof}
In the first case, it is immediate that $[f] \not = [g]$, since equal terms
have weak normal forms with the same number of components.
\textit{This argument does not seem to be valid in presence of equations.
Consider an equation $a = a + b$ for example, where a and b are constants of the
signature. Then}
\[
    [a] = [a + b] = [a] + [b]
\]

We only consider equations of SMC terms, \textit{i.e.}, with no $+$ on either side of $=$.

\end{document}
