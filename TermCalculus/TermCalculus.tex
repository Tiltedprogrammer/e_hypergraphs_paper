\documentclass[manuscript, nonacm, anonymous]{acmart}


\usepackage{amsmath,amsfonts}
\usepackage{algorithmic}
\usepackage{listings}
\usepackage{graphicx}
\usepackage{textcomp}
\usepackage{soul}
\usepackage{xcolor}

\usepackage{amsthm}
\usepackage{hyperref}
\renewcommand{\sectionautorefname}{Section}
\renewcommand{\subsectionautorefname}{Subsection}

\usepackage{cleveref}
\usepackage{stmaryrd}

\usepackage{mathtools}
\usepackage{tikz}
\usepackage{tikz-network}
\usepackage{xspace}
\usepackage{xargs}
\usepackage{fontenc}
\usepackage{csquotes}

\usepackage{subcaption}
\usepackage{caption}

\usepackage{tikz-cd}
\usepackage{tikzit}
% to reduce tikz picture white space
\usepackage{adjustbox}

% for quotienting
\usepackage{xfrac} 
\usepackage{wrapfig}

\usepackage{microtype}

% inference rules
\usepackage{mathpartir}

\usepackage{aliascnt}

% for tables
\usepackage{multirow}
\usepackage{bigdelim}
\usepackage{colortbl}
% code
\usepackage{minted}
\usepackage{subcaption}
\DeclareCaptionSubType{listing}


\newtheorem{theorem}{Theorem}[section]

% definition
\newaliascnt{definition}{theorem}
\newtheorem{definition}[definition]{Definition}
\aliascntresetthe{definition}
\providecommand*{\definitionautorefname}{Definition}

\newaliascnt{example}{theorem}
\newtheorem{example}[example]{Example}
\aliascntresetthe{example}
\providecommand*{\exampleautorefname}{Example}

\newaliascnt{proposition}{theorem}
\newtheorem{proposition}[proposition]{Proposition}
\aliascntresetthe{proposition}
\providecommand*{\propositionautorefname}{Proposition}

\newaliascnt{remark}{theorem}
\newtheorem{remark}[remark]{Remark}
\aliascntresetthe{remark}
\providecommand*{\remarkautorefname}{Remark}

\newaliascnt{lemma}{theorem}
\newtheorem{lemma}[lemma]{Lemma}
\aliascntresetthe{lemma}
\providecommand*{\lemmaautorefname}{Lemma}


% listings for autoref
\providecommand*{\listingautorefname}{Listing}


% !TEX root = ../main.tex

\newcommand{\missing}[1]{{\color{red}\bfseries [TODO]}}

\newcommand{\egg}{\texorpdfstring{\MakeLowercase{\texttt{egg}}}{\texttt{egg}}\xspace}
\newcommand{\Egg}{\texorpdfstring{\MakeLowercase{\texttt{egg}}}{\texttt{egg}}\xspace}
% \newcommand{\egg}{Lego\xspace}
% \newcommand{\Egg}{Lego\xspace}
\newcommand{\egraphs}{\mbox{e-graphs}\xspace}
\newcommand{\egraph}{\mbox{e-graph}\xspace}
\newcommand{\Egraph}{\mbox{E-graph}\xspace}
\newcommand{\Egraphs}{\mbox{E-graphs}\xspace}
\newcommand{\eclass}{\mbox{e-class}\xspace}
\newcommand{\Eclass}{\mbox{E-class}\xspace}
\newcommand{\enode}{\mbox{e-node}\xspace}
\newcommand{\eclasses}{\mbox{e-classes}\xspace}
\newcommand{\enodes}{\mbox{e-nodes}\xspace}
\newcommand{\Enodes}{\mbox{E-nodes}\xspace}
% \newcommand{\regraph}{Regraph\xspace}
% \newcommand{\Regraph}{Regraph\xspace}
\newcommand{\sz}{Szalinski\xspace}
\newcommand{\find}{\texttt{find}\xspace}

\newcommand{\equivid}{\equiv_{\sf id}}
\newcommand{\equivnode}{\equiv_{\sf node}}
\newcommand{\equivterm}{\equiv_{\sf term}}

\newcommand{\congrinv}{$\mathcal{I}_c$\xspace}

\newcommand{\egglogo}[1][]{\protect\includegraphics[height=1em, #1]{egg.png} }
\newcommand{\eggurl}{\url{https://github.com/mwillsey/egg}}

% TODOs
% https://tex.stackexchange.com/questions/9796/how-to-add-todo-notes

\newcommand{\eggtodo}[1]{\textcolor{Red}{\textbf{[CITE: #1]}}}

\newcommand{\defTodo}[2]{%
  \expandafter\newcommand\csname #1\endcsname[1]{%
    \todo[linecolor=#2,backgroundcolor=#2!25,bordercolor=#2,inline,size=\tiny]{\textbf{#1}: ##1}}}

\newcommand{\defTODO}[2]{%
  \expandafter\newcommand\csname #1\endcsname[1]{%
    \todo[linecolor=#2,backgroundcolor=#2!25,bordercolor=#2,inline,size=\tiny,caption={\textbf{(#1 LONG TODO)}}]{##1}}}

\newcommand{\AlekseiColor}{RoyalBlue}

% examples
% \newcommand{\RemyColor}{Red}
% \newcommand{\OliverColor}{OliveGreen}
% \newcommand{\ChandraColor}{Magenta}
% \newcommand{\PavelColor}{Cerulean}
% \newcommand{\ZachColor}{Plum}
% \newcommand{\BenColor}{Orchid}
% \newcommand{\JamesColor}{Salmon}


\defTodo{Aleksei}{\AlekseiColor}
% \defTodo{Max}{\MaxColor}

\defTODO{ALEKSEI}{\AlekseiColor}
% \defTODO{MAX}{\MaxColor}


% categorical stuff

\newcommand{\catname}[1]{\mathbf{#1}}

\newcommand{\Graph}{\catname{Graph}}
\newcommand{\Set}{\catname{Set}}

\newcommand\id{\textsf{id}}
\newcommand\sym{\textsf{sym}}

% tiny inline string diagrams
\newcommand{\comonoid}{%
	\begin{tikzpicture}[baseline=0.4ex, scale=0.4, line width=0.8pt]
		\path [use as bounding box] (0,0) rectangle (0.6,0.8);
		\draw (0.2,0) -- (0.2,0.4);
		\filldraw[black] (0.2,0.4) circle (0.08);
		\draw (0,0.8) .. controls (0,0.65) and (0.1,0.45) .. (0.2,0.4);
		\draw (0.4,0.8) .. controls (0.4,0.65) and (0.3,0.45) .. (0.2,0.4);
	\end{tikzpicture}%
}
\newcommand{\counit}{%
	\begin{tikzpicture}[baseline=0.4ex, scale=0.4, line width=0.8pt]
		\path [use as bounding box] (0,0) rectangle (0.4,0.8);
		\draw (0.2,0) -- (0.2,0.4);
		\filldraw[black] (0.2,0.4) circle (0.08);
	\end{tikzpicture}%
}

% camera-ready
\newcommand{\ifcameraready}[2]{\ifdefined\cameraready #2 \else #1 \fi}

\input{../sample.tikzstyles}
\input{../hypergraph.tikzstyles}
\input{../hypergraph.tikzdefs}


\def\BibTeX{{\rm B\kern-.05em{\sc i\kern-.025em b}\kern-.08em
    T\kern-.1667em\lower.7ex\hbox{E}\kern-.125emX}}


\title{Type Theory for E-Graphs}


\begin{document}

\maketitle

\section{Introduction}

\emph{Equality saturation} is a procedure that mitigates the phase ordering problem in term rewriting --- when the application of one rewrite rule enables or disables the application of another rewrite rule --- which can drastically affect the outcome when some metric on terms is considered.
The key property of equality saturation is that rewrite rules are applied non-destructively --- thus not disabling other rewrite rules from happening --- until a fixed point (or a timeout) is reached.
This procedure is powered by a data structure called \emph{e-graph}~\cite{10.1145/1594834.1480915}, and in fact these terms are almost synonymous in the literature.
The data structure efficiently represents a congruence relation over terms by grouping them into equivalence classes utilising sharing to avoid redundant duplication of common sub-terms.

The literature on e-graphs typically goes into algorithms or applications with few works focusing on semantics aspects.
The latter is usually understood as a fixed point of some algorithm~\cite{EggPaper}, as a semantics of a Datalog program~\cite{zhang_relational_2022} or as a fixed point of some operation defined on tree automata~\cite{tree_automata_e_graphs}.
Recently, there has been a categorical account of e-graphs introduced by~\citet{tiurin2025equivalencehypergraphsdporewriting} that interpreted e-graphs as string diagrams for semilattice enriched Cartesian categories.
This interpretation allowed viewing equality saturation as a double-pushout graph rewriting procedure and provided a data structure that absorbed certain representational bureaucracies.

One can view a type theory as a formal syntactic language for, or calculus, for a category theory.
The particular type of type theory depends on the structure of the category and vice versa.
A classical example is of simply typed lambda calculus being the internal language for Cartesian closed categories~\cite{LambekLambda}.
The benefit of this view is that one can prove categorical facts by building syntactic derivations in the type theory instead of chasing commutative diagrams.

\begin{example}
We can prove, for example, that $A \otimes \mathbf{1}$ is isomorphic to $A$.
For this we need to construct two morphisms $A \otimes \mathbf{1} \to A$ and $A \to A \otimes \mathbf{1}$ and show that their composition is identity.
First, let's build a morphism from $A \to A \otimes \mathbf{1}$

\[
\inferrule*{
  x : A \vdash x : A \qquad () \vdash \star : \mathbf{1}
}{x : A \vdash \{x,\star\} : A \otimes \mathbf{1}}
\]
the morphism $A \otimes \mathbf{1} \to A$ is then
\[
\inferrule*{
\inferrule*{}{z : A \otimes \mathbf{1} \vdash z : A \otimes \mathbf{1}}
\qquad
\inferrule*{y : \mathbf{1} \vdash y : \mathbf{1} \qquad x : A \vdash x : A}{x : A, y : \mathbf{1} \vdash \text{match}_{\mathbf{1}}(y,x) : A}
}{z : A \otimes \mathbf{1} \vdash \text{match}_{A \otimes \mathbf{1}}(z, xy.\text{match}_{\mathbf{1}}(y,x)) : A}
\]
Then we have
\begin{itemize}
  \item \[
  \inferrule*{
            \inferrule*{
  w : A \vdash w : A \qquad () \vdash \star : \mathbf{1}
}{w : A \vdash \{w,\star\} : A \otimes \mathbf{1}}
\qquad
\inferrule*{
\inferrule*{}{z : A \otimes \mathbf{1} \vdash z : A \otimes \mathbf{1}}
\qquad
\inferrule*{y : \mathbf{1} \vdash y : \mathbf{1} \qquad x : A \vdash x : A}{x : A, y : \mathbf{1} \vdash \text{match}_{\mathbf{1}}(y,x) : A}
}{z : A \otimes \mathbf{1} \vdash \text{match}_{A \otimes \mathbf{1}}(z, xy.\text{match}_{\mathbf{1}}(y,x)) : A}
}{w : A \vdash \text{match}_{A \otimes \mathbf{1}}(z, xy.\text{match}_{\mathbf{1}}(y,x))[\{w, \star\}/z]}
\]

\begin{align*}
  \text{match}_{A \otimes \mathbf{1}}(z, xy.\text{match}_{\mathbf{1}}(y,x))[\{w, \star\}/z] &\equiv \text{match}_{A \otimes \mathbf{1}}(\{w, \star\}, xy.\text{match}_{\mathbf{1}}(y,x))\\
  &\equiv_{\beta_{\otimes}} \text{match}_{\mathbf{1}}(\star,w)\\
  &\equiv_{\beta_{\mathbf{1}}} w
\end{align*}

  \item The other direction is substitution 
  \[
  \inferrule*{z : A \otimes \mathbf{1} \vdash \{w, \star\}[\text{match}_{A \otimes \mathbf{1}}(z, xy.\text{match}_{\mathbf{1}}(y,x)) / w]}
{z : A \otimes \mathbf{1} \vdash \{\text{match}_{A \otimes \mathbf{1}}(z, xy.\text{match}_{\mathbf{1}}(y,x)), \star\}}
  \]

\begin{align*}
  \underbrace{\{\text{match}_{A \otimes \mathbf{1}}(\underbrace{z}_{M}, xy.\text{match}_{\mathbf{1}}(y,x)), \star\}}_{N} &\equiv_{\eta_{\otimes}} \text{match}_{A \otimes \mathbf{1}}(z, xy.\{\text{match}_{A \otimes \mathbf{1}}(\{x,y\}, x'y'.\text{match}_{\mathbf{1}}(y',x')), \star\})\\ 
  &\equiv_{\beta_{\otimes}} \text{match}_{A \otimes \mathbf{1}}(z,xy.\{{\text{match}_{\mathbf{1}}(y,x), \star}\})\\
  &\equiv_{\eta_{\mathbf{1}}}\text{match}_{A \otimes \mathbf{1}}(z, xy.\text{match}_{\mathbf{1}}(y,\{\text{match}_{\mathbf{1}}(\star,x), \star\}))\\
  &\equiv_{\beta_{\mathbf{1}}} \text{match}_{A \otimes \mathbf{1}}(z, xy.\text{match}_{\mathbf{1}}(\underbrace{y}_{M},\underbrace{\{x, \star\}}_{N[\star / u]}))\\
  &\equiv_{\beta_{\mathbf{1}}} \text{match}_{A \otimes \mathbf{1}}(z, xy.\{x, y\})\\
  &\equiv_{\eta_{\otimes}} z\\
\end{align*}
\end{itemize}
\end{example}

Just like with string diagrams, the key property of this perspective is that certain categorical axioms are absorbed by the syntax.
In this work we develop a type theoretic counterpart to the categorical framework of~\citet{tiurin2025equivalencehypergraphsdporewriting}.
This perspective allows us to formulate equality saturation as a term rewriting problem.
String diagrams can be rendered as concrete data structures such as hypergraphs with interfaces which can be thought of as generalisation of e-graphs for symmetric monoidal theories.
This hypergraph representation while being theoretically elegant --- since it absorbs the axioms of symmetric monoidal categories thus already implicitly saturating a given term with these axioms --- requires the tensor to be symmetric.
The type theory we present can be viewed as syntax for e-graphs for theories with more flexible structure, \textit{e.g.}, those with non-symmetric tensor.

The key concepts of e-graphs can be intuitively grasped from some simple examples: \autoref{fig:e-graph-example} shows how a series of rewrite rules can be non-destructively applied to an e-graph~\cite{EggPaper}.

\begin{figure*}[t!]
	\centering
  \adjustbox{width=\textwidth}{
    \tikzfig{./figures/e-graph-example-2}
  }
  \caption{E-graphs and corresponding string diagrams~\cite{tiurin2025equivalencehypergraphsdporewriting}}
  \label{fig:e-graph-example}
\end{figure*}

Each column represents a term (or term rewrite rule), its conventional e-graph, and the equivalent string diagram (for traced Cartesian categories) representation using the \emph{e-hypergraphs} used by~\citet{tiurin2025equivalencehypergraphsdporewriting}.
For the initial term, $(a*2)/2$, which is already represented efficiently in the e-graph by sharing the node $2$, the string diagram version shares this node explicitly, using a copy node.

E-graphs consist of two types of nodes: \emph{e-classes}, representing equivalence classes of terms and denoted by dashed boxes, containing \emph{e-nodes}, representing equivalent terms that share the same root.
E-nodes reference e-classes instead of other e-nodes, in contrast to a traditional term graph.
By making a sequence of choices for e-nodes in e-classes, we can extract terms from the graph; intuitively, a different sequence of choices results in an equivalent term.
A term graph is embedded by an e-graph where each e-class has a single e-node.

The first rewrite rule, replacing multiplication by \emph{shift-left} creates the first non-singleton e-class, which includes the multiplication ($*$) and shift-left ($<\!\!<$) nodes, now sharing the node $a$.
This second e-graph illustrates the other representational difference\footnote{Despite the \enquote{difference}, these representations are isomorphic in the sense that there is a non-ambiguous translation from one representation to the other and vice versa.}, besides sharing, between conventional e-graphs and e-hypergraphs.
In the former, edges connect directly to e-nodes inside the e-class, whereas in the latter edges connect to the equivalence class itself.
Explicit deletion nodes are depicted with black dots with a single incoming wire
\scalebox{0.4}{
	\begin{tikzpicture}
		\begin{pgfonlayer}{nodelayer}
			\node [style=vertex] (0) at (1.5, 3.5) {};
			\node [style=none] (1) at (1.5, 3) {};
		\end{pgfonlayer}
		\begin{pgfonlayer}{edgelayer}
			\draw (0) to (1.center);
		\end{pgfonlayer}
	\end{tikzpicture}
}
inside the dashed boxes, indicating which of the equivalence-class-level edges are connected to which nodes inside the equivalence class.
Notably, the step $(d) \to (e)$ introduces a cycle in the e-graph as it merges the e-classes containing $*$ and $a$ and matching the left-hand side of the rule is with the enode $a * 1$.
This cycle allows the e-graph to represent infinitely many equivalent terms: $a$, $a * 1$, $((a * 2) / 2) * 1$, and so on.
String diagrammatically this cycle is modelled using a trace --- picking the right-hand side component of the topmost box results in the following term:
\[
\adjustbox{scale=0.6}{
\tikzfig{./figures/e-to-d-example-1}
}
\]

Picking $*$ and then $a$ results in the following diagram:

\[
\adjustbox{width=\textwidth}{
\tikzfig{./figures/e-to-d-example-2}
}
\]

Using the type theory, we can associate a term for each of these e-graph or string diagrams.
These are presented in.
The paper is about formalising this type theory.

% \begin{figure*}
% \begin{minipage}{0.45\textwidth}
%   \begin{minted}{haskell}
%     let x = 2 in
%        (a * x) / x
%   \end{minted}
%   \subcaption{(a)}
% \end{minipage}
% \hfill
% \begin{minipage}{0.45\textwidth}
%   \begin{minted}{haskell}
%     let x = 2
%     let e arg_1 arg_2 arg_3 = 
%      arg_1 * arg_2 + arg_3 << 1 in
%         (e a x 0) / x
%   \end{minted}
%   \caption{(b)}
% \end{minipage}
% \hfill
% \vspace{1em}
% \begin{minipage}{0.45\textwidth}
%   \begin{minted}{haskell}
%     let x = 2
%     let y = a
%     let e arg_1 arg_2 arg_3 = 
%      arg_1 * arg_2 + arg_3 << 1
%     let z arg_1 arg_2 arg_3 arg_4 = 
%      arg_1 / arg_2 + arg_3 * arg_4 in
%     z (e y x y) x y (x / x)
%   \end{minted}
%   \caption{(c)}
% \end{minipage}
% \hfill
% \begin{minipage}{0.45\textwidth}
%   \begin{minted}{haskell}
%     let x = 2
%     let y = a
%     let e arg_1 arg_2 arg_3 = 
%      arg_1 * arg_2 + arg_3 << 1
%     let z arg_1 arg_2 arg_3 arg_4 = 
%      arg_1 / arg_2 + arg_3 * arg_4
%     let w arg_1 arg_2 = arg_1 / arg_2 + 1 in
%     z (e y x y) x y (w x x)
%   \end{minted}
%   \caption{(d)}
% \end{minipage}
% \hfill
% \vspace{1em}
% \begin{minipage}{0.45\textwidth}
%   \begin{minted}{haskell}
%     let x = 2
%     let e arg_1 arg_2 arg_3 = 
%      arg_1 * arg_2 + arg_3 << 1
%     let z arg_1 arg_2 arg_4 = 
%      arg_1 / arg_2 + (z arg_1 arg_2 arg_4) * arg_4 + a
%     let w arg_1 arg_2 = arg_1 / arg_2 + 1 in
%     let y = z (e y x y) x (w x x) in y
%   \end{minted}
%   \caption{(e)}
% \end{minipage}
% \label{fig:terms-e-graph}
% \caption{Terms corresponding to e-graphs in \autoref{fig:e-graph-example-2}}
% \end{figure*}

\begin{listing*}[t]
\captionsetup[sub]{labelformat=parens}
\centering
% If needed (usually not in acmart):
% \captionsetup{type=listing}

\begin{minipage}[t]{0.45\textwidth}
  \begin{minted}{haskell}
let x = 2 in
   (a * x) / x
  \end{minted}
  \subcaption{}
\end{minipage}
\hfill
\begin{minipage}[t]{0.45\textwidth}
  \begin{minted}{haskell}
let x = 2
let e arg_1 arg_2 arg_3 =
  arg_1 * arg_2 + arg_3 << 1 in
  (e a x 0) / x
  \end{minted}
  \subcaption{}
\end{minipage}

\vspace{1em}

\begin{minipage}[t]{0.45\textwidth}
  \begin{minted}{haskell}
let x = 2
let y = a
let e arg_1 arg_2 arg_3 =
  arg_1 * arg_2 + arg_3 << 1
let z arg_1 arg_2 arg_3 arg_4 =
  arg_1 / arg_2 + arg_3 * arg_4 in
z (e y x y) x y (x / x)
  \end{minted}
  \subcaption{}
\end{minipage}
\hfill
\begin{minipage}[t]{0.45\textwidth}
  \begin{minted}{haskell}
let x = 2
let y = a
let e arg_1 arg_2 arg_3 =
  arg_1 * arg_2 + arg_3 << 1
let z arg_1 arg_2 arg_3 arg_4 =
  arg_1 / arg_2 + arg_3 * arg_4
let w arg_1 arg_2 = arg_1 / arg_2 + 1 in
z (e y x y) x y (w x x)
  \end{minted}
  \subcaption{}
\end{minipage}

\vspace{1em}

\begin{minipage}[t]{0.45\textwidth}
  \begin{minted}{haskell}
let x = 2
let e arg_1 arg_2 arg_3 =
  arg_1 * arg_2 + arg_3 << 1
let z arg_1 arg_2 arg_4 =
  arg_1 / arg_2 + (z arg_1 arg_2 arg_4) * arg_4 + a
let w arg_1 arg_2 = arg_1 / arg_2 + 1 in
let y = z (e y x y) x (w x x) in y
  \end{minted}
  \subcaption{}
\end{minipage}

\caption{Terms corresponding to e-graphs in \autoref{fig:e-graph-example}}
\label{fig:terms-e-graph}
\end{listing*}


\bibliographystyle{ACM-Reference-Format}
\bibliography{../bibliography}

\end{document}