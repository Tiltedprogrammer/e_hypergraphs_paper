%!LW recipe=pdflatex
\documentclass{article}

\usepackage{tikz-cd}
\usepackage{xargs}
\usepackage{amsmath}
\usepackage{stmaryrd}
\usepackage{amssymb}

% !TEX root = ../main.tex

\usepackage{docmute}
\usepackage{hyperref}
\usepackage{proof}

\usepackage{mathtools}
\usepackage{tikz}
\usepackage{tikz-network}
\usepackage{xspace}
\usepackage{xargs}
\usepackage{fontenc}

%\usepackage{minted}

\usepackage{subcaption}
\usepackage{caption}

\usepackage{tikz-cd}
\usepackage{tikzit}
% to reduce tikz picture white space
\usepackage{adjustbox}

% for quotienting
\usepackage{xfrac} 

\usepackage{import}

% category names
%\newcommand{\MdaCospans}{{\catname{MACsp_{D}(Hyp_{\Sigma})}}}

\newcommand\HypI[1]{\textbf{HypI}(#1)}
\newcommand\MdaCospans{\textbf{MHypI}(\Sigma)}
\newcommand{\Ecospans}{{\catname{EHypI(\Sigma)}}}
\newcommand{\MdaEcospans}{{\catname{MEHypI(\Sigma)}}}
\newcommand{\WellTypedMdaEcospans}{{\catname{MEHypI(\Sigma)}}}
% conflict
\newcommand{\hashtag}{{\#}}
\newcommand{\consistency}{{\smile}}
% There is already a remark in the preamble, but it does
% not work for some reason
\newtheorem{remark}{Remark}


\usepackage{xifthen}
\ifthenelse{\isundefined{\ismain}}{
  \newcommand{\ismain}{0}
}{}

\usepackage{enumitem}
 
\newcommand{\ignora}[1]{ }

\ignora{
\usepackage{aliascnt}
% alias-counters
\newaliascnt{definition}{thm}
\newaliascnt{proposition}{thm} 
\newaliascnt{lemma}{thm}
\newaliascnt{corollary}{thm}
\newaliascnt{conjecture}{thm}
\newaliascnt{remark}{thm}
\newaliascnt{example}{thm}
\newaliascnt{examples}{thm}
\newaliascnt{assumption}{thm}
\newaliascnt{construction}{thm}
\newaliascnt{claim}{thm}

% \theoremstyle{plain}
\theoremstyle{definition}

\newtheorem{theorem}[thm]{Theorem}
\newtheorem{proposition}[proposition]{Proposition} 
\aliascntresetthe{proposition}
\newtheorem{lemma}[lemma]{Lemma}
\aliascntresetthe{lemma}
\newtheorem{corollary}[corollary]{Corollary}
\aliascntresetthe{corollary}
\newtheorem{conjecture}[conjecture]{Conjecture}
\aliascntresetthe{conjecture}
\newtheorem{remark}[remark]{Remark}
\aliascntresetthe{remark}
\newtheorem{assumption}[assumption]{Assumption}
\aliascntresetthe{assumption}
\newtheorem{construction}[construction]{Construction}
\aliascntresetthe{construction}
\newtheorem{claim}[claim]{Claim}
\aliascntresetthe{claim}

\theoremstyle{definition}
\newtheorem{definition}[definition]{Definition}
\aliascntresetthe{definition}
\newtheorem{example}[example]{Example}
\aliascntresetthe{example}
\newtheorem{examples}[examples]{Examples}
\aliascntresetthe{examples}
}

\usepackage{cleveref} % after hyperref!
% possibly new thm environment
% \newtheorem{satz}[thm]{Satz}


% \newcommand{\TODO}[1]{{{\color{red}[\textbf{TODO:} #1]}}}
% \newcommand{\TODO}[1]{}

% !TEX root = ../main.tex

\newcommand{\missing}[1]{{\color{red}\bfseries [TODO]}}

\newcommand{\egg}{\texorpdfstring{\MakeLowercase{\texttt{egg}}}{\texttt{egg}}\xspace}
\newcommand{\Egg}{\texorpdfstring{\MakeLowercase{\texttt{egg}}}{\texttt{egg}}\xspace}
% \newcommand{\egg}{Lego\xspace}
% \newcommand{\Egg}{Lego\xspace}
\newcommand{\egraphs}{\mbox{e-graphs}\xspace}
\newcommand{\egraph}{\mbox{e-graph}\xspace}
\newcommand{\Egraph}{\mbox{E-graph}\xspace}
\newcommand{\Egraphs}{\mbox{E-graphs}\xspace}
\newcommand{\eclass}{\mbox{e-class}\xspace}
\newcommand{\Eclass}{\mbox{E-class}\xspace}
\newcommand{\enode}{\mbox{e-node}\xspace}
\newcommand{\eclasses}{\mbox{e-classes}\xspace}
\newcommand{\enodes}{\mbox{e-nodes}\xspace}
\newcommand{\Enodes}{\mbox{E-nodes}\xspace}
% \newcommand{\regraph}{Regraph\xspace}
% \newcommand{\Regraph}{Regraph\xspace}
\newcommand{\sz}{Szalinski\xspace}
\newcommand{\find}{\texttt{find}\xspace}

\newcommand{\equivid}{\equiv_{\sf id}}
\newcommand{\equivnode}{\equiv_{\sf node}}
\newcommand{\equivterm}{\equiv_{\sf term}}

\newcommand{\congrinv}{$\mathcal{I}_c$\xspace}

\newcommand{\egglogo}[1][]{\protect\includegraphics[height=1em, #1]{egg.png} }
\newcommand{\eggurl}{\url{https://github.com/mwillsey/egg}}

% TODOs
% https://tex.stackexchange.com/questions/9796/how-to-add-todo-notes

\newcommand{\eggtodo}[1]{\textcolor{Red}{\textbf{[CITE: #1]}}}

\newcommand{\defTodo}[2]{%
  \expandafter\newcommand\csname #1\endcsname[1]{%
    \todo[linecolor=#2,backgroundcolor=#2!25,bordercolor=#2,inline,size=\tiny]{\textbf{#1}: ##1}}}

\newcommand{\defTODO}[2]{%
  \expandafter\newcommand\csname #1\endcsname[1]{%
    \todo[linecolor=#2,backgroundcolor=#2!25,bordercolor=#2,inline,size=\tiny,caption={\textbf{(#1 LONG TODO)}}]{##1}}}

\newcommand{\AlekseiColor}{RoyalBlue}

% examples
% \newcommand{\RemyColor}{Red}
% \newcommand{\OliverColor}{OliveGreen}
% \newcommand{\ChandraColor}{Magenta}
% \newcommand{\PavelColor}{Cerulean}
% \newcommand{\ZachColor}{Plum}
% \newcommand{\BenColor}{Orchid}
% \newcommand{\JamesColor}{Salmon}


\defTodo{Aleksei}{\AlekseiColor}
% \defTodo{Max}{\MaxColor}

\defTODO{ALEKSEI}{\AlekseiColor}
% \defTODO{MAX}{\MaxColor}


% categorical stuff

\newcommand{\catname}[1]{\mathbf{#1}}

\newcommand{\Graph}{\catname{Graph}}
\newcommand{\Set}{\catname{Set}}

\newcommand\id{\textsf{id}}
\newcommand\sym{\textsf{sym}}

% tiny inline string diagrams
\newcommand{\comonoid}{%
	\begin{tikzpicture}[baseline=0.4ex, scale=0.4, line width=0.8pt]
		\path [use as bounding box] (0,0) rectangle (0.6,0.8);
		\draw (0.2,0) -- (0.2,0.4);
		\filldraw[black] (0.2,0.4) circle (0.08);
		\draw (0,0.8) .. controls (0,0.65) and (0.1,0.45) .. (0.2,0.4);
		\draw (0.4,0.8) .. controls (0.4,0.65) and (0.3,0.45) .. (0.2,0.4);
	\end{tikzpicture}%
}
\newcommand{\counit}{%
	\begin{tikzpicture}[baseline=0.4ex, scale=0.4, line width=0.8pt]
		\path [use as bounding box] (0,0) rectangle (0.4,0.8);
		\draw (0.2,0) -- (0.2,0.4);
		\filldraw[black] (0.2,0.4) circle (0.08);
	\end{tikzpicture}%
}

% camera-ready
\newcommand{\ifcameraready}[2]{\ifdefined\cameraready #2 \else #1 \fi}

\input{../sample.tikzstyles}
\input{../hypergraph.tikzstyles}
\input{../hypergraph.tikzdefs}

\newcommand\id{\textsf{id}}
\newcommand\sym{\textsf{sym}}


\begin{document}

\title{LICS24 Reviews Response}
\author{}

\maketitle

\section*{Review 1}
\paragraph{About theorem 6.5 (the central result of the paper: full completeness)}
\textit{I am doubtful about the statement because it seems to me that $[sym] = [id]$,
hence the functor $[-]$ is not faithful.
Here is a sketch of the argument: $[\sym]$ is a e-hypergraph with two vertices
where the inputs and the outputs are inverted. Now we embed it into a
hierarchical edge by applying the bottom left equality of Figure 3.
We get a cospan with 2 external inputs, 2 external ouputs, 2 strict internal
inputs, 2 strict internal outputs (the sets of strict internal inputs and
outputs are the same, but they are labelled differently by $f_{int}$ and $g_{int}$). We
do the same thing with $[\id]$ and notice that the two resulting cospans are
isomorphic. Thus they are equal in the category where cospans are quotiented by
isos.}

We assume that in $[\sym] = [\id]$ it was meant that $[\sym] = [\id \otimes \id]$, otherwise the equality does not typecheck.

By the definition $[\sym]$ is 
\[
    n \xrightarrow{f_{ext}} n' \xrightarrow{f_{int}} \mathcal{G} \xleftarrow{g_{int}} m' \xleftarrow{g_{ext}} m
\]
represented as
\begin{figure}[h!]
    \[
    \scalebox{0.5}{
        \tikzfig{../figures/reviews/sym}
    }
    \]
\end{figure}

And $[\id \otimes \id]$ is \[
    n \xrightarrow{f'_{ext}} n'' \xrightarrow{f'_{int}} \mathcal{G} \xleftarrow{g'_{int}} m'' \xleftarrow{g'_{ext}} m
\]
represented as 
\begin{figure}[h!]
    \[
    \scalebox{0.5}{
        \tikzfig{../figures/reviews/id}
    }
    \]
\end{figure}

Clearly, it should be the case that $\{u_1 \mapsto u_1, \ldots u_4 \mapsto u_4\}$ and similarly for $w_i$. I think this does not follow from our definition of isomorphic cospans, so perhaps we should've required that $\alpha$ and $\gamma$ are identities (below).
However, the interfaces are ordered according to a remark on page 21 in [3], so the only possible isomorphism between $n'$ and $n''$ (respectively, $m'$ and $m''$) is identity.
\[
    \scalebox{0.75}{
        \tikzfig{../figures/combinatorial_semantics/isomorphic_e_cospans}
    }
\]
Then, the cospans above are not isomorphic. Consider the cases.
\begin{enumerate}
    \item ${v_1 \mapsto v_1}$ and ${v_2 \mapsto v_2}$ under $\beta$. Then $g_{int};\beta(w_4) \not = g'_{int}(w_4)$
    \item ${v_1 \mapsto v_2}$ and ${v_2 \mapsto v_1}$ under $\beta$. Then $f_{int};\beta(w_4) \not = f'_{int}(w_4)$
\end{enumerate}
This means that the corresponding diagram does not commute and therefore the cospans are not isomorphic.

\paragraph{In the proof}
In the first case, it is immediate that $[f] \not = [g]$, since equal terms
have weak normal forms with the same number of components.
\textit{This argument does not seem to be valid in presence of equations.
Consider an equation $a = a + b$ for example, where a and b are constants of the
signature. Then}
\[
    [a] = [a + b] = [a] + [b]
\]

We only consider equations of SMC terms, \textit{i.e.}, with no $+$ on either side of $=$.

\paragraph{About Proposition A.14 (existence of the pushout complement)}
\textit{Why doesn't the pushout complement trivially exist since you already assume
given a pushout square?}

It is indeed the case.
The formulation should've been consider a morhisms $\mathcal{K} \to \mathcal{L} \to \mathcal{G}$.

\paragraph{In the proof:}
Then, no-dangling and no-identification
conditions are necessary for pushout complement to exist [26].
\textit{Are those conditions sufficient? Can you give a more precise reference? It seems
to me that [26] (in particular Proposition 9 about the existence of pushout
complements) focuses on the category of graphs only, so I don't see how it
applies to the category of e-hypergraphs.}

I do not know if those conditions are sufficient. 
I am only sure the three conditions of A.14 are sufficient.
We only proved that the pushout exists for a certain class of morphisms hence the additional conditions.
We did not, however, prove that the pushout does not exist for other classes of morphisms.
The no-dangling condition ensures that the sources or targets of the edge which is not a part of a match do not get deleted.
If it was the case then there would be no way to obtain the original graph by computing a pushout as there would be not enough vertices.
Similarly with no-identification condition, if $m$ was to identify two vertices of $\mathcal{L}$ while these were not identified by $f$ then there would be no way of getting $\mathcal{G}$ by only knowing $\mathcal{L}$, $\mathcal{L^{\bot}}$ and $\mathcal{K}$ as computing the union of $\mathcal{L}$ and $\mathcal{L^{\bot}}$ would produce more vertices than there was in $\mathcal{G}$.
The proposition ensures that after we remove the image of $\mathcal{L}$ from $\mathcal{G}$ (excluding the image of $f;m$), all the morphisms in the pushout square are suitable for our computation of the pushout.
There is a claim about the necessity and sufficiency of no-identification and no-dangling conditions for the pushout to exist in $\mathbf{Hyp}(\Sigma)$ in Proposition 3.17 of [3].

\paragraph{About Proposition A.15 (uniqueness of the boundary complement)}
The pushout in $\mathbf{EHyp}(\Sigma)$ should also be a pushout on the
underlying sets of vertices and hyperedges.
\textit{Why is that? Is it always the case for any pushout in $\mathbf{EHyp}(\Sigma)$?}

This is by construction as we compute the disjoint union of vertices and edges and identify them along the corresponding morphisms.
This construction yields the pushouts on the sets of vertices and edges.
The construction of the corresponding consistency and child relations are orthogonal as these do not change the sets of edges and vertices.
This is at least the case for a particular set of morphisms that we consider.

\textit{Then, since all the arrows are monos}
\textit{What if some vertex is both input and output in $\mathcal{L}$? Then $i' + j' \to L$ is not a
monomorphism, is it?}

Yes, then $i' + j' \to L$ is not a monomorphism.

\paragraph{About remark 2 (restriction to signatures without any symbol of arity $0 \to 0$)}
\textit{By incident vertices, do you mean input and output vertices? I think there is
still a problem if you consider an hypergraph $L$ composing an edge of arity $0 \to
1$ with an edge of arity $1 \to 0$: the boundary conditions prevent finding a match
of $L$ in any $G$.}

Yes, this is correct.

\paragraph{About Def 4.1 (e-hypergraphs)}
\textit{Why don't you require the condition (3) for t(e) as well?}

That's a typo. It should've been similar to consistency relation.

\textit{The consistency equivalence relation must additionally satisfy that if $v \in s(e)$ or $v \in t(e)$ then $e \consistency v$}
\textit{What if 'e' is a top-level edge?}

Should've been if $e$ is not a top-level edge.

\paragraph{About def 4.2 (e-hypergraph homomorphism)}

\textit{Condition (2): what if the image of 'v' or 'e' is top-level?}

Then we do not require such an equality, \textit{i.e.}, it might hold or might not meaning that a top-level vertex of edge might end up either top-level or nested under a homomorphism but nested edges and vertices should preserve their predecessors.

\paragraph{About def 4.5 (well-typed MDA hypergraphs)}
\textit{Why aren't non top-level hierarchical edges required to be well-typed?}

All hierarchical edges must be well-typed. The `apex' refers to the apex of a cospan, \textit{aka}, the carrier.

\paragraph{About def 5.2 (extended boundary complement)}
\textit{Isn't requiring that $i+j \to L^\bot$ is a mono too strong? For example if you
want to account for an equality relating some term involving the identity
morphism (for which all nodes are both inputs and outputs).}

Yes, it is strong and was needed to make the proofs easier. 
Dropping this condition is a future work.

\textit{Are monomorphisms in the category of hypergraphs the morphisms that are
injective on edges and vertices?}

Yes.

\paragraph{Minor remarks / typos}

\textit{In the figure above, shouldn't there be $n$ (rather than just 1) pairs of black
dots in each dashed subbox? (p2)}

There is a dot for each unused input of the corresponding instance within a box.
For example, $*$ requires $a$ and $2$ and therefore the wires from another copy of $2$ and from $1$ are terminated with dots.

\textit{removing from this occurrence everything that has no pre-image in $\mathcal{K}$ (p6)}
\textit{Isn't $C$ rather constructed by removing from $G$ everything from $L$ except for what
comes from $K$?}

Yes, this formulation is equivalent. We are removing $m(L)$ from $G$ with the exception of nodes and edges that have a pre-image in $K$, \textit{i.e.}, those that come from $K$.
\textit{What is a monogamous cospan? p(7)}

Apparently the definition of a monogamous cospan was overlooked.

\textit{This is confusing: you are not defining any equivalence relation here. Rather, 
this family of equivalence relations is provided by each hypergraph (if I
understand correctly). Def 4.1 (p7)}

Yes, it is indeed a family of relations.

\textit{Shouldn't a homomorphism preserves labelling?}

It preserves labelling as per the condition (1) on such a morphism: it should be a hypergraph-homomorphism, which, by definition, preserves labelling.

\textit{Unfolding definitions, it follows that}
$f = g \implies \llbracket f \rrbracket \Rrightarrow^{*} \llbracket g \rrbracket$.

\textit{The literal meaning makes this statement trivial: if $f = g$, then $[f]$ rewrites to
$[g]$ in 0 steps.
If we replace $f = g$ with $p_i(f) = p_i(g)$, where $f$ and $g$ are unquotiented terms and
$p_i$ is the canonical projection, this is wrong, because it would entail that the
rewriting relation is symmetric, which is not the case, unless we consider} $\Rrightarrow^{*}$
\textit{that each equation generates two rewrite rules, one for each direction of the
equality (if is the case, this should be said explicitly). Corollary 6.6 suffers
from the same defect, and the proof of Theorem 6.5 as well.}

$\llbracket \mathcal{E} \rrbracket$ consists of $\langle l, r \rangle$ and $\langle l, r \rangle$ for each $l = r$ in $\mathcal{E}$.

\textit{Aren't they partial functions, defined on any non-top level elements only?}
Yes

\textit{Condition (5) implicitly assumes that f(v) is not top-level, because the
consistency relation is only defined for such elements.}

It should be read as either all $f(v_i)$ are pair-wise consistent, or consistency for them is undefined, \textit{i.e.}, all of them are top-level.

\textit{In which category is this pushout proven to exist? Clearly not in the category
$\mathbf{EHyp}(\Sigma)$ of hypergraphs, since just after the proof you write that it was not
proven to be a hypergraph.}

It becomes a pushout in $\mathbf{EHyp}(\Sigma)$ right after it is proved that it is a valid e-hypergraph (Proposition A.12).
Probably the proof should've been ordered the other way around.

\textit{In the proof, you consider the identity relation on edges, then its
reflexive-transitive-symmetric closure. Doesn't it coincide with the identity
relation? Furthermore, doesn't quotienting a set X by the identity relation
yields the same set X?
Do you really need to take the transitive closure of the relation on vertices,
since f and g are monomorphisms?}

This was just to make the presentation uniform.

\section*{Review 2}
\textit{The discussion about Figure 1 in Section (1.1) does not provide enough details to understand what the graph describes. In particular, in the e-hypergraph formalism, I am not sure I understand what the black nodes inside the equivalence classes mean. For instance, why are these nodes only connected to the bottom of the boxes and never to the top?}

Black dots correspond to delete morphisms of the Cartesian category, \textit{i.e.}, they discard the unused input and hence are only conected to the bottom as inputs flow from bottom to top.

\textit{Is the free monoid monad the list monad?}

Yes

\textit{Without further assumptions (e.g., some form of adhesivity), pushout complements may not be unique, assuming it exists. Is there any guarantee in the DPOI framework for hypergraphs with interfaces? How does the construction of boundary complement (Definition 3.8) relate to initial pushouts (see, for instance, Chapter 6 of Fundamentals of Algebraic Graph Transformation Ehrig et a. 2006, page 125)?}

The category of plain hypergraphs ($\mathbf{Hyp}$), as used in [3], is adhesive.
We do not know if the category of e-hypergraphs ($\mathbf{EHyp}$) has any adhesive properties, however, in this work we are only interested in pushouts along monos and monomorphic matches and for such morphisms the pushout complement is unique as shown in A.15.
The boundary complement definition is a generalisation of a similar concept from [4] which is needed to make sure we remain within needed class of morphisms when computing the complement and the pushout.
We can not immediately relate initial pushouts to the boundary complement.

\textit{One could argue that since the consistency relation is built from the child relation, it should not be part of the tuple defining the e-hypergraph.}

?

\textit{In Figure 5, why do $a_1$, ... $a_n$, and $a'_1$, ..., $a'_k$ appear as internal interfaces on the rightmost hypergraph?}

Internal interface is external interface plus strict internal interface. Since both $a_i$'s and $a'_j$'s are external interfaces they should be present in internal interface as well.
The figure is essentially saying that we sum both internal interfaces and add extra copy of external interface (to account for the inputs and outputs of this new hierarchical edge).


\section*{Review 3}
\textit{However, I think the authors could improve the discussion of the 0-ary case and be more convincing about the fact that this could be handled without much difficulty. Otherwise, this would dramatically restrict the domain of application of those results.}

One workaround is to add more context to the left-hand side of a rewrite rule. 
Assume we have a rewrite rule $f \Rrightarrow g \otimes a$ where $a$ is a scalar.
If the occurrence of $f$ within $\mathcal{G}$ is top-level, then we can proceed as usual.
If, on the other hand, the occurrence is nested, then there is no way to specify that $a$ should have the same parent as $f$ had.
The workaround can be seen in the following example.

\[
    \scalebox{0.5}{
        \tikzfig{../figures/reviews/handling_scalars}
    }
\]

$\mathcal{F}, \mathcal{G}_i$ are variable-labelled e-hypergraphs that can be unified with concrete e-hypergraphs.
For example, $\mathcal{F}$ corresponds to $h$ and $\mathcal{G}_1$ corresponds to $k$.

\section*{Review 4}

Review 4 had no questions.

\end{document}
