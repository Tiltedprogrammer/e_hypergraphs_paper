\section{Conclusion}
\label{sec:conclusion}

We have seen how e-graphs over algebraic theories can be expressed in terms of Cartesian categories enriched over the category of semilattices. 
In fact, e-graphs are an instance of a more general construction that can account for monoidal theories in terms of SMCs,  similarly enriched.  
Having described a translation of e-graphs into this categorical framework, we also provide a combinatorial representation of morphisms in the free category.
As expected, this representation factors out the structural SMC equations, while remaining sensitive to the complexity-relevant equations induced by the enrichment.  
We then showed our theory to be sound and complete with respect to our categorical semantics.

In the future,  we intend to investigate a number of natural extensions to mathematical treatment of (acyclic) e-graphs developed here.
Some of the extensions that can be considered are: support for `functorial boxes'~\cite{mellies_functorial_2006} (in addition to e-boxes), which have a variety of applications, including the Cartesian-closure required for the implementation of functional programming languages~\cite{ghica-zanassi2023string};
% , as sketched in Fig. \ref{fig:app1}(a)
the `spiders' used by the ZX-calculus to model quantum circuits~\cite{coecke_interacting_2011,ZX};
% , as sketched in Fig. \ref{fig:app1}(b)
or trace~\cite{joyal_geometry_1991, Hasegawa-traced} which can be used to express feedback in categorical models of digital circuits~\cite{ghica_jung_2017,ghica_compositional_2023} and, indeed in conventional e-graphs to encode infinite equivalence classes.
% , as sketched in Fig.~\ref{fig:app2}. 
All these application domains stand to benefit from the use of optimisation techniques, which in turn can take advantage of the appropriate e-graph technique.
In each case above,  we aim to combine our combinatorial representation with existing combinatorial representations of the relevant structure as the hypergraph representation and DPO-rewriting theory of each case is already well-studied independently of any e-graph structure \cite{ghica_rewriting_2023,alvarez-picallo-functorial_2021}.
 % traced comonoid

% \begin{figure}
% \begin{subfigure}{0.45\linewidth}
% \[
% 	\scalebox{0.75}{
% 	\tikzfig{../figures/conclusion/example-e-graphs_1}
% 	}
% \]
% \subcaption{\;} 
% \end{subfigure}
% \hfill
% \begin{subfigure}{0.45\linewidth}
% \[
% 	\scalebox{0.75}{
% 	\tikzfig{../figures/conclusion/example-e-graphs_2}
% 	}
% \]
% \subcaption{\;}
% \end{subfigure}
% \caption{E-(hyper)graphs for functorial boxes and the ZX-calculus}
% \label{fig:app1}
% \end{figure}

% \begin{figure}
% 	\[
% 	\scalebox{0.65}{
% 	\tikzfig{../figures/conclusion/example-traced}
% 	}
% \]
% 	\caption{E-graphs with cycles and e-hypergraphs with trace}
% 	\label{fig:app2}
% \end{figure}

% In each case above,  we aim to combine our combinatorial representation with existing combinatorial representations of the relevant structure.  Indeed,  the hypergraph representation and DPO-rewriting theory of each case is already well-studied independently of any e-graph structure \cite{ghica_rewriting_2023,alvarez-picallo-functorial_2021}. % traced comonoid
%The hierarchical ``e-box'' structure within string diagrams and their associated e-hypergraphs (categorically modelled by semilattice enrichment) is essentially orthogonal to these developments. Combining the two, to give notions of e-hypergraphs specifically suited to these particular monoidal theories should be conceptually straightforward.  However, as we have seen in this paper, the technical details can be delicate. 
% Finally,  a term calculus (e-terms) can be reverse engineered from the string-diagram notation~\cite{ghica_structural_nominal, Heijltjes:FMCII}. Although term calculi are less important in the algorithmics of optimisation, they still play a role in specification and in aiding readability.  
