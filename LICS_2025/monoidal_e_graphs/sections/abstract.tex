% The technique of equipping graphs with an equivalence relation, called equality saturation, has recently proved both powerful and practical in program optimisation, particularly for satisfiability modulo theory solvers. 
The technique of \emph{equality saturation}, which equips graphs with an equivalence relation, has proven effective for program optimisation.
We give a categorical semantics to these structures, called \emph{e-graphs}, in terms of Cartesian categories enriched over the category of semilattices.
This approach generalises to monoidal categories, which opens the door to new applications of e-graph techniques, from algebraic to monoidal theories.
Finally, we present a sound and complete combinatorial representation of morphisms in such a category,  based on a generalisation of hypergraphs which we call \emph{e-hypergraphs}.
They have the usual advantage that many of their structural equations are absorbed into a general notion of isomorphism.
This new principled approach to e-graphs enables double-pushout (DPO) rewriting for these structures, which constitutes the main contribution of this paper.
