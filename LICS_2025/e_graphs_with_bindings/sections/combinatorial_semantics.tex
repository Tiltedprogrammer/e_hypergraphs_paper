\section{Combinatorial semantics}

\textcolor{red}{Add intro to hypergraphs}

We can extend the definition of e-hypergraphs for $\catname{SLat}$-enriched symmetric monoidal categories~\cite{ghica2024equivalencehypergraphsegraphsmonoidal} to accommodate new structure brought by closedness.


\begin{definition}
We define an e-hypergraph $\mathcal{G}$ over a closed symmetric monoidal signature $\Sigma = (\Sigma_{O}, \Sigma_{M})$ as a tuple $(V,E,s,t, l_{V}, l_{E} \textcolor{red}{<}, \textcolor{blue}{<},\consistency)$ where
\begin{itemize}
  \item $V$ is a set of vertices.
  \item $E = \textcolor{gray}{E} \cup \textcolor{red}{E} \cup \textcolor{blue}{E}$ is a set of hyperedges that that is formed of three disjoint sets of hyperedges defined below.
  \item $s,t : E \to V^{*}$ are source and target functions.
  \item $l_{E} : E \to \Sigma_{M} + 1$, $l_{V} : V \to \Sigma_{O} \cup \text{obj}_{\multimap}$ are \textit{label} functions, where $\text{obj}_{\multimap}$ consists of objects $A \multimap B$ for $A,B \in obj_{\Sigma_{O}}$.
  \item $\textcolor{red}{<} \subseteq \textcolor{red}{E} \times V + E$ and $\textcolor{blue}{<} \subseteq \textcolor{blue}{E} \times V + E $ with $\textcolor{red}{E}, \textcolor{blue} \subset E$ are disjoint partial orders.
        For $x \in V + E$, the set of predecessors of $x$ is denoted as $[x) = \{x' ~|~ \exists y_{1} = x', \ldots, y_{n} = x,  y_{i} \textcolor{red}{<^{\mu}} y_{i + 1} \text{ or } y_{i} \textcolor{blue}{<^{\mu}} y_{i+1}\; \}$.
        We will denote the immediate predecessor $y$ of $x$ as $y \textcolor{red}{<^{\mu}} x$ and $y \textcolor{blue}{<^{\mu}} x$.
        We will call edges $e$ such that $l(e) = \bot$ hierarchical and edges $e$ and vertices $v$ such that $[e) = \varnothing$ and $[v) = \varnothing$ \textit{top-level}.
        Each of these relations should satisfy the following
        \begin{enumerate}
          \item each set of predecessors consists exclusively of hierarchical edges;
          \item each $x$ has at most one immediate predecessor;
          \item edges $e$ such that there is no $x$ such that $e \textcolor{red}{<^{\mu}} x$ or $e \textcolor{red}{<^{\mu}} x$ are labelled;
          \item the relations are closed under connectivity, \emph{i.e.}, if $v \in s(e)$ then $e' \textcolor{red}{<^{\mu}} e$ iff $e' \textcolor{red}{<^{\mu}} v$, similarly for $v \in t(e)$ and $\textcolor{blue}{<}$.
        \end{enumerate}
  \item $\consistency$ is a \textit{consistency relation} which is given by the union of a family of equivalence relations $\consistency_p$ on each set $\{x \in V + E ~|~ p \textcolor{red}{<^\mu} x\}$ of elements which share the same parent where each relation is also closed under connectivity, \textit{i.e.}, if $v \in s(e)$ or $v \in t(e)$ such that $p \textcolor{red}{<^{\mu}}(v)$ and $p \textcolor{red}{<^{\mu}}(e)$ then $v \consistency_{p} e$.
  We require that $\consistency_{p} \not = (E_{p} + V_{p}) \times (E_{p} + V_{p})$ where $V_{p} = \{ v ~ | ~ p \textcolor{red}{<^{\mu}} v\}$ and $E_{p} = \{ e ~ | ~ p \textcolor{red}{<^{\mu}} e\}$.
\end{itemize} 
$\textcolor{red}{E}$ and $\textcolor{blue}{E}$ are then induced by the corresponding relations can consist of exclusively hierarchical edges, then we have $\textcolor{gray}{E} = E \setminus \textcolor{red}{E} \cup \textcolor{blue}{E}$.
\end{definition}

Intuitively, edges from $\textcolor{red}{E}$ will encode equivalence classes, \emph{i.e.}, e-boxes, while edges from $\textcolor{blue}{E}$ will encode lambda-abstraction boxes.

Consider an example in the middle part of Figure~\ref{fig:e-cospan-example}.
The e-hypergraph in the example is given by $V = \{v_{1}, \ldots w_{10}\}$, $E = \{e_{1}, \ldots, e_{7}\}$, $l_{E} = {e_{1} \mapsto \bot, e_{2} \mapsto \bot, e_{3} \mapsto g, \ldots, e_{7} \mapsto f}$ and then we have
\[
\begin{array}{cccc}
  v_{4} \textcolor{red}{<^{\mu}} e_{1} & v_{7} \textcolor{blue}{<^{\mu}} e_{2} & v_{4} \consistency e_{3} & v_{4} \not \consistency v_{10}\\
  e_{3} \textcolor{red}{<^{\mu}} e_{1} & e_{5} \textcolor{blue}{<^{\mu}} e_{2} &  e_{3} \consistency u_{1} & e_{2} \not \consistency e_{6}\\
  e_{2} \textcolor{red}{<^{\mu}} e_{1} & w_{7} \textcolor{blue}{<^{\mu}} e_{2} & e_{2} \consistency e_{4} & e_{4} \not \consistency e_{7}\\  
  \ldots & \ldots & \ldots & \ldots \\
  v_{12} \textcolor{red}{<^{\mu}} e_{1} & w_{9} \textcolor{blue}{<^{\mu}} e_{2} & v_{11} \consistency v_{12} & w_{6} \not \consistency w_{10}
\end{array}
\]
$\consistency_{e_{1}}$ defines a partition on immediate successors of $e_{1}$ which we delimit by a vertical dashed line.
Edges in $\textcolor{red}{E}$ are depicted with a dashed box, while edges in $\textcolor{blue}{E}$ --- with a round solid box.

\begin{figure}

\[
\adjustbox{scale=0.6}{
\tikzfig{./figures/closed_iso_example_2}
}
\]
\caption{Cospan of e-hypergraphs example}
\label{fig:e-cospan-example}
\end{figure}

\begin{definition}[E-hypergraph homomorphism]
TODO
\end{definition}

E-hypergraphs and their homomorphisms define a category $\catname{EHyp}({\Sigma})$.

Similarly, we have a category of (extended) cospans formed by such e-hypergraphs where morphisms are of the form
\[
n \xrightarrow{f_{\text{ext}}} n' \xrightarrow{f_{\text{int}}} \mathcal{G} \xleftarrow{g'_{\text{int}}} m' \xleftarrow{g_{\text{ext}}} m
\]

where $n$ and $m$ are external input and output interfaces respectively and $n'$, $m'$ are input and output internal interfaces.
$n,n',m,m'$ are discrete ordered e-hypergraphs.
We further require that $f_{\text{ext}}$ and $g_{\text{ext}}$ are monos and all vertices in the image of $f_{\text{ext}};f_{\text{int}}$ (respectively, $g_{\text{ext}};g_{\text{int}}$) are top-level.

To build a correspondence between terms of a closed SMC and e-hypergraphs with interfaces (above) we restrict the cospans of the latter to monogamous cospans.

\begin{definition}
  We call a cospan 
  \[
n \xrightarrow{f_{\text{ext}}} n' \xrightarrow{f_{\text{int}}} \mathcal{G} \xleftarrow{g'_{\text{int}}} m' \xleftarrow{g_{\text{ext}}} m
\]
\textit{weak} monogamous if
\begin{itemize}
  \item in- and out- degrees of every vertex is at most 1;
  \item $f_{\text{int}}$ and $g_{\text{int}}$ are monos;
  \item Vertices with in-degree (respectively, out-degree) of 0 are precisely the image of $f_{\text{int}}$ (respectively, $g_{\text{int}}$).
\end{itemize}

\end{definition}


\begin{definition}[Isomorphic cospans]
Consider a relation 
\[
R = \{ x R y \text{ if } f_{\text{int}}(x) \consistency f_{\text{int}}(y) \;\text{or}\; \exists z \;\text{s.t.}\; z \textcolor{blue}{<^{\mu}} f_{\text{int}}(x) \text{ and } z \textcolor{blue}{<}^{\mu} f_{\text{int}}(y)\}
\]
for $x, y \in n'$.
And let $S$ be its reflexive closure.
The latter partitions $n'$ into non-empty subsets $\{p_{j}\}^{k}_{j=1}$.
We get an analogous partition for $m'$.

Two extended cospans are isomorphic if there exist isomorphisms $\alpha$, $\beta$ and $\gamma$ making the following diagram commute and such that $\alpha$ and $\gamma$ preserve order within each $p_j$.
\[
\scalebox{0.8}{
    \tikzfig{../../figures/combinatorial_semantics/isomorphic_e_cospans}
}
\]
\end{definition}


% Consider an example of isomorphic and non-isomorphic cospans below and their respective string diagrams

% \begin{figure}
%   \begin{subfigure}[c]{0.4\linewidth}
%     \[
%     \adjustbox{scale=0.6}{
%     \tikzfig{./figures/closed_iso_example_1}
%     }
%     \]
%   \end{subfigure}
%   \hfill
%   \begin{subfigure}[c]{0.4\linewidth}
%     \[
%     \adjustbox{scale=0.6}{
%     \tikzfig{./figures/closed_iso_example_2}
%     }
%     \]
%   \end{subfigure}
% \end{figure}

\begin{definition}

We define ordered sets of \textit{input} (respectively, \textit{output}) vertices of a hierarchical edge $e$ as the vertices $v_{i}$ such that $e < v_{i}$ which are in the image of $f_{\text{int}}$ (respectively, $g_{\text{ext}}$) with the induced ordering.
\end{definition}

\begin{definition}

Given an ordered set $S$ of labelled vertices we can form words formed by concatenating the labels.
We will denote such words as $w(S)$.
\end{definition}

\begin{definition}
We call a monogamous cospan
\[
  n \xrightarrow{f_{\text{ext}}} n' \xrightarrow{f_{\text{int}}} \mathcal{G} \xleftarrow{g'_{\text{int}}} m' \xleftarrow{g_{\text{ext}}} m
\]
\textit{well-typed} if all hierarchical edges of $\mathcal{G}$ are well-typed in the sense below.
\begin{itemize}
  \item For each $e \in \textcolor{red}{E}$ consider sets $I$ and $O$ of its input and output vertices partitioned according to $\consistency_{e}$.
        $e$ is well-typed if for each element $S$ of the partition of $I$ $w(S) = w(s(e))$ and similarly for $O$ and $t(e)$.
  \item For each $e \in \textcolor{blue}{E}$ consider sets $I$ and $O$ of its input and output vertices.
        $e$ is well-typed if there exists an object $B \in \text{obj}_{\Sigma_{O}}$ such that $[...s(e), B] = w(I)$ and $B \multimap w(O) = w(t(e))$
\end{itemize}
\end{definition}

\textcolor{red}{TODO: dpo rewriting}