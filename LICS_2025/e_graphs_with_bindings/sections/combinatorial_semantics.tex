\section{Combinatorial semantics}


We can extend the definition of e-hypergraphs for $\catname{SLat}$-enriched symmetric monoidal categories~\cite{ghica2024equivalencehypergraphsegraphsmonoidal} to accommodate new structure brought by closedness.

\begin{definition}
Closed monoidal signature $\Sigma$ is a pair $(\Sigma_{O},\Sigma_{M})$ of signatures for objects and signatures for morphisms with sources and targets in $\text{obj}_{\Sigma_{O}}$ defined as
\begin{itemize}
  \item $I$ (monoidal unit) is in $\text{obj}_{\Sigma_{O}}$
  \item For every $o \in \Sigma_{O}$, $o \in \text{obj}_{\Sigma_{O}}$;
  \item For every $o_1$ and $o_2$ $ \in \text{obj}_{\Sigma_{O}}$, $o_1 \otimes o_2 \in \text{obj}{\Sigma_{O}}$ and $o_1 \multimap o_2 \in \text{obj}_{\Sigma_{O}}$
\end{itemize}
\end{definition}

\begin{definition}
 Closed symmetric monoidal theory ($\catname{CSMT}$) over a closed monoidal signature $\Sigma$ and equations $\mathcal{E}$ denoted as $\catname{CSMT}(\Sigma, \mathcal{E})$ is given by a set of terms $T$
 \begin{align*}
 T \coloneq\; &t : A \to B \in \Sigma_{M} \;|\; t_1 \otimes t_2 : A \otimes C \to B \otimes D \text{ for } t_{1} : A \to B, t_{2} : C \to D \in T\\
            & \;|\; t_{1};t_{2} : A \to C \text{ for } t_{1} : A \to B, t_{2} : B \to C \in T \; | \;  \Lambda(t) : A \to X \multimap B \text{ for } t : A \otimes X \to B \\
            &  \;|\; \text{eval}(t_1 \otimes t_2) : B \text{ for } t_1 : X \multimap B, t_2 : X \in T \\
            &  \;|\; \text{sym}_{A,B} \;|\; id_{A}
 \end{align*}
 quotiented by the laws of a closed symmetric monoidal category.
\end{definition}
% \question{Maybe types and terms combined is not a good idea}
\begin{definition}

We define an e-hypergraph $\mathcal{G}$ over a closed symmetric monoidal signature $\Sigma$ as a tuple $(V,E,s,t, l_{V}, l_{E} \textcolor{red}{<}, \textcolor{blue}{<},\consistency)$ where
\begin{itemize}
  \item $V$ is a set of vertices;
  \item $E = \textcolor{gray}{E} \cup \textcolor{red}{E} \cup \textcolor{blue}{E}$ is a set of hyperedges that include \textit{plain} edges and two types of \textit{hierarchical} edges respectively;
  \item $s,t : E \to V^{*}$ are source and target functions;
  \item $l_{E} : E \to \Sigma_{M} + 1$, $l_{V} : V \to \text{obj}_{\Sigma_{O}}$ are \textit{label} functions;
  \item $\textcolor{red}{<} : V + E \to \textcolor{red}{E}$ and $\textcolor{blue}{<} : V + E \to \textcolor{blue}{E}$ are partial \textit{child} functions that are disjoint: if $\textcolor{red}{<}(x)$ is defined then $\textcolor{blue}{<}(x)$ is undefined and vice-versa;
  \item $\consistency = \bigcup_{p} \consistency_{p}$, where $\consistency_{p}$ is a \textit{consistency} relation defined on each set $\{x \in V + E \;|\; p \;\textcolor{red}{<^{\mu}}\; x\}$;
\end{itemize} 
\end{definition}

$\textcolor{red}{E}$ will contain edges that encode the equivalence classes and $\textcolor{blue}{E}$ the ones that defined the lambda abstraction.
All the above functions and relations must satisfy the obvious properties (TODO: add them).

Similarly, we have a category of (extended) cospans formed by such e-hypergraphs where morphisms are of the form
\[
n \xrightarrow{f_{\text{ext}}} n' \xrightarrow{f_{\text{int}}} \mathcal{G} \xleftarrow{g'_{\text{int}}} m' \xleftarrow{g_{\text{ext}}} m
\]

where $n$ and $m$ are external input and output interfaces respectively and $n'$, $m'$ are input and output internal interfaces.
$n,n',m,m'$ are discrete ordered e-hypergraphs.
We further require that $f_{\text{ext}}$ and $g_{\text{ext}}$ are monos and all vertices in the image of $f_{\text{ext}};f_{\text{int}}$ (respectively, $g_{\text{ext}};g_{\text{int}}$) are top-level.

To build a correspondence between terms of a closed SMC and e-hypergraphs with interfaces (above) we restrict the cospans of the latter to monogamous cospans.

\begin{definition}
  We call a cospan 
  \[
n \xrightarrow{f_{\text{ext}}} n' \xrightarrow{f_{\text{int}}} \mathcal{G} \xleftarrow{g'_{\text{int}}} m' \xleftarrow{g_{\text{ext}}} m
\]
\textit{weak} monogamous if
\begin{itemize}
  \item in- and out- degrees of every vertex is at most 1;
  \item $f_{\text{int}}$ and $g_{\text{int}}$ are monos;
  \item Vertices with in-degree (respectively, out-degree) of 0 are precisely the image of $f_{\text{int}}$ (respectively, $g_{\text{int}}$).
\end{itemize}

\end{definition}


\begin{definition}[Isomorphic cospans]
Consider a relation 
\[
R = \{ x R y \text{ if } f_{\text{int}}(x) \consistency f_{\text{int}}(y) \;\text{or}\; \exists z \;\text{s.t.}\; z \textcolor{blue}{<^{\mu}} f_{\text{int}}(x) \text{ and } z \textcolor{blue}{<}^{\mu} f_{\text{int}}(y)\}
\]
for $x, y \in n'$.
And let $S$ be its reflexive closure.
The latter partitions $n'$ into non-empty subsets $\{p_{j}\}^{k}_{j=1}$.
We get an analogous partition for $m'$.

Two extended cospans are isomorphic if there exist isomorphisms $\alpha$, $\beta$ and $\gamma$ making the following diagram commute and such that $\alpha$ and $\gamma$ preserve order within each $p_j$.
\[
\scalebox{0.8}{
    \tikzfig{../../figures/combinatorial_semantics/isomorphic_e_cospans}
}
\]
\end{definition}


Consider an example of isomorphic and non-isomorphic cospans below and their respective string diagrams

\begin{figure}
  \begin{subfigure}[c]{0.4\linewidth}
    \[
    \adjustbox{scale=0.6}{
    \tikzfig{./figures/closed_iso_example_1}
    }
    \]
  \end{subfigure}
  \hfill
  \begin{subfigure}[c]{0.4\linewidth}
    \[
    \adjustbox{scale=0.6}{
    \tikzfig{./figures/closed_iso_example_2}
    }
    \]
  \end{subfigure}
\end{figure}

\begin{definition}

We define ordered sets of \textit{input} (respectively, \textit{output}) vertices of a hierarchical edge $e$ as the vertices $v_{i}$ such that $e < v_{i}$ which are in the image of $f_{\text{int}}$ (respectively, $g_{\text{ext}}$) with the induced ordering.
\end{definition}

\begin{definition}

Given an ordered set $S$ of labelled vertices we can form words formed by concatenating the labels.
We will denote such words as $w(S)$.
\end{definition}

\begin{definition}
We call a monogamous cospan
\[
  n \xrightarrow{f_{\text{ext}}} n' \xrightarrow{f_{\text{int}}} \mathcal{G} \xleftarrow{g'_{\text{int}}} m' \xleftarrow{g_{\text{ext}}} m
\]
\textit{well-typed} if all hierarchical edges of $\mathcal{G}$ are well-typed in the sense below.
\begin{itemize}
  \item For each $e \in \textcolor{red}{E}$ consider sets $I$ and $O$ of its input and output vertices partitioned according to $\consistency_{e}$.
        $e$ is well-typed if for each element $S$ of the partition of $I$ $w(S) = w(s(e))$ and similarly for $O$ and $t(e)$.
  \item For each $e \in \textcolor{blue}{E}$ consider sets $I$ and $O$ of its input and output vertices.
        $e$ is well-typed if there exists an object $B \in \text{obj}_{\Sigma_{O}}$ such that $[...s(e), B] = w(I)$ and $B \multimap w(O) = w(t(e))$
\end{itemize}
\end{definition}