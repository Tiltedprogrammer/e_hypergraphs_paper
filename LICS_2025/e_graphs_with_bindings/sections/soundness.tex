\section{Soundness and completeness}

In this section we explore soundness and completeness of DPOI rewriting introduced in previous section with respect to rewriting of closed $\Sigma^{+}$-terms.
First we recall the rewriting of $\Sigma$-terms modulo SMC laws.

\begin{definition}
\label{def:rewrite}
    We say that a $\Sigma$-term $f$ rewrites to a $\Sigma$-term $g$ modulo SMC laws via a rewrite rule $\langle l, r \rangle$ (notation, $\leadsto_{\langle l, r \rangle}$) if they are representable as
    \[
    f = c_{1};(id_{k} \otimes l);c_{2} \qquad g = c_{1};(id_{k} \otimes r);c_{2}~.
    \]
\end{definition}
This notion generalises to rewriting of closed $\Sigma$-terms
\begin{definition}
    For closed-$\Sigma$-terms $f,g$, $f \leadsto_{\langle l, r \rangle}$ modulo SMC laws if they are representable as either in Definition~\ref{def:rewrite}, or as 
    \[
    f = c_{1};(id_{k} \otimes d);c_{2} \qquad g = c_{1};(id_{k} \otimes e);c_{2}
    \]
    and $d \leadsto_{\langle l, r \rangle} e$
\end{definition}
\begin{lemma}
\label{lemma:normal_form}
    Every closed $\Sigma^{+}$ term $f$ can be equivalently represented as
    \[
    f_{1} + \ldots + f_{n}
    \] modulo SMC laws and $\catname{SLat}$-equations and distributivity law~\ref{law:distributivity} where none of $f_{i}$ contain the join operator.
\end{lemma}
% \begin{proof}
% Such forms are given by orienting $\catname{SLat}$-equations such that $+$ gets propagated to the top, as, for example $f \otimes (g + h) = f \otimes g + g \otimes h$.
% Most notably, we can also escape lambda abstractions by using distributivity law~\ref{law:distributivity}.
% \end{proof}
This takes us to the following definition of rewriting for closed $\Sigma^{+}$-terms.
\begin{definition}
    For closed $\Sigma^{+}$-term $f,g$, $f \leadsto_{\langle l, r \rangle}$ SMC laws, $\catname{SLat}$-equations and distributivity law~\ref{law:distributivity} if they are representable as
    \[
    f = f_{1} + \ldots + f_{i} + \ldots + f_{n} \qquad g = f_{1} + \ldots + f_{j} + \ldots + f_{n}
    \]
    such that there is a permutation $\sigma$ such that there exist indices $i,j$ and $\sigma(f_{i}) \leadsto f_{j}$.
\end{definition}
Next, to formulate the correspondence between rewriting systems we first define interpretation $\llbracket - \rrbracket$ of closed $\Sigma^{+}$ terms in $\WellTypedMdaEcospans$.
It follows by first defining the interpretation of operators by extending the interpretation of plain $\Sigma$ terms as morphisms in $\MdaCospans$ as given in Appendix~\ref{sec:appendix:interpretation}.
The only missing cases, $\llbracket \textsf{ev}_{A,B} \rrbracket$ and $\llbracket \Lambda_{A,B,C} \rrbracket$, are given in Figure~\ref{fig:ev_and_lambda}.

\begin{figure}
    \begin{subfigure}{0.45\linewidth}
\[
\adjustbox{scale=0.6}{
    \tikzfig{./figures/ev_interpretation}
}
\]
    \end{subfigure}
    \hfill
    \begin{subfigure}{0.45\linewidth}
        \[
        \adjustbox{scale=0.6}{
            \tikzfig{./figures/lambda_interpretation}
        }
        \]
    \end{subfigure}
\caption{$\llbracket \textsf{ev} \rrbracket$ (left) and $\llbracket \Lambda(f) \rrbracket$ (right)}
\label{fig:ev_and_lambda}
\end{figure}

Then we can make $\WellTypedMdaEcospans$ into $\catname{SLat}$-category by defining a join of two cospans and introducing DPOI rewrite rules to satisfy the laws of $\catname{SLat}$ symmetric monoidal category.
We also add a structural rewrite schema rule corresponding to distributivity law~\ref{law:distributivity}.
The join of two cospans is defined as in Figure~\ref{fig:f+g} and the quotienting is done through DPOI rewrite schema rules, one for each axiom.
For example, the distributivity rule for $\otimes$ is given by the following schema rule
\[
\adjustbox{width=\linewidth}{
\tikzfig{./figures/semilattice_rule_1}~.
}
\]

We collect all such rewrite schemas into a set $\mathcal{S}$ and then a category $\WellTypedMdaEcospans / \mathcal{S}$ is a closed $\catname{SLat}$ SMC.

\begin{definition}[Quotient by rewrites]  
    Given a set of DPOI rewrite rules $\mathcal{E}$,  we denote by $\WellTypedMdaEcospans/\mathcal{E}$ the category $\WellTypedMdaEcospans$ quotiented by the following relation.
    \[
        f \sim g \quad \text{if} \quad f \Rrightarrow^{*}_{\mathcal{E}} g ~ . 
    \]
\end{definition}
    
This explicit quotienting allows us to only consider cospans of the form $f_1 + \ldots + f_{n}$, similar to Lemma~\ref{lemma:normal_form}, such that carrier of $f_{i}$ does not contain any hierarchical edges from $\textcolor{red}{E}$.

$\llbracket - \rrbracket$ is then induced by its definition on operators by induction
\[
\llbracket f;g \rrbracket = \llbracket f \rrbracket ; \llbracket g \rrbracket \quad
\llbracket f \otimes g \rrbracket = \llbracket f \rrbracket \otimes \llbracket g \rrbracket \quad
\llbracket f + g \rrbracket = \llbracket f \rrbracket + \llbracket g \rrbracket
\]

\begin{proposition}[Proposition 26 and Proposition 27~\cite{fscd}]
\label{prop:fscd}
For two closed $\Sigma$-terms $f$ and $g$ and a $\mathcal{E}$ equation $l = r$  $f \leadsto_{\langle l, r \rangle} g$ if and only if $\llbracket f \rrbracket \Rrightarrow_{\langle \llbracket l \rrbracket, \llbracket r \rrbracket \rangle} \llbracket g \rrbracket$ where the latter is defined in a subcategory of $\WellTypedMdaEcospans$ where carriers only contain edges from $\colorbox{yellow}{E} \cup \textcolor{blue}{E}$.
\end{proposition}

\begin{proposition}
    For two closed $\Sigma^{+}$-terms $f$ and $g$ and a $\mathcal{E}$ equation $l = r$  $f \leadsto_{\langle l, r \rangle} g$ if and only if $\llbracket f \rrbracket \Rrightarrow_{\langle \llbracket l \rrbracket, \llbracket r \rrbracket \rangle} \llbracket g \rrbracket$ in $\WellTypedMdaEcospans / \mathcal{S}$.
\end{proposition}
\begin{proof}
By definition, we have 
\[
  f = f_{1} + \ldots + f_{n} \quad \text{and} \quad g = f_{1} + \ldots + f_{n},
\] such that $f_{i} \leadsto f_{j}$ for some $i, j$.
This yields two cospans of the form $\llbracket f_{1} \rrbracket + \ldots + \llbracket f_{n} \rrbracket$ for $\llbracket f \rrbracket$ and $\llbracket g \rrbracket$ and, by noting that every $f_{i}$ is a $\Sigma$-term, and carriers in every $\llbracket f_{i} \rrbracket$ have no edges from $\textcolor{red}{E}$, we can apply Proposition~\ref{prop:fscd}.
\end{proof}

\begin{figure}
\[
        \adjustbox{scale=0.5}{
            \tikzfig{./figures/f_plus_g_new}
        }
\]
\captionsetup{belowskip=-1ex}
\caption{$+$ of two morphisms in $\WellTypedMdaEcospans$}
\label{fig:f+g}
\end{figure}

\section{Conclusion}
In this work we introduced e-graphs with bindings formalised as morphisms of a closed $\catname{SLat}$-SMC.
We then provided a combinatorial representation for such morphisms in terms of hierarchical hypergraphs (e-hypergraphs) with a suitable notion of sound and complete DPO rewriting, making it possible for an e-hypergraph to represent equivalences derivable from a set of equations $\mathcal{E}$.
An important future work is exploring whether such combinatorial structures may lead to better algorithmics for doing equality saturation for $\lambda$-calculus.
