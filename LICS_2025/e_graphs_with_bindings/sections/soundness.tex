\section{Soundness and completeness}

In this section we explore soundness and completeness of DPOI rewriting introduced in previous section with respect to rewriting of closed $\Sigma^{+}$-terms.
First we recall the rewriting of $\Sigma$-terms modulo SMC laws.

\begin{definition}
\label{def:rewrite}
    We say that a $\Sigma$-term $f$ rewrites ($\leadsto$) to a $\Sigma$-term $g$ modulo SMC-laws via a rewrite rule $\langle l, r \rangle$ if they are representable as
    \[
    f = c_{1};(id_{k} \otimes l);c_{2} \qquad g = c_{2};(id_{k} \otimes r);c_{2}~.
    \]
    These representations are so-called SMC normal forms.
\end{definition}

This notion generalises to rewriting of closed $\Sigma$-terms

\begin{definition}
    We say that a closed-$\Sigma$-term $f$ rewrites ($\leadsto$) to a closed-$\Sigma$-term $g$ modulo SMC-laws via a rewrite rule $\langle l, r \rangle$ if they are representable as either in Definition~\ref{def:rewrite}, or as 
    \[
    f = c_{1};(id_{k} \otimes d);c_{2} \qquad g = c_{2};(id_{k} \otimes e);c_{2}
    \]
    and $d \leadsto_{\langle l, r \rangle} e$
\end{definition}

\begin{lemma}

    Every closed $\Sigma^{+}$ term $f$ can be equivalently represented as
    \[
    f_{1} + \ldots + f_{n}
    \] where none of $f_{i}$ contain the join operator.
\end{lemma}
\begin{proof}
Such forms are given by orienting $\catname{SLat}$-equations such that $+$ gets propagated to the top, as, for example $f \otimes (g + h) = f \otimes g + g \otimes h$.
Most notably, we can also escape lambda abstractions by using distributivity law~\ref{law:distributivity}.
\end{proof}

This takes us to the following definition of rewriting for closed $\Sigma^{+}$-terms.
\begin{definition}
    We say that a closed-$\Sigma^{+}$-term $f$ rewrites ($\leadsto$) to a closed-$\Sigma^{+}$-term $g$ modulo SMC-laws via a rewrite rule $\langle l, r \rangle$ if they are representable as
    \[
    f = f_{1} + \ldots + f_{n} \qquad g = f_{1} + \ldots + f_{n}
    \]
    such that there is a permutation $\sigma$ such that there exist indices $i,j$ and $\sigma(f_{i}) \leadsto f_{j}$.
\end{definition}

To formulate the correspondence between rewriting systems we first define interpretation of closed $\Sigma^{+}$ terms in $\WellTypedMdaEcospans$.
It follows by first defining the interpretation of generators by extending the interpretation of plain $\Sigma$ terms as morphisms in $\MdaCospans$ as given in Appendix~\ref{sec:appendix:interpretation}.
The only missing cases are $[\textsf{ev}_{A,B}]$ and $[\Lambda_{A,B,C}]$.

\begin{figure}
    \begin{subfigure}{0.45\linewidth}
\[
\adjustbox{scale=0.5}{
    \tikzfig{./figures/ev_interpretation}
}
\]
    \end{subfigure}
    \hfill
    \begin{subfigure}{0.45\linewidth}
        \[
        \adjustbox{scale=0.5}{
            \tikzfig{./figures/lambda_interpretation}
        }
        \]
    \end{subfigure}

\end{figure}

Then we can make $\WellTypedMdaEcospans$ into closed $\catname{SLat}$ category by defining a join of two cospans, $\Lambda$ of a cospan, and a designated $\textsf{ev}$ cospan and introducing DPOI rewrite rules to make these construction satisfy the laws of closed $\catname{SLat}$ symmetric monoidal category.
The join of two cospans is defined as in Figure~\ref{fig:f+g} and the quotienting is done through DPOI rewrite schema rules one for each axiom.
For example, the distributivity rule for $\otimes$ is given by the following schema rule
\[
\adjustbox{width=\linewidth}{
\tikzfig{./figures/semilattice_rule_1}~.
}
\]

We collect all such rewrite schemas into a set $\mathcal{S}$ and then a category $\WellTypedMdaEcospans / \mathcal{S}$ is a closed $\catname{SLat}$ SMC.

\begin{proposition}

For two closed $\Sigma^{+}$-terms $f$ and $g$ and a $\mathcal{E}$-equation $l = r$  $f \leadsto g$ if and only if $\llbracket f \rrbracket \Rrightarrow{} \llbracket g \rrbracket$ in $\WellTypedMdaEcospans / \mathcal{S}$.
\end{proposition}
\begin{proof}
    
\end{proof}

\begin{figure}
\[
        \adjustbox{scale=0.5}{
            \tikzfig{./figures/f_plus_g_new}
        }
\]
\caption{$+$ of two morphisms in $\WellTypedMdaEcospans$}
\label{fig:f+g}
\end{figure}