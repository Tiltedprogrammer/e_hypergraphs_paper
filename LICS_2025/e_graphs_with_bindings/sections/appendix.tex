\subsection{$\catname{SLat}$}
\label{sec:appendix:slat}

In this section we will define the category of semilattices that we use as a base for enrichment throughout the paper.

\begin{definition}[Semilattice]
    A \textit{semilattice} is a set equipped with an operation that we denote as $+$ which is associative, commutative and idempotent.
  \end{definition}
  
  Note that we do not require the existence of a unit for $+$. 
  Semilattices that satisfy this extra requirement are sometimes called \textit{bounded}, i.e., they are idempotent commutative monoids.
  
  \begin{definition}[Semilattice homomorphism]
  
  A homomorphism between two semilattices $S_{1}$ and $S_{2}$ is a map $h$ that respects $+$.
  That is, for all $s,s' \in S_{1}$, $h(s +_{S_{1}} s') = h(s) +_{S_{2}} h(s')$.
  \end{definition}
  
  \begin{definition}[Category of semilattices]
    
  Semilattices with their respective homomorphisms form a category that we denote $\catname{SLat}$.
  \end{definition}
  
  \begin{proposition}
    $\catname{SLat}$ is a closed symmetric monoidal category.
  \end{proposition}
  \begin{proof}
    The tensor product of two semilattices $S_{1}$ and $S_{2}$ is defined as follows.
    $S_{1} \otimes S_{2}$ consists of pairs $(s_1,s_2)$ $s_{1} \in S_{1}$, $s_{2} \in S_{2}$ quotiented by commutativity, idempotence and associativity and additionally by the following relations
    \begin{itemize}
      \item $(s_{1} +_{S_{1}} s_{1}',s_{2}) \equiv (s_{1},s_{2}) +_{S_{1} \otimes S_{2}} (s_{1}',s_{2})$
      \item $(s_{1}, s_{2} +_{S_{2}} s_{2}') \equiv (s_{1},s_{2}) +_{S_{1} \otimes S_{2}} (s_{1}',s_{2})$
    \end{itemize}
  
    The unit for this tensor product is $I = \{*\}$ --- a one-element semilattice.
    Clearly $S \otimes I \cong S$ by mapping $(s,*) \mapsto s$ for all $s \in S$ and vice versa.
    The symmetry, associators and unitors are then obvious morphisms.
    Finally, the category is closed since the set of homomorpisms between two semilattices is a semilattice by defining $(f + g)(x)$ as $f(x) + g(x)$.
    $f + g$ is a homomorphism since $(f + g)(x+y) = f(x+y) + g(x+y) = f(x) + f(y) + g(x) + g(y) = f(x) + g(x) + f(y) + g(y) = f(x+y) + g(x+y)$.
  \end{proof}
  
  This makes $\catname{SLat}$ a suitable base for enrichment.

  \begin{definition}
    Forgetful functor $U : \catname{SLat} \to \catname{Set}$ is given by $\catname{SLat}(I_{\catname{SLat}, -}) : \catname{SLat} \to \catname{Set}$.
    \end{definition}
    
    Intuitively, the above functor returns the underlying set of a given semilattice $S$ as each morphism from $\{*\} \to S$ picks out an element of $S$.
    
    \begin{proposition}[Special case of Proposition 6.4.6~\cite{Borceux_1994}]
      The forgetful functor $U : \catname{SLat} \to \catname{Set}$ has a left adjoint free functor $F : \catname{Set} \to \catname{SLat}$.
    \end{proposition}
    \begin{proof}
      The functor $F$ is defined by letting $F(A) = \coprod_{A} I_{\catname{SLat}}$ where the latter is a coproduct which is a `free' product of semilattices defined as follows.
      The elements of $S_{1} \coprod S_{2}$ are sequences $s_{1} + s_{2} + \ldots + s_{n}$ where each $s_{i}$ is either from $S_{1}$ or $S_{2}$ quotiented by all relations in $S_{1}$ and $S_{2}$.
      The adjunction is then given by the following natural isomorphisms
      \begin{align*}
      \catname{SLat}(\coprod_{A}(I_{\catname{SLat}}), B) &\cong \prod_{A}(\catname{SLat}(I_{\catname{SLat}}, B))\\
                                                         &\cong \catname{Set}(A,\catname{SLat}(I_{\catname{SLat}}, B))\\
                                                         &\cong \catname{Set}(A, U(B))
      \end{align*}
      The fist isomorphism is given by the fact that hom-functor makes limits into colimits in its first argument, the second being given by the property that $|\catname{Set}(A,B)| = |B|^{|A|} = |\underbrace{B \times \ldots \times B}_{|A|}|$, and the last isomorphism is given by the definition of $U$.
      Furthermore, we have, 
      \[
      F(I) \cong I_{\catname{SLat}}
      \]
      and 
      \begin{align*}
      F(A) \otimes F(B) &\cong (\coprod_{A} I_{\catname{SLat}}) \otimes (\coprod_{B} I_{\catname{SLat}})\\
            &\cong \coprod_{A} (I_{\catname{SLat}} \otimes \coprod_{B} I_{\catname{SLat}})\\
            &\cong \coprod_{A} (\coprod_{B} (I_{\catname{SLat}} \otimes I_{\catname{SLat}}))\\
            &\cong \coprod_{A} (\coprod_{B} I_{\catname{SLat}})\\
            &\cong \coprod_{A \times B} I_{\catname{SLat}}\\
            &\cong F(A \times B)
      \end{align*}
    In the above we used the fact that functors $- \otimes X$ and $X \otimes -$ preserve colimits as they are both left-adjoint by symmetry and monoidal closedness of $\catname{SLat}$.
    \end{proof}

\subsection{Pushout in $\catname{EHyp}(\Sigma)$}
\label{sec:appendix:pushout}

In constructing the pushout the functional versions of e-hypergraph relations will be useful.
\begin{remark}
    $\textcolor{red}{<}$, $\textcolor{blue}{<}$ and $\consistency$ can be considered as (partial) functions defined on $V_{\mathcal{F}} + E_{\mathcal{F}}$, \textit{i.e.}, on the coproduct of vertices and edges.
    To make things well-typed, we will use corresponding coproduct injections $\iota_{V} : {V_{\mathcal{F}}} \to V_{\mathcal{F}} + E_{\mathcal{F}}$ and $\iota_{E} : {E_{\mathcal{F}}} \to V_{\mathcal{F}} + E_{\mathcal{F}}$ when passing either a vertex or an edge into these functions.
    For example, an immediate successor of a vertex $x$ can be written functionally as $\textcolor{red}{<_{\mathcal{F}}^{\mu}}(\iota_{V_{\mathcal{F}}}(x))$.
\end{remark}

\begin{remark}
    Using functional notation we can reformulate the notion of a homomorphism between two e-hypergraphs in the following way.
\end{remark}
\begin{definition}
        \label{def:e-homo-2}    
        A \emph{homomorphism} $\phi: \mathcal{F} \to \mathcal{G}$ of e-hypergraphs $\mathcal{F},\mathcal{G}$ is a pair of functions $\phi_V : V_{\mathcal{F}} \to V_{\mathcal{G}}, \phi_E : E_{\mathcal{F}} \to E_{\mathcal{G}}$ such that
        
        \begin{enumerate}
            \item $\phi$ is hypergraph homomorphism.
            
            \item When $x$ is not a top-level vertex such that $\textcolor{red}{<}(\iota_{V}(x))$ is defined, 
                \[
                \phi_{E}(\textcolor{red}{<_{\mathcal{F}}^{\mu}}(\iota_{V_{\mathcal{F}}}(x))) = \textcolor{red}{<_{\mathcal{G}}^{\mu}}(\phi_{V};\iota_{V_{\mathcal{G}}}(x))
                \]
                and
                \[
                \phi_{E}(\textcolor{red}{<_{\mathcal{F}}^{\mu}}(\iota_{E_{\mathcal{F}}}(x))) = \textcolor{red}{<_{\mathcal{G}}^{\mu}}(\phi_{E};\iota_{E_{\mathcal{G}}}(x))  
                \] when $x$ is a not top-level edge.
                \item When $x$ is not a top-level vertex such that $\textcolor{blue}{<}(\iota_{V}(x))$ is defined, 
                \[
                \phi_{E}(\textcolor{blue}{<_{\mathcal{F}}^{\mu}}(\iota_{V_{\mathcal{F}}}(x))) = \textcolor{blue}{<_{\mathcal{G}}^{\mu}}(\phi_{V};\iota_{V_{\mathcal{G}}}(x))
                \]
                and
                \[
                \phi_{E}(\textcolor{blue}{<_{\mathcal{F}}^{\mu}}(\iota_{E_{\mathcal{F}}}(x))) = \textcolor{blue}{<_{\mathcal{G}}^{\mu}}(\phi_{E};\iota_{E_{\mathcal{G}}}(x))  
                \] when $x$ is a not top-level edge.
                \item
            When $x \in E_{\mathcal{F}}$
            \[
                [\phi_{V};\iota_{V_{\mathcal{G}}}, \phi_{E};\iota_{E_{\mathcal{G}}} ]^{*}(\consistency_{\mathcal{F}}(\iota_{E_{\mathcal{F}}}(x)))
                \subseteq
                \consistency_{\mathcal{G}}(\phi_{E};\iota_{E_{\mathcal{G}}}(x))
            \]
            where $\phi_{V};\iota_{V_{\mathcal{G}}} : V_{\mathcal{F}} \to V_{\mathcal{G}} + E_{\mathcal{G}}$, and similarly for $\phi_{E};\iota_{E_\mathcal{G}}$ so that $[\phi_{V};\iota_{V_{\mathcal{G}}}, \phi_{E};\iota_{E_{\mathcal{G}}}] : V_{\mathcal{F}} + E_{\mathcal{F}} \to  V_{\mathcal{G}} + E_{\mathcal{G}}$.
            \item When $x \in V_{\mathcal{F}}$
            \[
                [\phi_{V};\iota_{V_{\mathcal{G}}}, \phi_{E};\iota_{E_{\mathcal{G}}}]^{*}(\consistency_{\mathcal{F}}(\iota_{V_{\mathcal{F}}}(x)))
                \subseteq
                \consistency_{\mathcal{G}}(\phi_{V};\iota_{V_{\mathcal{G}}}(x)).
            \]
            \end{enumerate}
\end{definition}

\begin{theorem}[Existence of pushouts in $\catname{EHyp}(\Sigma)$]
\label{th:existence_of_pushouts}
Consider the following span in $\catname{EHyp}(\Sigma)$
% https://q.uiver.app/#q=WzAsMyxbMCwwLCJaIl0sWzIsMCwiWCJdLFswLDIsIlkiXSxbMCwxLCJmIl0sWzAsMiwiZyIsMl1d
\[\begin{tikzcd}
	Z && X \\
	\\
	Y
	\arrow["f", from=1-1, to=1-3]
	\arrow["g"', from=1-1, to=3-1]
\end{tikzcd}\]
such that
\begin{enumerate}
\label{pushout:assumptions}
    \item $Z$ is a \textit{discrete} e-hypergraph.
    \item \label{assumption:equal_predecessors} $[f_{V}(v_i)) = [f_{V}(v_j))$ and $[g_{V}(v_i)) = [g_{V}(v_j))$ for all $v_{i},v_{j}$ in $V_{Z}$.
    \item \label{assumption:non_ambiguous_predecessors} If $[f_{V}(v)) \not = \varnothing$ then $[g_{V}(v)) = \varnothing$ and if $[g_{V}(v)) \not = \varnothing$ then $[f_{V}(v)) = \varnothing$.
    \item $\consistency(f_{V}(v_i)) = \consistency(f_{V}(v_j))$ and $\consistency(g_{V}(v_i)) = \consistency(g_{V}(v_j))$ for all $v_i,v_j$ in $V_{Z}$.
\end{enumerate}    
then the pushout $X +_{f,g} Y$ exists.
\end{theorem}
\begin{proof}

We next explicitly construct a pushout.
Consider the diagram below.
    \[
        \adjustbox{scale=1.25}{
            \begin{tikzcd}
            Z \arrow[r, "f"] \arrow[d, "g"']                                   & X \arrow[d, "\sfrac{\iota_1}{\sim R}"] \arrow[rdd, "j_1", bend left] &   \\
            Y \arrow[r, "\sfrac{\iota_2}{\sim R}"] \arrow[rrd, "j_2"', bend right] & \sfrac{X+Y}{\sim R} \arrow[rd, "u"]                              &   \\
                                                                            &                                                                  & Q
            \end{tikzcd}}
    \]
    Then, the pushout of e-hypergraphs $X$ and $Y$ is computed in two steps.
    First, a coproduct of $X+Y$ is computed which is 
    \[
        X + Y = \{V_{X} + V_{Y}, E_{X} + E_{Y}, s_{X+Y}, t_{X+Y}, \consistency_{X+Y}, \textcolor{red}{<_{X+Y}}, \textcolor{blue}{<_{X+Y}}\}
    \]
    where $s_{X+Y} : E_{X} + E_{Y} \to (V_{X} + V_{Y})^{*}$ which can be defined as a copairing $[s'_{X}, s'_{Y}] : E_{X} + E_{Y} \to (V_{X} + V_{Y})^{*}$ and $s'_{X} : E_{X} \to (V_{X} + V_{Y})^{*}$, $s'_{Y} : E_{Y} \to (V_{X} + V_{Y})^{*}$ defined as $s'_{X} = s_{X};\iota_{1,V}^{*}$ and $s'_{Y} = s_{Y};\iota_{2,V}^{*}$ where $\iota_1,\iota_2$ are corresponding coproduct injections, similarly for $t_{X+Y}, \textcolor{red}{<_{X+Y}}, \textcolor{blue}{<_{X+Y}}, \consistency_{X+Y}$.
    We will omit labels as they are irrelevant to pushout construction.
    Consider relations
    \begin{align*}    
    S_{V} &= \{\\
          &\;(x_i,y_j) \in (V_{X} + V_{Y}) \times (V_{X} + V_{Y})\; | \\
          &\;\exists z \in V_{Z} \; . \; x_i = f_{V};\iota_{1,V}(z) \text{ and } y_j = g_{V};\iota_{2,V}(z)\\
          &\;\text{ where $x_i \in V_{X}$ and $y_j \in V_{Y}$ }\\
          &\}\\
          &\cup\\
          &\{\\
          &\;(y_j,x_j) \in (V_{X} + V_{Y}) \times (V_{X} + V_{Y})\; | \\
          &\;\exists z \in V_{Z} \; . \; x_i = f_{V};\iota_{1,V}(z) \text{ and } y_j = g_{V};\iota_{2,V}(z)\\
          &\;\text{ where $x_i \in V_{X}$ and $y_j \in V_{Y}$ }\\
          &\}\\
          &\cup\\
          &\{\\
          &\;(x,x)\;\text{where}\; x \in V_{X} + V_{Y}\\
          &\}\\
        \\
    S_{E} &= \{(x,x) \text{ where } x \in E_{X} + E_{Y}\}
    \end{align*}
and let relations $R_{V},R_{E}$ be their transitive closures respectively.


We then quotient the set vertices and edges in $X + Y$ by $R_{V}$ and $R_{E}$ respectively.

\begin{align*}
        \sfrac{X + Y}{\sim (R_{V},R_{E})} = (&\\
            &\sfrac{V_{X} + V_{Y}}{\sim (R_{V},R_{E})},\\
            &\sfrac{E_{X} + E_{Y}}{\sim (R_{V},R_{E})},\\
            &\sfrac{s_{X+Y}}{\sim (R_{V},R_{E})},\\
            &\sfrac{t_{X,Y}}{\sim (R_{V},R_{E})},\\
            &\consistency_{\sfrac{X+Y}{\sim (R_{V},R_{E})}},\\
            &\textcolor{red}{<}_{\sfrac{X + Y}{\sim (R_{V},R_{E})}}\\
            &\textcolor{blue}{<}_{\sfrac{X + Y}{\sim (R_{V},R_{E})}}\\
        &)    
\end{align*}

We will refer to $\sim (R_{V},R_{E})$ just as $\sim$, \textit{e.g.}, by writing $\sfrac{V_{X} + V_{Y}}{\sim}$ and concrete relation will be clear from the context.
Where necessary we will refer to $S_{V}$ and $S_{R}$ as $\sim_{S}$.
We have 
\[
    \sfrac{s_{X+Y}}{\sim} : \sfrac{E_{X} + E_{Y}}{\sim} \to (\sfrac{V_{X} + V_{Y}}{\sim})^{*}
\]
and there is an obvious surjective function $[]_{V} : (V_{X} + V_{Y}) \to (\sfrac{V_{X} + V_{Y}}{\sim})$ that maps elements to their equivalence classes and $[]_{V}^{*}$ is its extension to sequences.
There is also $[] : E_{X} + E_{Y} \to \sfrac{E_{X} + E_{Y}}{\sim}$ and we will omit subscripts as the correct type will be clear from the argument.
We then define 
\[
    \sfrac{s_{X+Y}}{\sim}([e]) = s_{X+Y};[]^{*}(e) = [s_{X+Y}(e)]^{*}
\]
We will also use subscripts when it is important to tell if an element of $E_{X} + E_{Y}$ has a pre-image in either $E_{X}$ or $E_{Y}$ by writing $e_{x}$ or $e_{y}$.
Similarly, we will write $v_{x}$ to refer to a vertex with a pre-image in $V_{X}$.
Likewise, 
\[
    t_{\sfrac{X+Y}{\sim}}([e]) = [t_{X+Y}(e)]^{*}
\]

These definitions make $([]_{V},[]_{E})$ automatically a homomorphism with respect to source and target maps.
These maps are also automatically well-defined functions since $[e_1] = [e_2]$ implies $e_1 = e_2$ since the mappings for edges are mono.


Recall that $\textcolor{red}{<}$ is essentially a transitive closure of $\textcolor{red}{<^{\mu}}$ and hence we can first define 
    \[\textcolor{red}{<_{\sfrac{X+Y}{\sim}}^{\mu}}(\iota_{\sfrac{E_{X} + E_{Y}}{\sim}}[e])
    \] and 
    \[
        \textcolor{red}{<_{\sfrac{X+Y}{\sim}}^{\mu}}(\iota_{\sfrac{V_{X} + V_{Y}}{\sim}}[u])
    \] where $\iota_*$ are injections into $\sfrac{V_{X} + V_{Y}}{\sim} + \sfrac{E_{X} + E_{Y}}{\sim}$.
    We will consider several cases.
    First, assume that $u$ is an element of $V_{X+Y}$ and
    \begin{enumerate}
        \item If there exists $v$ such that $\textcolor{red}{<_{X+Y}^{\mu}}(\iota_{V_{X} + V_{Y}}(v))$ is defined and $u \sim v$
              we let
              \[
                \textcolor{red}{<_{\sfrac{X+Y}{\sim}}^{\mu}}(\iota_{\sfrac{V_{X} + V_{Y}}{\sim}}([u])) = [\textcolor{red}{<_{X+Y}^{\mu}}(\iota_{V_{X} + V_{Y}}(v))]
              \]
        \item \label{def:child_respects_connectivity} $u$ has no pre-image in $V_{Z}$ and $<_{X+Y}^{\mu}(\iota_{V_{X} + V_{Y}}(u))$ is undefined (\textit{i.e.}, $[u) = \varnothing$).
              If there exists $v$ such that $\textcolor{red}{<_{X+Y}^{\mu}}(\iota_{V_{X} + V_{Y}}(v)) = e'$ and such that there is an \textit{undirected} path from $[u]$ to $[v]$, then we define
              \[ 
                \textcolor{red}{<_{\sfrac{X+Y}{\sim}}^{\mu}}(\iota_{\sfrac{V_{X} + V_{Y}}{\sim}}[u]) = <_{\sfrac{X+Y}{\sim}}^{\mu}(\iota_{\sfrac{V_{X} + V_{Y}}{\sim}}[v])
              \]
    \end{enumerate}
    % Note that the existence of a path in the definition (2) implies that there exists $v' \sim v$ such that there is a path from $u$ to $v'$ and furthermore $v \not = v'$.
    % If $v' = v$ then it would mean that $[u) \not = \varnothing$ since $[v) \not = \varnothing$ and there is a path from $u$ to $v$.
    % There is no case when $u$ has a pre-image in $V_{Z}$ and yet $<_{X+Y}^{\mu}(\iota_{V_{X} + V_{Y}}(u))$ is undefined and there exists $v$ such that $<_{X+Y}^{\mu}(\iota_{V_{X} + V_{Y}}(v)) = e$ and there is a path from $[u]$ to $[v]$.
    % The existence of a path would mean that there is $u' \sim u$ and $v' \sim v$ such that there is a path from $u'$ to $v'$ and since $u$ has a pre-image in $V_{Z}$, $u = f_{V};\iota_{1,V}(z_1)$.
    % $v' \sim v$ means that both $v'$ and $v$ have a pre-image in $V_{Z}$ (unless $v' = v$ in which case $[u') \not = \varnothing$ as there is a path from $u'$ to $v'$) and since $u = f_{V};\iota_{1,V}(z_1)$ and $[u) = \varnothing$ $f_{V};\iota_{1,V}(z) = \varnothing$ for all $z$.
    % Since $[v) \not = \varnothing$ and $v' \sim v$, $v$ is necessarily in the image of $g_{V};\iota_{2,V}$ and for all $z$ $[g_{V};\iota_{2,V}(z)) \not = \varnothing$ as well as for some $z'$ such that $u'' = g_{V};\iota_{2,V}(z')$ and $u'' \sim_{S} u$ which would mean that the case (1) is applicable to $u$.
    % Similarly, when $u = g_{V};\iota_{2,V}(z)$.
    
    Otherwise, we leave $\textcolor{red}{<_{\sfrac{X+Y}{\sim}}^{\mu}}(\iota_{\sfrac{V_{X} + V_{Y}}{\sim}}([u]))$ undefined.
    % \question{In the case when images of $f$ and $g$ are top-level the function is undefined. Shall we introduce bottom?}
    Now, assume that $e$ is an element of $E_{X+Y}$.
    Consider two cases.
    \begin{enumerate}
    \item  $\textcolor{red}{<_{X+Y}^{\mu}}(\iota_{E_{X} + E_{Y}}(e)) = e'$.
            Then we define
        \[
            \textcolor{red}{<_{\sfrac{X+Y}{\sim}}^{\mu}}(\iota_{\sfrac{E_{X} + E_{Y}}{\sim}}[e]) = [\textcolor{red}{<_{X+Y}^{\mu}}(\iota_{E_{X} + E_{Y}}(e))]
        \]
    \item $[e) = \varnothing$ and there exists $v$ such that $<_{X+Y}^{\mu}(\iota_{V_{X} + V_{Y}}(v)) = e'$ and such that there is an undirected path from $[e]$ to $[v]$, we define 
        \[
            \textcolor{red}{<_{\sfrac{X+Y}{\sim}}^{\mu}}(\iota_{\sfrac{E_{X} + E_{Y}}{\sim}}[e]) = \textcolor{red}{<_{\sfrac{X+Y}{\sim}}^{\mu}}(\iota_{\sfrac{V_{X} + V_{Y}}{\sim}}[v])
        \] 
    \end{enumerate}
    Otherwise we leave $\textcolor{red}{<_{\sfrac{X+Y}{\sim}}^{\mu}}(\iota_{E_{X} + E_{Y}}{\sim}([e]))$ undefined.
    % We will also build this relation in steps, first we do steps for vertices (1) - (3), then step (1) for edges and finally, we will interleave step (2) for edges and step (4) for vertices until fixed point.
    % Essentially, this interleaving means that vertices inherit the relation from incident edges and vice versa.
    Clearly, all the cases above are disjoint.
    We similarly define $\textcolor{blue}{<_{\sfrac{X+Y}{\sim}}^{\mu}}$.

    Let's now define $\consistency_{\sfrac{X+Y}{\sim}}$.
    We can consider the consistency relation from the coproduct as a function 
    \[
        \consistency_{X+Y} : (V_{X} + V_{Y}) + (E_{X} + E_{Y}) \to 2^{(V_{X} + V_{Y}) + (E_{X} + E_{Y})}
    \]
    Quotienting the values of the function gives us 
    \[
        \consistency_{X+Y}' : V_{X} + V_{Y} + E_{X} + E_{Y} \to 2^{(\sfrac{V_{X} + V_{Y}}{\sim} + \sfrac{(E_{X} + E_{Y})}{\sim})}
     \]
    which is essentially $[ []_{V};\iota_{\sfrac{V_{X} + V_{Y}}{\sim}}, []_{E};\iota_{\sfrac{E_{X} + E_{Y}}{\sim}}]^{*}$ (a copairing extended to sequences that we will further denote as $[ []_{V}^{\consistency} []_{E}^{\consistency}]$) applied to the return value of $\consistency_{X+Y}$.
    We then first define an auxiliary relation $\consistency^{\hashtag}$ similarly to how $<^{\mu}_{\sfrac{X+Y}{\sim}}$ was defined. We begin with defining $\consistency^{\hashtag}_{\sfrac{X+Y}{\sim}}$ for vertices.

    \begin{enumerate}
        \item If there exists $v$ such that $\consistency_{X+Y}(\iota_{V_{X} + V_{Y}}(v)) \not = \varnothing$ and $u \sim v$, we let
              \ifdefined \ONECOLUMN
              \[
                \consistency_{\sfrac{X+Y}{\sim}}^{\hashtag}(\iota_{\sfrac{V_{X} + V_{Y}}{\sim}}([u]))
                =
                [[]_{V}^{\consistency},[]_{E}^{\consistency}]^{*}(\consistency_{X+Y}(\iota_{V_{X} + V_{Y}}(v)))
              \]
              \else
              \begin{align*}
                &\consistency_{\sfrac{X+Y}{\sim}}^{\hashtag}(\iota_{\sfrac{V_{X} + V_{Y}}{\sim}}([u]))\\
                &=\\
                &[[]_{V}^{\consistency},[]_{E}^{\consistency}]^{*}(\consistency_{X+Y}(\iota_{V_{X} + V_{Y}}(v)))
            \end{align*}
            \fi
        \item $u$ has no pre-image in $V_{Z}$ and $\consistency_{X+Y}(\iota_{V_{X} + V_{Y}}(u)) = \varnothing$ and there exists $v$ such that $\consistency_{X+Y}(\iota_{V_{X} + V_{Y}}(v)) \not = \varnothing$ and such that there is an undirected path from $[u]$ to $[v]$.
        Then we let
        \[
            \consistency_{\sfrac{X+Y}{\sim}}^{\hashtag}(\iota_{\sfrac{V_{X} + V_{Y}}{\sim}}([u])) = \consistency_{\sfrac{X+Y}{\sim}}^{\hashtag}(\iota_{\sfrac{V_{X} + V_{Y}}{\sim}}([v]))
        \]
    \end{enumerate}

    Next we define $\consistency_{\sfrac{X+Y}{\sim}}^{\hashtag}$ for edges.

    \begin{enumerate}
        \item If $\consistency_{X+Y}(\iota_{E_{X} + E_{Y}}(e)) \not = \varnothing$, then
                \begin{align*}
                    \consistency_{\sfrac{X+Y}{\sim}}^{\hashtag}(\iota_{\sfrac{E_{X} + E_{Y}}{\sim}}([e]_{E})) =
                    [[]_{V}^{\consistency}, []_{E}^{\consistency}]^{*}(\consistency_{X+Y}(\iota_{E_{X} + E_{Y}}(e)))
                \end{align*}
        \item $\consistency_{X+Y}(\iota_{E_{X} + E_{Y}}(e)) = \varnothing$ and there exists $v$ such that $\consistency_{X+Y}(\iota_{V_{X} + V_{Y}}(v)) \not = \varnothing$ and such that there is an undirected path from $[e]$ to $[v]$
        then
        \[
            \consistency_{\sfrac{X+Y}{\sim}}^{\hashtag}(\iota_{\sfrac{E_{X} + E_{Y}}{\sim}}([e]_{E})) = \consistency_{\sfrac{X+Y}{\sim}}^{\hashtag}(\iota_{V_{X} + V_{Y}}([v]))
        \]
    \end{enumerate}
    
    The well-definedness of this construction follows by the same argument as the well-definedness of $<_{\sfrac{X+Y}{\sim}}^{\mu}$.
    Then we define
    \ifdefined \ONECOLUMN
    \[
        \consistency_{\sfrac{X+Y}{\sim}}(\iota_{\sfrac{V_{X} + V_{Y}}{\sim}}([v])) \qquad \text{and} \qquad \consistency_{\sfrac{X+Y}{\sim}}(\iota_{\sfrac{E_{X} + E_{Y}}{\sim}}([e]))
    \]
    \else
    \[
        \consistency_{\sfrac{X+Y}{\sim}}(\iota_{\sfrac{V_{X} + V_{Y}}{\sim}}([v]))
    \]
    and
    \[
        \consistency_{\sfrac{X+Y}{\sim}}(\iota_{\sfrac{E_{X} + E_{Y}}{\sim}}([e]))
    \]
    \fi

    as closures of $\consistency^{\hashtag}_{\sfrac{X+Y}{\sim}}$ as below.
    \[
      (\consistency^{\hashtag}_{\sfrac{X+Y}{\sim}}(\iota_{\sfrac{*}{\sim}}([x])))^{c}
    \]
    where $c$ denotes a closure and $\iota_{\sfrac{*}{\sim}}$ is $\iota_{\sfrac{V_{X} + V_{Y}}{\sim}}$ or $\iota_{\sfrac{E_{X} + E_{Y}}{\sim}}$ depending on whether $[x_i]$ comes from $V_{X} + V_{Y}$ or $E_{X} + E_{Y}$, is the smallest set such that
    \begin{itemize}
        \item $[x] \in (\consistency^{\hashtag}_{\sfrac{X+Y}{\sim}}(\iota_{\sfrac{*}{\sim}}([x])))^{c}$ if $\consistency^{\hashtag}_{\sfrac{X+Y}{\sim}}(\iota_{\sfrac{*}{\sim}}([x])) \not = \varnothing$
        \item if $[y] \in \consistency^{\hashtag}_{\sfrac{X+Y}{\sim}}(\iota_{\sfrac{*}{\sim}}([x]))$ then 
        \[
            [x] \in (\consistency^{\hashtag}_{\sfrac{X+Y}{\sim}}(\iota_{\sfrac{*}{\sim}}([y])))^{c}
        \]
        % \item if $[e] \in \consistency^{\hashtag}_{\sfrac{X+Y}{\sim}}(\iota_{\sfrac{V_{X} + V_{Y}}{\sim}}([v]))$ then 
        % \[
        %     [v] \in (\consistency^{\hashtag}_{\sfrac{X+Y}{\sim}}(\iota_{\sfrac{E_{X} + E_{Y}}{\sim}}([e])))^{c}
        % \]
        \item for any sequence $([x_1], \ldots, [x_n])$ such that
        \ifdefined \ONECOLUMN
        \[[x_i] \in \consistency^{\hashtag}_{\sfrac{X+Y}{\sim}}(\iota_{\sfrac{*}{\sim}}([x_{i+1}])) \qquad \text{or} \qquad [x_{i+1}] \in \consistency^{\hashtag}_{\sfrac{X+Y}{\sim}}(\iota_{\sfrac{*}{\sim}}([x_{i}]))\]
        \else
        \[
            [x_i] \in \consistency^{\hashtag}_{\sfrac{X+Y}{\sim}}(\iota_{\sfrac{*}{\sim}}([x_{i+1}]))
        \] or 
        \[
            [x_{i+1}] \in \consistency^{\hashtag}_{\sfrac{X+Y}{\sim}}(\iota_{\sfrac{*}{\sim}}([x_{i}]))
        \]
        \fi
         for $i < n$ both
        \ifdefined \ONECOLUMN
        \[
            [x_1] \in (\consistency^{\hashtag}_{\sfrac{X+Y}{\sim}}(\iota_{\sfrac{*}{\sim}}([x_{n}])))^{c} \qquad \text{and} \qquad [x_n] \in (\consistency^{\hashtag}_{\sfrac{X+Y}{\sim}}(\iota_{\sfrac{*}{\sim}}([x_{1}])))^{c}
        \]
        \else
         \[
            [x_1] \in (\consistency^{\hashtag}_{\sfrac{X+Y}{\sim}}(\iota_{\sfrac{*}{\sim}}([x_{n}])))^{c}
        \]
        and
        \[
            [x_n] \in (\consistency^{\hashtag}_{\sfrac{X+Y}{\sim}}(\iota_{\sfrac{*}{\sim}}([x_{1}])))^{c}
        \]
        \fi
    \end{itemize}

    This construction is analogous to how a pushout was constructed in~\cite{ghica2024equivalencehypergraphsegraphsmonoidal} and we refer the reader for more detailed proofs there.
    The only addition is $\textcolor{blue}{<_{\sfrac{X+Y}{\sim}}}^{\mu}$ that is constructed analogously to $\textcolor{red}{<_{\sfrac{X+Y}{\sim}}}^{\mu}$.
\end{proof}
The construction can be illustrated with the example in Figure~\ref{fig:pushout_example}.
Vertices and edges that are being identified as they have a common preimage inherit the relations from the vertex or edge that has it defined either directly or via a path.
In particular, a coproduct of two feet contains\begin{tikzpicture}
	\begin{pgfonlayer}{nodelayer}
		\node [style=node, label={above:$v_1$}] (0) at (-5.5, 3.5) {};
		\node [style=node, label={above:$v_2$}] (1) at (-4.5, 3.5) {};
		\node [style=node, label={above:$v_3$}] (2) at (-3.5, 3.5) {};
		\node [style=node, label={above:$v'_1$}] (3) at (-2.5, 3.5) {};
		\node [style=node, label={above:$v'_2$}] (4) at (-1.5, 3.5) {};
		\node [style=node, label={above:$v'_3$}] (5) at (-0.5, 3.5) {};
	\end{pgfonlayer}
\end{tikzpicture}
such that $\textcolor{blue}{<_{X+Y}}$ is only defined for $v_1,v_2,v_3$.
Therefore, we assign $\textcolor{blue}{<_{\sfrac{X+Y}{\sim}}}(\iota_{\sfrac{V_{X} + V_{Y}}{\sim}}[v_1']) = [\textcolor{blue}{<_{X+Y}}(\iota_{V_{X} + V_{Y}}(v_1))]$ and so on.
This motivates the requirement for conditions (2), (3), and (4) of Theorem~\ref{th:existence_of_pushouts}, otherwise such inheritance would be ambiguous.

\begin{figure}[t]
\[
\adjustbox{scale=0.6}{
  \tikzfig{./figures/pushout_example}
}
\]
\caption{Pushout in $\catname{EHyp}(\Sigma)$}
\label{fig:pushout_example}
\end{figure}



\subsection{Interpretation of $\Sigma$-terms as cospans of hypergraphs}
\label{sec:appendix:interpretation}
\begin{figure*}
  \[
  \adjustbox{width=\textwidth}{
  \tikzfig{./figures/interpretation}
  }
  \]
  \caption{Base cases for $[-] : \textbf{SMT}(\Sigma) \to \MdaCospans$}
  \label{fig:base_cases}
\end{figure*}

The base cases for interpretation function $[-] : \textbf{SMT}(\Sigma) \to \MdaCospans$ is given in the Figure~\ref{fig:base_cases}.
Then it is defined inductively by letting $[f \otimes g] \Coloneqq [f] \otimes [g]$ and $[f;g] \Coloneqq [f];[g]$ and by freeness induces the corresponding functor.

\subsection{DPOI rewriting}
\label{sec:appendix:dpoi}

Consider an example of DPOI rewriting in Figure~\ref{fig:dpoi-example} as defined in Definition~\ref{def:dpoi-e}.
The rule in the top half of the diagram is an instance of a $\beta$-reduction rule as defined for string diagrams~\cite{ghica2024stringdiagramslambdacalculifunctional}.
Note that the match satisfies all the conditions of the boundary complement as all vertices from the interface have the same parent.
First, to compute the boundary complement, we remove from the source e-hypergraph everything that has no pre-image in the interface above.
The mda-cospan for the complement then will be
\[
[u_1] \to [u_1, v_1, v_2] + [v_3] \to \mathcal{L}^{\bot} \xleftarrow{} [v_3] + [v_1,v_2] \xleftarrow{} [u_2] ~.
\]
Note how we removed strict internal interfaces of $\mathcal{L}$ from internal interfaces of $\mathcal{G}$, yielding $[u_1,v_1,v_2]$ and similarly for output interfaces.
After computing the pushout for the second half of the diagram we add strict internal interfaces of $\mathcal{R}$ but since there are none, we are done as per condition (2) of Definition~\ref{def:dpoi-e}.

\begin{figure}
\[
\adjustbox{width=\linewidth}{
\tikzfig{./figures/dpo_example}
}
\]
\caption{$\beta$-rule DPOI example.}
\label{fig:dpoi-example}
\end{figure}