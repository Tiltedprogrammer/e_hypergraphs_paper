\section{$\catname{SLat}$}
\label{sec:appendix:1}

In this section we will define the category of semilattices that we use as a base for enrichment throughout the paper.

\begin{definition}[Semilattice]
    A \textit{semilattice} is a set equipped with an operation that we denote as $+$ which is associative, commutative and idempotent.
  \end{definition}
  
  Note that we do not require the existence of a unit for $+$. 
  Semilattices that satisfy this extra requirement are sometimes called \textit{bounded}, i.e., they are idempotent commutative monoids.
  
  \begin{definition}[Semilattice homomorphism]
  
  A homomorphism between two semilattices $S_{1}$ and $S_{2}$ is a map $h$ that respects $+$.
  That is, for all $s,s' \in S_{1}$, $h(s +_{S_{1}} s') = h(s) +_{S_{2}} h(s')$.
  \end{definition}
  
  \begin{definition}[Category of semilattices]
    
  Semilattices with their respective homomorphisms form a category that we denote $\catname{SLat}$.
  \end{definition}
  
  \begin{proposition}
    $\catname{SLat}$ is a closed symmetric monoidal category.
  \end{proposition}
  \begin{proof}
    The tensor product of two semilattices $S_{1}$ and $S_{2}$ is defined as follows.
    $S_{1} \otimes S_{2}$ consists of pairs $(s_1,s_2)$ $s_{1} \in S_{1}$, $s_{2} \in S_{2}$ quotiented by commutativity, idempotence and associativity and additionally by the following relations
    \begin{itemize}
      \item $(s_{1} +_{S_{1}} s_{1}',s_{2}) \equiv (s_{1},s_{2}) +_{S_{1} \otimes S_{2}} (s_{1}',s_{2})$
      \item $(s_{1}, s_{2} +_{S_{2}} s_{2}') \equiv (s_{1},s_{2}) +_{S_{1} \otimes S_{2}} (s_{1}',s_{2})$
    \end{itemize}
  
    The unit for this tensor product is $I = \{*\}$ --- a one-element semilattice.
    Clearly $S \otimes I \cong S$ by mapping $(s,*) \mapsto s$ for all $s \in S$ and vice versa.
    The symmetry, associators and unitors are then obvious morphisms.
    Finally, the category is closed since the set of homomorpisms between two semilattices is a semilattice by defining $(f + g)(x)$ as $f(x) + g(x)$.
    $f + g$ is a homomorphism since $(f + g)(x+y) = f(x+y) + g(x+y) = f(x) + f(y) + g(x) + g(y) = f(x) + g(x) + f(y) + g(y) = f(x+y) + g(x+y)$.
  \end{proof}
  
  This makes $\catname{SLat}$ a suitable base for enrichment.

  \begin{definition}
    Forgetful functor $U : \catname{SLat} \to \catname{Set}$ is given by $\catname{SLat}(I_{\catname{SLat}, -}) : \catname{SLat} \to \catname{Set}$.
    \end{definition}
    
    Intuitively, the above functor returns the underlying set of a given semilattice $S$ as each morphism from $\{*\} \to S$ picks out an element of $S$.
    
    \begin{proposition}[Special case of Proposition 6.4.6~\cite{Borceux_1994}]
      The forgetful functor $U : \catname{SLat} \to \catname{Set}$ has a left adjoint free functor $F : \catname{Set} \to \catname{SLat}$.
    \end{proposition}
    \begin{proof}
      The functor $F$ is defined by letting $F(A) = \coprod_{A} I_{\catname{SLat}}$ where the latter is a coproduct which is a `free' product of semilattices defined as follows.
      The elements of $S_{1} \coprod S_{2}$ are sequences $s_{1} + s_{2} + \ldots + s_{n}$ where each $s_{i}$ is either from $S_{1}$ or $S_{2}$ quotiented by all relations in $S_{1}$ and $S_{2}$.
      The adjunction is then given by the following natural isomorphisms
      \begin{align*}
      \catname{SLat}(\coprod_{A}(I_{\catname{SLat}}), B) &\cong \prod_{A}(\catname{SLat}(I_{\catname{SLat}}, B))\\
                                                         &\cong \catname{Set}(A,\catname{SLat}(I_{\catname{SLat}}, B))\\
                                                         &\cong \catname{Set}(A, U(B))
      \end{align*}
      The fist isomorphism is given by the fact that hom-functor makes limits into colimits in its first argument, the second being given by the property that $|\catname{Set}(A,B)| = |B|^{|A|} = |\underbrace{B \times \ldots \times B}_{|A|}|$, and the last isomorphism is given by the definition of $U$.
      Furthermore, we have, 
      \[
      F(I) \cong I_{\catname{SLat}}
      \]
      and 
      \begin{align*}
      F(A) \otimes F(B) &\cong (\coprod_{A} I_{\catname{SLat}}) \otimes (\coprod_{B} I_{\catname{SLat}})\\
            &\cong \coprod_{A} (I_{\catname{SLat}} \otimes \coprod_{B} I_{\catname{SLat}})\\
            &\cong \coprod_{A} (\coprod_{B} (I_{\catname{SLat}} \otimes I_{\catname{SLat}}))\\
            &\cong \coprod_{A} (\coprod_{B} I_{\catname{SLat}})\\
            &\cong \coprod_{A \times B} I_{\catname{SLat}}\\
            &\cong F(A \times B)
      \end{align*}
    In the above we used the fact that functors $- \otimes X$ and $X \otimes -$ preserve colimits as they are both left-adjoint by symmetry and monoidal closedness of $\catname{SLat}$.
    \end{proof}